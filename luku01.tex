\chapter{Johdanto}

Kurssin \emph{Tietorakenteet ja algoritmit} tarkoituksena
on opettaa menetelmiä, joiden avulla voimme ratkaista
\emph{tehokkaasti} laskennallisia ongelmia.
Ohjelmoinnin peruskursseilla olemme keskittyneet
ohjelmointitaidon opetteluun.
Nyt on aika siirtyä askel eteenpäin ja alkaa kiinnittää
huomiota myös siihen, miten nopeasti algoritmit toimivat.

Algoritmien tehokkuudella on suuri merkitys käytännössä.
Esimerkiksi netissä toimiva reittiopas on käyttökelpoinen sen vuoksi,
että se antaa ehdotuksen reitistä heti sen jälkeen, kun olemme
ilmoittaneet, mistä mihin haluamme matkustaa.
Jos reittiehdotusta pitäisi odottaa vaikkapa minuutti tai tunti,
tämä rajoittaisi paljon palvelun käyttöä.

Jotta reittiopas toimisi tehokkaasti, sen taustalla on
hyvin suunniteltu algoritmi.
Tällä kurssilla opimme, kuinka voimme luoda itse vastaavia algoritmeja.
Tutustumme kurssilla sekä algoritmien suunnittelun teoriaan että
käytäntöön -- haluamme ymmärtää syvällisesti, mistä algoritmeissa on kysymys,
mutta myös osata toteuttaa niitä käytännössä.

\section{Mitä algoritmit ovat?}

\index{algoritmi}
\index{syöte}
\index{tuloste}

\emph{Algoritmi} (\emph{algorithm}) on toimintaohje, jota seuraamalla voimme ratkaista
jonkin laskennallisen ongelman.
Algoritmille annetaan \emph{syöte} (\emph{input}),
joka kuvaa ratkaistavan ongelman tapauksen,
ja algoritmin tulee tuottaa \emph{tuloste} (\emph{output}),
joka on vastaus sille annettuun syötteeseen.

Tarkastellaan esimerkkinä ongelmaa,
jossa syötteenä on $n$ kokonaislukua sisältävä lista ja
tehtävänä on laskea lukujen summa.
Esimerkiksi jos syöte on $[2,4,1,8]$,
haluttu tuloste on 15, koska $2+4+1+8=15$.
Voimme ratkaista tämän ongelman algoritmilla,
joka käy luvut läpi silmukalla ja laskee niiden
summan muuttujaan.

Algoritmin toiminnan esittämiseen on useita mahdollisuuksia.
Yksi tapa on selostaa sanallisesti, kuinka algoritmi toimii,
kuten teimme äsken.
Toinen tapa taas on antaa jollain ohjelmointikielellä
kirjoitettu koodi, joka toteuttaa algoritmin.
Esimerkiksi voimme toteuttaa lukujen summan laskevan
algoritmin näin Java- ja Python-kielillä:

\begin{code}
int summa = 0;
for (int i = 0; i < n; i++) {
    summa += luvut[i];
}
System.out.println(summa);
\end{code}

\begin{code}
summa = 0
for i in range(n):
    summa += luvut[i]
print(summa)
\end{code}

\index{pseudokoodi}

Voimme myös esittää algoritmin \emph{pseudokoodina}
todellisen ohjelmointikielen sijasta.
Tämä tarkoittaa, että kirjoitamme koodia,
joka on lähellä käytössä olevia ohjelmointikieliä, mutta voimme
päättää koodin tarkan kirjoitusasun itse ja ottaa joitakin vapauksia,
joiden ansiosta voimme kuvata algoritmin mukavammin.
Voisimme esimerkiksi esittää äskeisen algoritmin pseudokoodina seuraavasti:

\begin{code}
summa = 0
for i = 0 to n-1
    summa += luvut[i]
print(summa)
\end{code}

Tässä kirjassa esitämme algoritmeja pseudokoodin avulla,
koska kurssi käsittelee yleistä ohjelmoinnin teoriaa,
joka ei liity tiettyyn ohjelmointikieleen.
Tämän lisäksi tutustumme siihen, miten tietyt algoritmit
ja tietorakenteet on toteutettu Javan ja Pythonin
standardikirjastossa.

\section{Ohjelmoinnin peruspalikat}

Kiehtova seikka ohjelmoinnissa on, että monimutkaisetkin algoritmit
syntyvät yksinkertaisista aineksista.
Käymme seuraavaksi läpi ohjelmoinnin peruspalikat,
jotka muodostavat pohjan algoritmien suunnittelulle.
Tutustumme myös samalla tarkemmin kirjassa käytettyyn pseudokoodiin.

\subsubsection{Muuttuja}

\index{muuttuja}
\emph{Muuttuja} (\emph{variable}) sisältää algoritmin
käsittelemää tietoa.
Pseudokoodissa käytäntönä on, että voimme käyttää muuttujia
suoraan ilman erillistä määrittelyä:

\begin{code}
a = 5
b = 7
c = a+b
\end{code}

\subsubsection{Ehtolause}

\index{ehtolause}
\emph{Ehtolause} (\emph{conditional statement}) saa ohjelman
toiminnan haarautumaan.
Käytämme pseudokoodissa seuraavaa syntaksia:

\begin{code}
if x%2 == 0
    print("parillinen")
else
    print("pariton")
\end{code}

\subsubsection{Silmukka}

\index{silmukka}
\emph{Silmukka} (\emph{loop}) toistaa sen sisällä olevaa koodia.
Merkitsemme pseudokoodissa seuraavasti for-silmukan,
joka käy läpi luvut väliltä 1--100:

\begin{code}
for i = 1 to 100
    print(i)
\end{code}

Seuraava for-silmukka käy puolestaan läpi listassa olevat alkiot:

\begin{code}
for x in lista
    print(x)
\end{code}

Toinen silmukkatyyppi on while-silmukka, joka toistaa koodia niin kauan
kuin silmukan alussa annettu ehto on voimassa.
Esimerkiksi seuraava silmukka jatkuu niin kauan kuin $x$:n arvo on ainakin 1:

\begin{code}
while x >= 1
    print(x)
    x /= 2
\end{code}

\subsubsection{Taulukko}

\index{taulukko}
\index{indeksi}

Ohjelmoinnin perustana oleva tietorakenne on \emph{taulukko} (\emph{array}),
joka sisältää kokoelman peräkkäin olevia alkioita.
Voimme viitata taulukon alkioihin \texttt{[]}-syntaksilla:

\begin{code}
luvut[0] = 4
luvut[3] = 2
\end{code}

Noudatamme kirjassa yleistä käytäntöä, jossa taulukon alkiot on indeksoitu
$0,1,\dots,n-1$, kun taulukossa on $n$ alkiota.

Taulukon ominaisuuksia ovat, että siinä on kiinteä määrä alkioita ja
pääsemme tehokkaasti käsiksi mihin tahansa alkioon indeksin perusteella.
Taulukko vastaa tiedon tallennustapaa tietokoneen muistissa,
ja voimme toteuttaa taulukon avulla minkä tahansa muun tietorakenteen.

\index{lista}

Taulukko on perustietorakenne esimerkiksi Javassa,
mutta monissa nykyään suosituissa kielissä (kuten Pythonissa)
taulukon sijasta perustietorakenne on \emph{lista},
jota voi käyttää taulukon tavoin mutta jonka koko voi muuttua.
Yleensä lista on kuitenkin toteutettu sisäisesti taulukon avulla,
ja luvussa 4 näemme, miten tämä tapahtuu käytännössä.

\subsubsection{Aliohjelmat}

\index{aliohjelma}
\index{proseduuri}
\index{funktio}

Käytämme kirjassa termiä \emph{proseduuri} (\emph{procedure})
aliohjelmasta, jolla ei ole palautusarvoa,
ja termiä \emph{funktio} (\emph{function}) aliohjelmasta, jolla on palautusarvo.
Esimerkiksi seuraava proseduuri tulostaa lukuja:

\begin{code}
procedure tulosta(n)
    for i = 1 to n
        print(i)
\end{code}

Seuraava funktio puolestaan laskee lukujen summan:

\begin{code}
function summa(n)
    s = 0
    for i = 1 to n
        s += i
    return s
\end{code}

Aliohjelmiin liittyvä termistö vaihtelee ohjelmointikielen mukaan:
esimerkiksi Javassa käytetään molemmissa tapauksissa termiä
\emph{metodi} ja Pythonissa käytetään termiä \emph{funktio} tai
\emph{metodi} tilanteesta riippuen.

\subsubsection{Rekursio}

\index{rekursio}

Usein algoritmien suunnittelussa esiintyvä ohjelmointitekniikka on
\emph{rekursio} (\emph{recursion}),
joka tarkoittaa, että aliohjelma kutsuu itseään.
Tässä on yksi esimerkki rekursion käyttämisestä:

\begin{code}
procedure testi(n)
    if n == 0
        return
    else
        print("moikka")
        testi(n-1)
\end{code}

Tämä proseduuri tulostaa $n$ kertaa rivin "moikka".
Esimerkiksi kutsu \texttt{testi}(3) saa aikaan seuraavan tulostuksen:

\begin{code}
moikka
moikka
moikka
\end{code}

Ideana on, että jos $n=0$, proseduuri ei tee mitään,
koska ei ole mitään tulostettavaa.
Muussa tapauksessa proseduuri tulostaa rivin "moikka"
ja kutsuu sitten itseään parametrilla $n-1$.

Käytämme rekursiota usein kurssin aikana,
ja tutustumme pikkuhiljaa tarkemmin sen mahdollisuuksiin.

\subsubsection{***}

Olemme nyt käyneet läpi ainekset,
joiden avulla voimme toteuttaa \emph{minkä tahansa} algoritmin.
On huojentava tieto, että näinkin pieni määrä tekniikoita
riittää algoritmien suunnittelussa.
Nyt kaikki on vain kiinni siitä, miten osaamme \emph{soveltaa}
näitä tekniikoita eri tilanteissa.
