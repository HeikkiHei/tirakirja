\chapter{Lyhimmät polut}

Lyhimmän polun etsiminen on keskeinen verkko-ongelma,
jolle on helppoa löytää käytännön sovelluksia.
Esimerkiksi voimme haluta selvittää,
mikä on nopein reitti kahden katuosoitteen välillä
tai mikä on halvin tapa lentää kaupungista toiseen.
Näissä ja muissa sovelluksissa on tärkeää,
että algoritmit toimivat mahdollisimman nopeasti.

Olemme käyttäneet aiemmin leveyshakua lyhimmän
polun etsimiseen.
Tämä onkin hyvä ratkaisu silloin, kun haluamme löytää polun,
jonka kaarten määrä on pienin.
Tässä luvussa keskitymme kuitenkin vaikeampaan
tilanteeseen, jossa verkko on \emph{painotettu}
eli jokaiseen kaareen liittyy paino, ja haluamme löytää polun,
jossa painojen summa on pienin.
Kuvassa \ref{fig:verpai} on esimerkkinä painotettu verkko,
jossa jokaisen kaaren vieressä on ilmoitettu sen paino.
Tässä verkossa lyhin polku solmusta $1$ solmuun $5$ on
$1 \rightarrow 3 \rightarrow 4 \rightarrow 5$,
jonka pituus on $2+4+3=9$.

Tutustumme tässä luvussa ensin Bellman–Fordin algoritmiin,
joka etsii lyhimmät polut annetusta lähtösolmusta
kaikkiin verkon solmuihin.
Tämän jälkeen siirrymme Dijkstran algoritmiin,
joka tekee saman tehokkaammin olettaen, että verkossa
ei ole negatiivisen painoisia kaaria.
Lopuksi käsitte\-lemme Floyd–Warshallin algoritmin,
joka etsii samanaikaisesti lyhimmät polut kaikkien
verkon solmujen välillä.

\begin{figure}
\center
\begin{center}
\begin{tikzpicture}[scale=0.6,label distance=-1.5mm]
\small
\node[draw, circle] (1) at (1,3) {$1$};
\node[draw, circle] (2) at (4,3) {$2$};
\node[draw, circle] (3) at (1,1) {$3$};
\node[draw, circle] (4) at (4,1) {$4$};
\node[draw, circle] (5) at (6,2) {$5$};

\path[draw,thick,->] (1) -- node[font=\small,label=above:5] {} (2);
\path[draw,thick,->] (1) -- node[font=\small,label=left:2] {} (3);
\path[draw,thick,->] (1) -- node[font=\small,label=above:9] {} (4);
\path[draw,thick,->] (3) -- node[font=\small,label=below:4] {} (4);
\path[draw,thick,->] (2) -- node[font=\small,label=above:8] {} (5);
\path[draw,thick,->] (4) -- node[font=\small,label=right:2] {} (2);
\path[draw,thick,->] (4) -- node[font=\small,label=below:3] {} (5);
\end{tikzpicture}
\end{center}
\caption{Painotettu verkko.}
\label{fig:verpai}
\end{figure}

\section{Bellman–Fordin algoritmi}

Bellman–Fordin algoritmi etsii lyhimmät polut
annetusta lähtösolmusta kaikkiin verkon solmuihin.
Algoritmi muodostaa taulukon, joka kertoo jokaiselle
solmulle sen \emph{etäisyyden} eli lyhimmän polun pituuden lähtösolmusta.
Algoritmi toimii missä tahansa verkossa,
kunhan verkossa ei ole negatiivista sykliä eli sykliä,
jonka painojen summa on negatiivinen.

Bellman–Fordin algoritmi pitää yllä \emph{arvioita}
solmujen etäisyyksistä niin,
että aluksi etäisyys lähtösolmuun on 0 ja etäisyys
kaikkiin muihin solmuihin on ääretön.
Tämän jälkeen algoritmi alkaa parantaa etäisyyksiä
etsimällä verkosta kaaria, joiden kautta kulkeminen
lyhentää polkuja.
Jokaisessa vaiheessa algoritmi etsii kaaren $a \rightarrow b$,
jolle pätee, että pääsemme solmuun $b$ lyhempää polkua
kulkemalla kaarella solmusta $a$.
Kun mitään etäisyyttä ei voi enää parantaa,
algoritmi päättyy ja kaikki etäisyydet vastaavat
todellisia lyhimpien polkujen pituuksia.

\begin{figure}
\center
\begin{center}
\begin{tikzpicture}[scale=0.6,label distance=-1.5mm]
\scriptsize
\newcommand\verkko[6]{
\node[draw, circle] (1) at (0,-1) {$1$};
\node[draw, circle] (2) at (2,0) {$2$};
\node[draw, circle] (3) at (2,-2) {$3$};
\node[draw, circle] (4) at (4,0) {$4$};
\node[draw, circle] (5) at (4,-2) {$5$};
\path[draw,thick,->] (1) -- node[font=\small,label=above:8] {} (2);
\path[draw,thick,->] (1) -- node[font=\small,label=below:2] {} (3);
\path[draw,thick,->] (3) -- node[font=\small,label=right:4] {} (2);
\path[draw,thick,->] (2) -- node[font=\small,label=above:5] {} (4);
\path[draw,thick,->] (3) -- node[font=\small,label=below:7] {} (5);
\path[draw,thick,->] (5) -- node[font=\small,label=right:3] {} (4);
\node[color=red] at (0,-0.25) {$#2$};
\node[color=red] at (2,0.75) {$#3$};
\node[color=red] at (2,-2.75) {$#4$};
\node[color=red] at (4,0.75) {$#5$};
\node[color=red] at (4,-2.75) {$#6$};
\node at (2,-3.5) {vaihe #1};
}
\begin{scope}
\verkko{1}{0}{\infty}{\infty}{\infty}{\infty}
\end{scope}
\begin{scope}[xshift=6.5cm]
\verkko{2}{0}{8}{\infty}{\infty}{\infty}
\path[draw=red,thick,->,line width=1.5pt] (1) -- (2);
\end{scope}
\begin{scope}[xshift=13cm]
\verkko{3}{0}{8}{2}{\infty}{\infty}
\path[draw=red,thick,->,line width=1.5pt] (1) -- (3);
\end{scope}
\begin{scope}[yshift=-5.5cm]
\verkko{4}{0}{8}{2}{13}{\infty}
\path[draw=red,thick,->,line width=1.5pt] (2) -- (4);
\end{scope}
\begin{scope}[yshift=-5.5cm,xshift=6.5cm]
\verkko{5}{0}{6}{2}{13}{\infty}
\path[draw=red,thick,->,line width=1.5pt] (3) -- (2);
\end{scope}
\begin{scope}[yshift=-5.5cm,xshift=13cm]
\verkko{6}{0}{6}{2}{13}{9}
\path[draw=red,thick,->,line width=1.5pt] (3) -- (5);
\end{scope}
\begin{scope}[yshift=-11cm]
\verkko{7}{0}{6}{2}{12}{9}
\path[draw=red,thick,->,line width=1.5pt] (5) -- (4);
\end{scope}
\begin{scope}[yshift=-11cm,xshift=6.5cm]
\verkko{8}{0}{6}{2}{11}{9}
\path[draw=red,thick,->,line width=1.5pt] (2) -- (4);
\end{scope}
\end{tikzpicture}
\end{center}
\caption{Esimerkki Bellman–Fordin algoritmin toiminnasta.}
\label{fig:belfor}
\end{figure}

Kuva \ref{fig:belfor} näyttää esimerkin Bellman–Fordin algoritmin toiminnasta,
kun lähtösolmuna on solmu $1$.
Jokaisen solmun vieressä on ilmoitettu sen etäisyysarvio:
aluksi etäisyys solmuun 1 on 0 ja etäisyys kaikkiin muihin solmuihin on ääretön.
Jokainen etäisyyden muutos näkyy kuvassa omana vaiheenaan.
Ensin parannamme etäisyyttä solmuun 2
kulkemalla kaarta $1 \rightarrow 2$,
jolloin etäisyydeksi tulee $8$.
Sitten parannamme etäisyyttä solmuun $3$
kulkemalla kaarta $1 \rightarrow 3$,
jolloin solmun uudeksi etäisyydeksi tulee $2$.
Jatkamme samalla tavalla, kunnes emme voi enää parantaa
mitään etäisyyttä, ja vaiheen $8$ jälkeen kaikki etäisyydet
vastaavat lyhimpien polkujen pituuksia.
Esimerkiksi solmun 4 etäisyys on lopuksi 11,
koska lyhin polku solmusta $1$ solmuun $4$ on
$1 \rightarrow 3 \rightarrow 2 \rightarrow 4$,
jonka pituus on $2+4+5=11$.

\subsection{Algoritmin toteutus}

Bellman–Fordin algoritmi on kätevää toteuttaa niin,
että säilytämme verkon kaaria \emph{kaarilistassa}.
Jokainen listalla oleva olio sisältää tiedon
yhdestä verkon kaaresta.
Voimme määritellä kaarilistan näin:

\begin{code}
ArrayList<Kaari> kaaret = new ArrayList<>();
\end{code}

Oletamme, että listan oliot ovat seuraavanlaisia:

\begin{code}
public class Kaari {
    public int alku, loppu, paino;

    public Kaari(int alku, int loppu, int paino) {
        this.alku = alku;
        this.loppu = loppu;
        this.paino = paino;
    }
}
\end{code}

Lisäksi määrittelemme taulukon,
jossa on solmujen etäisyyksiä:

\begin{code}
int[] etaisyys = new int[n+1];
\end{code}

Seuraava koodi alustaa taulukon, kun $x$ kuvaa lähtösolmua.
Tässä \texttt{INF} on vakio, joka edustaa ääretöntä.
Käytännössä tämä on jokin suuri luku,
joka on suurempi, kuin mikä tahansa todellinen etäisyys.

\begin{code}
for (int i = 1; i <= n; i++) {
    etaisyys[i] = INF;
}
etaisyys[x] = 0;
\end{code}

Nyt olemme valmiita toteuttamaan itse algoritmin.
Toteutamme algoritmin niin, että se käy joka kierroksella
läpi kaikki verkon kaaret.
Jos pystymme parantamaan etäisyyttä kaaren avulla,
algoritmi merkitsee muistiin uuden etäisyyden sekä
myös tiedon siitä, että olemme muuttaneet etäisyyttä.
Algoritmi jatkuu niin kauan, kunnes tulee kierros,
jonka aikana mikään etäisyys ei muutu,
jolloin etäisyydet ovat lopulliset.

\begin{code}
while (true) {
    boolean muutos = false;
    for (Kaari kaari : kaaret) {
        int vanha = etaisyys[kaari.loppu];
        int uusi = etaisyys[kaari.alku]+kaari.paino;
        if (uusi < vanha) {
            muutos = true;
            etaisyys[kaari.loppu] = uusi;
        }
    }
    if (!muutos) break;
}
\end{code}


\subsection{Algoritmin analyysi}

Tärkeitä kysymyksiä Bellman–Fordin algoritmiin liittyen ovat,
miksi algoritmi toimii ja miten tehokas se on.
Jotta voimme vastata näihin kysymyksiin,
tarvitsemme kaksi havaintoa verkon lyhimmistä poluista.

Ensimmäinen havainto on, että jos lyhin polku solmusta $s_1$ solmuun $s_k$ on
$s_1 \rightarrow s_2 \rightarrow \dots \rightarrow s_k$,
niin myös lyhin polku solmusta $s_1$ solmuun $s_2$ on $s_1 \rightarrow s_2$,
lyhin polku solmusta $s_1$ solmuun $s_3$ on $s_1 \rightarrow s_2 \rightarrow s_3$, jne.,
eli jokainen polun alkuosa on myös lyhin polku vastaavaan solmuun.
Jos näin ei olisi, voisimme parantaa lyhintä polkua solmusta $s_1$ solmuun $s_k$
parantamalla jotain polun alkuosaa, mikä aiheuttaisi ristiriidan.

Toinen havainto on,
että $n$ solmun verkossa jokainen lyhin polku voi
sisältää enintään $n-1$ kaarta,
kun oletamme, että verkossa ei ole negatiivista sykliä.
Jos polkuun kuuluisi $n$ tai enemmän kaaria,
jokin solmu esiintyisi polulla monta kertaa.
Tämä ei ole kuitenkaan mahdollista,
koska ei olisi järkeä kulkea monta kertaa saman solmun kautta,
kun haluamme saada aikaan lyhimmän polun.

Tarkastellaan nyt, mitä tapahtuu algoritmin kierroksissa.
Ensimmäisen kierroksen jälkeen olemme löytäneet lyhimmät polut,
joissa on enintään yksi kaari.
Toisen kierroksen jälkeen olemme löytäneet lyhimmät polut,
joissa on enintään kaksi kaarta.
Sama jatkuu, kunnes $n-1$ kierroksen jälkeen olemme löytäneet
lyhimmät polut, joissa on enintään $n-1$ kaarta.
Koska missään lyhimmässä polussa ei voi olla enempää kaaria,
olemme löytäneet kaikki lyhimmät polut.
Algoritmi suorittaa siis enintään $n-1$ kierrosta,
joista jokainen käy läpi kaikki verkon kaaret ajassa $O(m)$.
Niinpä algoritmi löytää lyhimmät polut ajassa $O(nm)$.

\begin{figure}
\center
\begin{center}
\begin{tikzpicture}[scale=0.7,label distance=-1.5mm]
\node[draw, circle] (1) at (1,3) {$1$};
\node[draw, circle] (2) at (4,3) {$2$};
\node[draw, circle] (3) at (1,1) {$3$};
\node[draw, circle] (4) at (4,1) {$4$};
\node[draw, circle] (5) at (6,2) {$5$};

\path[draw,thick,->] (1) -- node[font=\small,label=above:1] {} (2);
\path[draw,thick,->] (1) -- node[font=\small,label=left:3] {} (3);
\path[draw,thick,<-] (3) -- node[font=\small,label=below:4] {} (4);
\path[draw,thick,->] (2) -- node[font=\small,label=left:$-7$] {} (4);
\path[draw,thick,<-] (2) -- node[font=\small,label=above:2] {} (5);
\path[draw,thick,->] (4) -- node[font=\small,label=below:3] {} (5);
\end{tikzpicture}
\end{center}
\caption{Negatiivinen sykli $2 \rightarrow 4 \rightarrow 5 \rightarrow 2$,
jonka avulla voimme lyhentää polkuja loputtomasti.}
\label{fig:belsyk}
\end{figure}

Mitä tapahtuu sitten, jos verkossa on negatiivinen sykli?
Esimerkiksi kuvan \ref{fig:belsyk} verkossa on negatiivinen sykli
$2 \rightarrow 4 \rightarrow 5 \rightarrow 2$, jonka paino on $-2$.
Tässä tilanteessa Bellman–Fordin algoritmi jää jumiin, koska se pystyy parantamaan
loputtomasti syklin kautta kulkevia polkuja.
Oikeastaan ongelma on siinä, että lyhin polku ei ole mielekäs käsite,
jos polun osana on negatiivinen sykli.
Voimme kuitenkin havaita negatiivisen syklin siitä,
että jokin etäisyys paranee vielä $n-1$ kierroksen jälkeen.

\section{Dijkstran algoritmi}

Dijkstran algoritmi on Bellman–Fordin algoritmin tehostettu versio,
jonka toiminta perustuu oletukseen, että verkossa ei ole
negatiivisen painoisia kaaria.
Bellman–Fordin algoritmin tapaan Dijkstran algoritmi pitää
yllä arvioita etäisyyksistä lähtösolmusta $x$ muihin solmuihin.
Erona on kuitenkin tapa, miten Dijkstran algoritmi parantaa etäisyyksiä.

Algoritmi etsii jokaisessa vaiheessa solmun, jota ei ole vielä
käsitelty ja jonka etäisyys on pienin.
Sitten algoritmi käy läpi kaikki solmusta lähtevät kaaret ja
koettaa parantaa etäisyyksiä niiden avulla.
Tämän jälkeen solmu on käsitelty eikä sen etäisyys enää
muutu algoritmin aikana.

\begin{figure}
\center
\begin{center}
\begin{tikzpicture}[scale=0.7,label distance=-1.5mm]
\small
\newcommand\verkko[6]{
\node[draw, circle] (1) at (0,-1) {$1$};
\node[draw, circle] (2) at (2,0) {$2$};
\node[draw, circle] (3) at (2,-2) {$3$};
\node[draw, circle] (4) at (4,0) {$4$};
\node[draw, circle] (5) at (4,-2) {$5$};
\path[draw,thick,->] (1) -- node[font=\small,label=above:8] {} (2);
\path[draw,thick,->] (1) -- node[font=\small,label=below:2] {} (3);
\path[draw,thick,->] (3) -- node[font=\small,label=right:4] {} (2);
\path[draw,thick,->] (2) -- node[font=\small,label=above:5] {} (4);
\path[draw,thick,->] (3) -- node[font=\small,label=below:7] {} (5);
\path[draw,thick,->] (5) -- node[font=\small,label=right:3] {} (4);
\node[color=red] at (0,-0.25) {$#2$};
\node[color=red] at (2,0.75) {$#3$};
\node[color=red] at (2,-2.75) {$#4$};
\node[color=red] at (4,0.75) {$#5$};
\node[color=red] at (4,-2.75) {$#6$};
\node at (2,-3.5) {vaihe #1};
}
\begin{scope}
\verkko{1}{0}{\infty}{\infty}{\infty}{\infty}
\end{scope}
\begin{scope}[xshift=6.5cm]
\node[draw, circle, fill=lightgray] (1) at (0,-1) {$1$};
\verkko{2}{0}{8}{2}{\infty}{\infty}
\path[draw=red,thick,->,line width=2pt] (1) -- (2);
\path[draw=red,thick,->,line width=2pt] (1) -- (3);
\end{scope}
\begin{scope}[xshift=13cm]
\node[draw, circle, fill=lightgray] (1) at (0,-1) {$1$};
\node[draw, circle, fill=lightgray] (3) at (2,-2) {$3$};
\verkko{3}{0}{6}{2}{\infty}{9}
\path[draw=red,thick,->,line width=2pt] (3) -- (2);
\path[draw=red,thick,->,line width=2pt] (3) -- (5);
\end{scope}
\begin{scope}[yshift=-5.5cm]
\node[draw, circle, fill=lightgray] (1) at (0,-1) {$1$};
\node[draw, circle, fill=lightgray] (3) at (2,-2) {$3$};
\node[draw, circle, fill=lightgray] (2) at (2,0) {$2$};
\verkko{4}{0}{6}{2}{11}{9}
\path[draw=red,thick,->,line width=2pt] (2) -- (4);
\end{scope}
\begin{scope}[yshift=-5.5cm,xshift=6.5cm]
\node[draw, circle, fill=lightgray] (1) at (0,-1) {$1$};
\node[draw, circle, fill=lightgray] (3) at (2,-2) {$3$};
\node[draw, circle, fill=lightgray] (2) at (2,0) {$2$};
\node[draw, circle, fill=lightgray] (5) at (4,-2) {$5$};
\verkko{5}{0}{6}{2}{11}{9}
\end{scope}
\begin{scope}[yshift=-5.5cm,xshift=13cm]
\node[draw, circle, fill=lightgray] (1) at (0,-1) {$1$};
\node[draw, circle, fill=lightgray] (3) at (2,-2) {$3$};
\node[draw, circle, fill=lightgray] (2) at (2,0) {$2$};
\node[draw, circle, fill=lightgray] (5) at (4,-2) {$5$};
\node[draw, circle, fill=lightgray] (4) at (4,0) {$4$};
\verkko{6}{0}{6}{2}{11}{9}
\end{scope}
\end{tikzpicture}
\end{center}
\caption{Esimerkki Dijkstran algoritmin toiminnasta.}
\label{fig:dijalg}
\end{figure}

Kuva \ref{fig:dijalg} näyttää esimerkin Dijkstran algoritmin
toiminnasta.
Solmun harmaa väri tarkoittaa, että se on käsitelty.
Alussa valitsemme solmun 1, koska sen etäisyys 0 on pienin.
Sitten valittavana ovat solmut 2, 3, 4 ja 5,
joista valitsemme solmun 3, jonka etäisyys 2 on pienin.
Sama jatkuu, kunnes olemme käsitelleet kaikki verkon solmut.

Huomaa, että aina kun olemme käsitelleet solmun,
olemme saaneet selville sen lopullisen etäisyyden.
Esimerkiksi vaiheen 3 jälkeen tiedämme,
että lyhin polku solmuun 3 on pituudeltaan 2,
vaikka emme ole vielä käsitelleet kaikkia verkon solmuja.

\subsection{Analyysi}

Dijkstran algoritmiin liittyy kaksi tärkeää kysymystä:
miksi algoritmi toimii ja kuinka voimme toteuttaa sen tehokkaasti?

Algoritmin toiminnassa on keskeisessä osassa oletuksemme,
että minkään kaaren paino ei voi olla negatiivinen.
Tämän oletuksen ansiosta voimme aina käsitellä seuraavaksi solmun,
jonka etäisyys on pienin, ja olla varmoja siitä, että sen nykyinen
etäisyys on pienin mahdollinen.

Asiaa voi ajatella käänteisesti: jos pienimmän etäisyyden solmun
etäisyys ei olisi lopullinen, meidän tulisi pystyä parantamaan
sitä kulkemalla solmuun jotain toista kautta.
Kuitenkin kaikkialla muualla on vain solmuja, joiden etäisyydet
ovat suurempia tai yhtä suuria eivätkä etäisyydet voi lyhentyä,
koska verkossa ei ole negatiivisia kaaria.
Tästä syystä voimme turvallisesti valita pienimmän etäisyyden
solmun ja käsitellä sen.

\begin{figure}
\center
\begin{center}
\begin{tikzpicture}[scale=0.7,label distance=-1.5mm]
\node[draw, circle] (1) at (0,1) {$1$};
\node[draw, circle] (2) at (2.5,2) {$2$};
\node[draw, circle] (3) at (2.5,0) {$3$};
\node[draw, circle] (4) at (5,1) {$4$};

\path[draw,thick,->] (1) -- node[font=\small,label=above:5] {} (2);
\path[draw,thick,->] (1) -- node[font=\small,label=below:7] {} (3);
\path[draw,thick,->] (2) -- node[font=\small,label=above:3] {} (4);
\path[draw,thick,->] (3) -- node[font=\small,label=below:$-4$] {} (4);
\end{tikzpicture}
\end{center}
\caption{Dijkstran algoritmi ei toimi oikein tässä verkossa
negatiivisen kaaren takia.}
\label{fig:dijneg}
\end{figure}

Jos verkossa on negatiivinen kaari,
Dijkstran algoritmi ei toimi välttä\-mättä oikein.
Kuva \ref{fig:dijneg} näyttää esimerkin tällaisesta verkosta.
Dijkstran algoritmi seuraa ahneesti ylempää polkua ja toteaa,
että pienin etäisyys solmusta 1 solmuun 4 on 8.
Kuitenkin parempi tapa olisi kulkea alempaa polkua,
jolloin negatiivisen kaaren asiosta pienin etäisyys on vain 3.

Jotta voimme toteuttaa Dijkstran algoritmin tehokkaasti,
meidän täytyy pystyä löytämään nopeasti seuraavaksi käsiteltävä
solmu kussakin vaiheessa.
Tarvitsemme siis tietorakenteen, jossa voimme säilyttää
solmuja järjestykses\-sä niiden etäisyyden mukaan.
Sopivia tietorakenteita tähän ovat binääri\-hakupuu ja keko,
joita käyttäen pystymme päivittämään etäisyyttä ja löytämään
pienimmän etäisyyden $O(\log n)$-ajassa.

Algoritmin ajankäyttö muodostuu kahdesta osasta:
meidän tulee löytää $n$ kertaa pienimmän etäisyyden solmu,
ja lisäksi etäisyydet saattavat muuttua $m$ kertaa
kunkin kaaren käsittelyn yhteydessä.
Niinpä algoritmin aikavaativuudeksi tulee $O((n+m) \log n)$.

\subsection{Toteutus}

Dijkstran algoritmi on mukavaa toteuttaa niin, että
se käyttää verkon vieruslistaesitystä

\begin{code}
ArrayList<Kaari>[] verkko = new ArrayList<>();
\end{code}

Luomme seuraavat taulukot, jotka pitävät kirjaa solmujen etäisyyksistä
sekä käsitellyistä solmuista:

\begin{code}
int[] etaisyys = new int[n+1];
boolean[] kasitelty = new boolean[n+1];
\end{code}

Lisäksi jotta voimme löytää nopeasti pienimmän etäisyyden solmun,
otamme käyttöön prioriteettijonon

\begin{code}
PriorityQueue<Solmu> solmut = new PriorityQueue<>();
\end{code}

Etäisyyksiä varten tarvitsemme uuden luokan,
jonka oliot järjestyvät etäisyyden mukaan pienimmästä suurimpaan:

\begin{code}
class Solmu implements Comparable<Solmu> {
    public int id, etaisyys;

    public int compareTo(Solmu s) {
        if (etaisyys != s.etaisyys) return etaisyys-s.etaisyys;
        else return id-s.id;
    }

    public boolean equals(Object o) {
        Solmu s = (Solmu)o;
        return etaisyys == s.etaisyys && id == s.id;
    }

    public Solmu(int id, int etaisyys) {
        this.id = id;
        this.etaisyys = etaisyys;
    }
}
\end{code}

Haluamme selvittää lyhimmät polut solmusta $x$ alkaen,
joten alustamme rakenteet seuraavasti:

\begin{code}
for (int i = 1; i <= n; i++) {
    etaisyys[i] = 1e9;
}
etaisyys[x] = 0;
solmut.add(new Solmu(x,0));
\end{code}

Tämän jälkeen voimme toteuttaa algoritmin näin:

\begin{code}
while (!solmut.isEmpty()) {
    Solmu solmu = solmut.poll();
    if (kasitelty[solmu.id]) continue;
    kasitelty[solmu.id] = true;
    for (Kaari kaari : verkko[solmu]) {
        int vanha = etaisyys[kaari.kohde];
        int uusi = solmu.etaisyys+kaari.paino;
        if (uusi < vanha) {
            etaisyys[kaari.kohde] = uusi;
            solmut.add(new Solmu(kaari.kohde,uusi))
        }
    }
}
\end{code}

Huomaa, että koska voimme poistaa prioriteettijonosta
vain pienimmän etäisyyden solmun,
lisäämme solmun etäisyyden muuttuessa uuden solmun jonoon.
Jos olemme käsitelleet jonosta tulevan solmun,
emme käsittele sitä uudestaan.

Vaikka jonossa on ''ylimääräisiä'' solmuja, ratkaisu säilyy
tehokkaana ja toimii ajassa $O((n+m) \log n)$.
Jonossa voi olla enintään $m$ solmua, ja koska $m=O(n^2)$,
jokainen operaatio jonoon vie aikaa $O(\log (n^2))=O(2 \log n)=O(\log n)$.

\section{Floyd–Warshallin algoritmi}

Floyd-Warshallin algoritmi on idealtaan erilainen kuin
Bellman–Fordin ja Dijkstran algoritmit:
se laskee samalla kertaa kaikki mahdolliset
solmujen välimatkat verkossa.
Algoritmin lopputuloksena on etäisyysmatriisi,
josta voimme lukea lyhimmän polun pituuden mistä tahansa solmusta $a$
mihin tahansa solmuun $b$.

Algoritmin alussa etäisyysmatriisi sisältää vain polut,
jotka vastaavat yksittäistä verkon kaarta.
Sitten algoritmi suorittaa $n$ kierrosta,
jotka on numeroitu $1,2,\dots,n$.
Kierroksella $k$ algoritmi etsii polkuja, joissa on välisolmuna
solmu $k$ sekä mahdollisesti solmuja $1,2,\dots,k-1$.
Viimeisen kierroksen jälkeen jokainen solmu voi olla
välisolmuna poluilla, jolloin algoritmi on saanut selville
kaikki lyhimmät polut.

\begin{figure}
\center
\begin{center}
\begin{tikzpicture}[scale=0.6,label distance=-1.5mm]
\small
\newcommand\verkko[5]{
\begin{scope}[xshift=0.75cm,yshift=2cm]
\node[draw, circle, fill=#2] (1) at (0,0) {$1$};
\node[draw, circle, fill=#3] (2) at (2.5,0) {$2$};
\node[draw, circle, fill=#4] (3) at (0,-2.5) {$3$};
\node[draw, circle, fill=#5] (4) at (2.5,-2.5) {$4$};
\path[draw,thick,->] (1) -- node[font=\small,label=above:5] {} (2);
\path[draw,thick,->] (1) -- node[font=\small,label=left:1] {} (3);
\path[draw,thick,->] (2) -- node[font=\small,label=right:3] {} (4);
\path[draw,thick,->] (3) -- node[font=\small,label=above:2] {} (2);
\path[draw,thick,->] (4) -- node[font=\small,label=below:4] {} (3);
\end{scope}

\foreach \x in {1,2,3,4} \node at (-0.5,-2.5-\x) {\x};
\foreach \x in {1,2,3,4} \node at (-0.5+\x,-2.5) {\x};
\draw (0,-3) grid (4,-7);

\node at (2,-8) {vaihe #1};
}
\begin{scope}
\verkko{1}{lightgray}{white}{white}{white}
\foreach \x/\v in {1/\infty,2/5,3/1,4/\infty} \node at (-0.5+\x,-3.5) {$\v$};
\foreach \x/\v in {1/\infty,2/\infty,3/\infty,4/3} \node at (-0.5+\x,-4.5) {$\v$};
\foreach \x/\v in {1/\infty,2/2,3/\infty,4/\infty} \node at (-0.5+\x,-5.5) {$\v$};
\foreach \x/\v in {1/\infty,2/\infty,3/4,4/\infty} \node at (-0.5+\x,-6.5) {$\v$};
\end{scope}
\begin{scope}[xshift=5.5cm]
\verkko{2}{white}{lightgray}{white}{white}
\foreach \x/\v in {1/\infty,2/5,3/1,4/} \node at (-0.5+\x,-3.5) {$\v$};
\foreach \x/\v in {1/\infty,2/\infty,3/\infty,4/3} \node at (-0.5+\x,-4.5) {$\v$};
\foreach \x/\v in {1/\infty,2/2,3/\infty,4/} \node at (-0.5+\x,-5.5) {$\v$};
\foreach \x/\v in {1/\infty,2/\infty,3/4,4/\infty} \node at (-0.5+\x,-6.5) {$\v$};
\node[color=red] at (3.5,-3.5) {$8$};
\node[color=red] at (3.5,-5.5) {$5$};
\end{scope}
\begin{scope}[xshift=11cm]
\verkko{3}{white}{white}{lightgray}{white}
\foreach \x/\v in {1/\infty,2/,3/1,4/8} \node at (-0.5+\x,-3.5) {$\v$};
\foreach \x/\v in {1/\infty,2/\infty,3/\infty,4/3} \node at (-0.5+\x,-4.5) {$\v$};
\foreach \x/\v in {1/\infty,2/2,3/\infty,4/5} \node at (-0.5+\x,-5.5) {$\v$};
\foreach \x/\v in {1/\infty,2/,3/4,4/} \node at (-0.5+\x,-6.5) {$\v$};
\node[color=red] at (1.5,-3.5) {$3$};
\node[color=red] at (1.5,-6.5) {$6$};
\node[color=red] at (3.5,-6.5) {$9$};
\end{scope}
\begin{scope}[xshift=16.5cm]
\verkko{4}{white}{white}{white}{lightgray}
\foreach \x/\v in {1/\infty,2/3,3/1,4/8} \node at (-0.5+\x,-3.5) {$\v$};
\foreach \x/\v in {1/\infty,2/,3/,4/3} \node at (-0.5+\x,-4.5) {$\v$};
\foreach \x/\v in {1/\infty,2/2,3/,4/5} \node at (-0.5+\x,-5.5) {$\v$};
\foreach \x/\v in {1/\infty,2/6,3/4,4/9} \node at (-0.5+\x,-6.5) {$\v$};
\node[color=red] at (1.5,-4.5) {$9$};
\node[color=red] at (2.5,-4.5) {$7$};
\node[color=red] at (2.5,-5.5) {$9$};
\end{scope}
\end{tikzpicture}
\end{center}
\caption{Esimerkki Floyd–Warshallin algoritmin toiminnasta.}
\label{fig:flowar}
\end{figure}

Kuva \ref{fig:flowar} näyttää esimerkin Floyd–Warshallin algoritmin toiminnasta.
Etäisyysmatriisin rivillä $a$ sarakkeessa $b$ on lyhin tunnettu etäisyys
solmusta $a$ solmuun $b$.
Alussa matriisin sisältö on muodostettu suoraan verkon vierusmatriisista,
ja vaiheessa $k$ parannamme etäisyyksiä poluilla, jotka kulkevat solmun $k$
(ja mahdollisesti solmujen $1,2,\dots,k-1$) kautta.

Vaiheessa 1 emme voi muodostaa mitään uusia polkuja,
koska mistään solmusta ei pääse solmuun 1.
Vaiheessa 2 pääsemme solmun 2 kautta solmusta 1 solmuun 4 ja
solmusta 3 solmuun 4; vastaavat polkujen pituudet ovat 8 ja 5.
Sitten muodostamme vastaavalla tavalla solmujen 3 ja 4
kautta kulkevat polut.

\subsection{Analyysi}

Yksi tapa ymmärtää Floyd–Warshallin algoritmia on
ajatella algoritmin toimintaa ''käänteisesti'' rekursiivisesti:
kun verkossa on lyhin polku solmusta $a$ solmuun $b$,
millainen tämä polku voi olla?

Jos solmu $x$ kuuluu polkuun, meille syntyy kaksi osaongelmaa:
meidän tulee etsiä ensin lyhin polku solmusta $a$ solmuun $x$
ja sitten lyhin polku solmusta $x$ solmuun $b$.
Näiden polkujen muodostamisessa voimme jälleen käy läpi tapauksia,
mitkä solmut kuuluvat polkuihin.
Esimerkiksi lyhin polku solmusta $a$ solmuun $x$
voi kulkea vuorostaan solmun $y$ kautta,
jolloin haluamme etsiä lyhimmät polut solmusta $a$ solmuun $y$
ja solmusta $y$ solmuun $x$.

Floyd–Warshallin algoritmissa muodostamme joka vaiheessa
polkuja, joissa voi olla välisolmuina solmuja $1,2,\dots,i$.
Kun haluamme muodostaa lyhimmän polun solmusta $a$ solmuun $b$,
meillä on kaksi vaihtoehtoa:
Jos solmu $i$ on välisolmuna, yhdistämme lyhimmät polut
solmusta $a$ solmuun $i$ ja solmusta $i$ solmuun $b$.
Jos taas solmu $i$ ei ole välisolmuna, olemme käsitelleet
polun jo aiemmin. 

Seuraavaksi näemme, kuinka voimme toteuttaa algoritmin
tehokkaasti ajassa $O(n^3)$ kolmen sisäkkäisen for-silmukan avulla.

\subsection{Toteutus}

Toteutuksemme lähtökohtana on verkon vierusmatriisiesitys:

\begin{code}
int[][] verkko = new int[n+1][n+1];
\end{code}

Tämän esityksen rinnalle luomme etäisyysmatriisin,
joka tulee sisältämään solmujen etäisyydet:

\begin{code}
int[][] etaisyys = new int[n+1][n+1];
\end{code}

Ensin alustamme etäisyysmatriisin niin,
että jos solmusta $a$ solmuun $b$ on kaari,
etäisyys on kaaren paino, ja muuten etäisyys on ääretön.
Voimme tehdä tämän seuraavasti:

\begin{code}
for (int i = 1; i <= n; i++) {
    for (int j = 1; j <= n; j++) {
        if (verkko[i][j] == 0) {
            etaisyys[i][j] = 1e9;
        } else {
            etaisyys[i][j] = verkko[i][j];
        }
    }
}
\end{code}

Tämän jälkeen voimme laskea etäisyydet seuraavasti:

\begin{code}
for (int k = 1; k <= n; k++) {
    for (int i = 1; i <= n; i++) {
        for (int j = 1; j <= n; j++) {
            int vanha = etaisyys[i][j];
            int uusi = etaisyys[i][k]+etaisyys[k][j];
            etaisyys[i][j] = Math.min(vanha,uusi);
        }
    }
}
\end{code}

Tässä muuttuja $k$ on kierroslaskuri, joka ilmaisee, mikä solmu toimii
välisolmuna uusilla poluilla.

\section{Esimerkki: X}