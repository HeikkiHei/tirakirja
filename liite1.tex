\chapter{Matemaattinen tausta}

\section*{Summakaavat}

\index{summakaava}

Voimme laskea lukujen $1,2,\dots,n$ summan kaavalla
\[1+2+\dots+n = \frac{n(n+1)}{2}.\]
Esimerkiksi
\[1+2+3+4+5 = \frac{5 \cdot 6}{2}=15.\]
Kaavan voi ymmärtää niin, että laskemme yhteen $n$ lukua,
joiden suuruus on \emph{keskimäärin} $(n+1)/2$.

Toinen hyödyllinen kaava on
\[2^0+2^1+\dots+2^n = 2^{n+1}-1.\]
Esimerkiksi
\[1+2+4+8+16=32-1.\]
Tässä voimme ajatella, että aloitamme luvusta $2^n$
ja lisäämme siihen aina puolet pienemmän luvun lukuun $1$ asti.
Tämän seurauksena pääsemme yhtä vaille lukuun $2^{n+1}$ asti.

Esitämme joskus summia merkinnän $\sum$ avulla.
Siinä on ideana antaa muuttujan ala- ja yläraja sekä
joka askeleella summaan lisättävä arvo.
Esimerkiksi voimme merkitä
\[1^2 + 2^2 + \dots + n^2 = \sum_{i=1}^n i^2.\]

Tämä merkintä on itse asiassa hyvin lähellä ohjelmoinnin
for-silmukkaa, koska seuraava koodi ajaa saman asian:

\begin{code}
summa = 0
for i = 1 to n
    summa += i*i
\end{code}

\section*{Logaritmit}

\index{logaritmi}

Logaritmin määritelmän mukaan $\log_b n =x$
tarkalleen silloin kun $b^x=n$.
Esimerkiksi $\log_2 32=5$, koska $2^5=32$.

Logaritmi $\log_b n$ kertoo,
montako kertaa meidän tulee jakaa luku $n$ luvulla $b$,
ennen kuin pääsemme lukuun 1.
Esimerkiksi $\log_2 32 =5$, koska tarvitsemme 5 puolitusta:
\[32 \rightarrow 16 \rightarrow 8 \rightarrow 4 \rightarrow 2 \rightarrow 1\]
Tässä kirjassa oletamme, että logaritmin kantaluku on 2,
jos ei ole toisin mainittu,
eli voimme kirjoittaa lyhyesti $\log 32 = 5$.

Logaritmeille pätevät kaavat
\[\log_b(x \cdot y) = \log_b(x)+\log_b(y)\]
ja
\[\log_b(x / y) = \log_b(x)-\log_b(y).\]
Ylemmästä kaavasta seuraa myös
\[\log_b(x^k) = k \log_b(x).\]
Lisäksi voimme vaihtaa logaritmin kantalukua kaavalla
\[\log_u(x) = \frac{\log_b(x)}{\log_b(u)}.\]
