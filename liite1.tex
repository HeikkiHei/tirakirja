\chapter{Matemaattinen tausta}

Tämä liite käy läpi matematiikan merkintöjä ja käsitteitä,
joita käytetään kirjassa algoritmien analysoinnissa.

\section*{Merkintöjä}

Merkinnät $\lfloor x \rfloor$ ja $\lceil x \rceil$ tarkoittavat,
että pyöristämme luvun $x$ alaspäin ja ylöspäin kokonaisluvuksi.
Esimerkiksi $\lfloor 5.23 \rfloor = 5$ ja $\lceil 5.23 \rceil = 6$.

\index{kertoma}

Kertoma $n!$ lasketaan $1 \cdot 2 \cdot 3 \cdots n$.
Esimerkiksi $5! = 1 \cdot 2 \cdot 3 \cdot 4 \cdot 5 = 120$.

\index{jakojäännös}

Merkintä $a \bmod b$ tarkoittaa,
mikä on jakojäännös, kun $a$ jaetaan $b$:llä.
Esimerkiksi $32 \bmod 5 = 2$, koska $32 = 6 \cdot 5 + 2$.

\section*{Summakaavat}

\index{summakaava}

Voimme laskea lukujen $1,2,\dots,n$ summan kaavalla
\[1+2+\dots+n = \frac{n(n+1)}{2}.\]
Esimerkiksi
\[1+2+3+4+5 = \frac{5 \cdot 6}{2}=15.\]
Kaavan voi ymmärtää niin, että laskemme yhteen $n$ lukua,
joiden suuruus on \emph{keskimäärin} $(n+1)/2$.

Usein hyödyllinen on myös geometrinen summa
\[a^0+a^1+\dots+a^n = \frac{a^{n+1}-1}{a-1},\]
joka pätee, kun $a \neq 1$. Tämän erikoistapaus on kaava
\[2^0+2^1+\dots+2^n = 2^{n+1}-1.\]
Tässä voimme ajatella, että aloitamme luvusta $2^n$
ja lisäämme siihen aina puolet pienemmän luvun lukuun $1$ asti.
Tämän seurauksena pääsemme yhtä vaille lukuun $2^{n+1}$ asti.

\section*{Logaritmi}

\index{logaritmi}

Logaritmin määritelmän mukaan $\log_b n =x$
tarkalleen silloin kun $b^x=n$.
Esimerkiksi $\log_2 32=5$, koska $2^5=32$.

Logaritmi $\log_b n$ kertoo,
montako kertaa meidän tulee jakaa luku $n$ luvulla $b$,
ennen kuin pääsemme lukuun 1.
Esimerkiksi $\log_2 32 =5$, koska tarvitsemme 5 puolitusta:
\[32 \rightarrow 16 \rightarrow 8 \rightarrow 4 \rightarrow 2 \rightarrow 1\]
Tässä kirjassa oletamme, että logaritmin kantaluku on 2,
jos ei ole toisin mainittu,
eli voimme kirjoittaa lyhyesti $\log 32 = 5$.

Logaritmeille pätevät kaavat
\[\log_b(x \cdot y) = \log_b(x)+\log_b(y)\]
ja
\[\log_b(x / y) = \log_b(x)-\log_b(y).\]
Ylemmästä kaavasta seuraa myös
\[\log_b(x^k) = k \log_b(x).\]
Lisäksi voimme vaihtaa logaritmin kantalukua kaavalla
\[\log_u(x) = \frac{\log_b(x)}{\log_b(u)}.\]

\section*{Odotusarvo}

\index{odotusarvo}

\emph{Odotusarvo} kuvaa, mikä on keskimääräinen tulos,
jos satunnainen tapahtuma toistuu monta kertaa eli mitä
tulosta voimme tavallaan odottaa.

Kun tapahtumalla on $n$ mahdollista tulosta,
odotusarvo lasketaan kaavalla
\[p_1 t_1 + p_2 t_2 + \dots + p_n t_n,\]
missä $p_i$ ja $t_i$ ovat tapahtuman $i$ todennäköisyys ja tulos.
Esimerkiksi kun heitämme noppaa, tuloksen odotusarvo on
\[1/6 \cdot (1+2+3+4+5+6) = 7/2.\]
