\chapter{Alkusanat}

Ohjelmoinnin oppiminen on pitkä prosessi,
josta voi erottaa kaksi vaihetta.
Ensimmäinen vaihe on oppia ohjelmoinnin perustaidot,
kuten miten käyte\-tään muuttujia, ehtolauseita, silmukoita ja taulukoita.
Ohjelmoinnin peruskurssit käsittelevät näitä aiheita.
Toinen vaihe, johon keskitymme tällä kurssilla,
on oppia luomaan \emph{tehokkaita} algoritmeja.

Kun saamme eteemme ohjelmointiongelman,
se on portti seikkailuun, jossa voi odottaa monenlaisia haasteita.
Kaikki keinot ovat sallittuja, kunhan vain saamme aikaan
algoritmin, joka ratkaisee ongelman tehokkaasti.
Tämä kurssi opettaa monia tekniikoita ja ideoita,
joista on hyötyä algoritmisten ongelmien ratkaisemisessa.

Algoritmien suunnittelu on keskeisessä asemassa tietojenkäsittelytieteen
teoreettisessa tutkimuksessa, mutta tehokkaat algoritmit ovat
tärkeitä myös monissa käytännön sovelluksissa.
Tulemme huomaamaan kurssin aikana jatkuvasti,
mikä yhteys teoreettisilla tuloksilla on siihen,
miten hyvin algoritmit toimivat käytännössä.

Jos sinulla on palautetta kirjasta, voit lähettää sitä
sähköpostitse osoitteeseen \url{ahslaaks@cs.helsinki.fi}.
Lukijoiden palautteesta on paljon hyötyä
kirjan kehittämisessä.
Kirjan uusin versio on aina saatavilla GitHubissa
osoitteessa \url{https://github.com/pllk/tirakirja}.