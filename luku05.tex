\chapter{Algoritmien suunnittelu}

\section{Ahneet algoritmit}

Ahne algoritmi valitsee joka askeleella sillä hetkellä parhaalta
näyttävän vaihtoehdon, joka on usein pienin tai suurin mahdollinen
tapa muodostaa ratkaisua eteenpäin.
Tyypillinen ahneen algoritmin ensimmäinen vaihe on aineiston
järjestäminen, mikä antaa lähtökohdan algoritmin toiminnalle.

Tarkastellaan esimerkkinä tehtävää, jossa tiedossamme on
$n$ vapaata asuntoa sekä $m$ henkilöä, jotka haluavat vuokrata asunnon.
Jokaisella asunnolla on tietty vuokra,
ja jokaisella asunnon hakijalla on vuokraraja,
kuinka suuren vuokran hän voi maksaa \emph{korkeintaan}.
Kuinka voimme jakaa asunnot niin, että mahdollisimman moni
halukas vuokralainen saa asunnon?

Esimerkiksi jos asuntojen vuokrat ovat $[500,550,700,900]$
ja hakijoiden vuokrarajat ovat $[400,600,800,900]$,
voimme järjestää niin, että kolme hakijaa saa asunnon.
Esimerkiksi voimme muodostaa parit $(500,600)$, $(700,800)$
ja $(900,900)$, missä ensimmäinen arvo on asunnon vuokra
ja toinen arvo on hakijan vuokraraja.
On selvää, että tämä on optimaalinen ratkaisu:
vuokrarajan 400 esittänyt hakija ei voi mitenkään saada asuntoa,
koska halvimmankin asunnon vuokra on 500.

Voimme ratkaista ongelman ahneella algoritmilla,
joka käy läpi asunnon hakijat järjestyksessä vuokrarajan
mukaan pienimmästä suurimpaan.
Jokaisen hakijan kohdalla annamme hänelle halvimman
jäljellä olevan asunnon, jos sen vuokra ei ylitä vuokrarajaa.
Jos halvin asunto on liian kallis, emme anna asuntoa ja
siirrymme suoraan seuraavaan hakijaan.
Tällainen algoritmi vie aikaa $O(n \log n + m \log m)$,
koska meidän riittää järjestää asunnot ja hakijat
ja käydä niitä sitten läpi rinnakkain.

Yllä olevassa esimerkissä aloitamme vuokrarajan 400
esittäneestä hakijasta, joka kuitenkaan ei saa asuntoa.
Tämän jälkeen käymme läpi loput hakijat
vuokrarajoilla 600, 800 ja 900, ja muodostamme parit
$(500,600)$, $(550,800)$ ja $(700,900)$.
Tuloksena on toinen mahdollinen optimaalinen ratkaisu,
jossa kolme hakijaa saa asunnon.

Tärkeä kysymys ahneissa algoritmeissa on, \emph{miksi}
algoritmi toimii. Miten voimme olla varmoja, että yllä kuvattu tapa
jakaa asunnot tuottaa aina optimaalisen ratkaisun?

Tässä tehtävässä voimme ajatella asiaa niin,
että jos voimme antaa jonkin asunnon hakijalle,
emme voi koskaan saada parempaa ratkaisua,
jos päätämmekin olla antamatta asuntoa.
Koska käsittelemme hakijat järjestyk\-sessä vuokrarajan mukaan,
jokaisella myöhemmällä hakijalla on paremmat mahdollisuudet
vuokrata asuntoa kuin nykyisellä hakijalla.
Jos emme anna asuntoa nykyiselle hakijalle $x$
ja joku myöhempi hakija $y$ saa sen, voimme kuitenkin 
muuttaa lopullista ratkaisua niin, että $x$
saa asunnon $y$:n sijasta ja ratkaisu säilyy yhtä hyvänä.
Niinpä on turvallinen päätös antaa asunto nykyiselle hakijalle.

\section{Pino ja jono}

\section{Summataulukko}

\section{Kahden osoittimen tekniikka}

\section{Binäärihaku}
