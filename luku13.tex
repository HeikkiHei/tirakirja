\chapter{Komponentit ja virittävät puut}

Tähän mennessä olemme tarkastelleet verkkoja,
joiden rakenne säilyy samana koko algoritmin ajan.
Mitä tapahtuu sitten, jos verkkoon tuleekin \emph{muutoksia},
kuten lisäämme verkkoon uusia kaaria?

Tutustumme tässä luvussa union-find-rakenteeseen,
joka on hyödyl\-linen työkalu verkkojen käsittelyssä.
Rakenteen avulla voimme pitää kirjaa verkon yhtenäisistä
komponenteista ja päivittää rakennetta tehokkaasti,
kun lisäämme verkkoon kaaria.
Voimme esimerkiksi tarkkailla, montako yhte\-näistä
komponenttia verkossa on milläkin hetkellä.

Käsittelemme myös pienimmän virittävän puun ongelmaa,
jossa haluamme kytkeä verkon solmut toisiinsa kaaria käyttäen niin,
että kaarten yhteispaino on pienin.
Voimme ratkaista ongelman tehokkaasti Kruskalin algoritmilla,
joka perustuu union-find-rakenteeseen,
tai Primin algoritmilla, joka muistuttaa Dijkstran algoritmia.

\section{Union-find-rakenne}

\index{union-find-rakenne}

Union-find-rakenne on tietorakenne, joka
pitää yllä kokoelmaa alkioiden joukkoja ja tarjoaa
seuraavat tehokkaat operaatiot:

\begin{itemize}
\item tarkasta, ovatko kaksi alkiota samassa joukossa
\item yhdistä kaksi joukkoa samaksi joukoksi
\end{itemize}

Oletamme, että alkiot ovat $1,2,\dots,n$,
ja jokainen alkio kuuluu tarkalleen yhteen joukkoon.
Esimerkiksi kun $n=8$, joukot voivat olla vaikkapa
$A=\{1,4\}$, $B=\{2,5,6\}$ ja $C=\{3,7,8\}$.
Tällä hetkellä esimerkiksi alkiot 1 ja 2 ovat eri joukoissa.
Kun yhdistämme sitten joukot $A$ ja $B$,
niistä syntyy joukko $\{1,2,4,5,6\}$.
Tästä lähtien alkiot 1 ja 2 ovatkin samassa joukossa.

\subsection{Rakenteen toteutus}

Toteutamme union-find-rakenteen niin,
että jokaisessa joukossa
yksi alkioista on joukon \emph{edustaja}.
Kutakin joukkoa vastaa puu,
jonka juurena on joukon edustaja ja
muut alkiot viittaavat edustajaan yhden tai useamman kaaren kautta.
Kun haluamme tarkastaa, ovatko kaksi alkiota samassa joukossa,
selvitämme niiden edustajat ja vertaamme niitä toisiinsa.

Kuvassa \ref{fig:unifin} on esimerkkinä union-find-rakenne,
joka vastaa joukkoja $A=\{1,4\}$, $B=\{2,5,6\}$ ja $C=\{3,7,8\}$.
Tässä tapauksessa joukkojen edustajat ovat 1, 5 ja 3.
Esimerkiksi jos haluamme tarkastaa, ovatko alkiot 2 ja 6
samassa joukossa, selvitämme ensin alkioiden edustajat
kulkemalla polkuja $2 \rightarrow 5$ ja $6 \rightarrow 5$.
Kummankin alkion edustaja on $5$, joten toteamme,
että alkiot ovat samassa joukossa.

\begin{figure}
\center
\begin{center}
\begin{tikzpicture}[scale=0.6]
\node[draw, circle] (1) at (0,0) {$1$};
\node[draw, circle] (4) at (0,-2) {$4$};

\node[draw, circle] (2) at (3,-2) {$2$};
\node[draw, circle] (5) at (4,0) {$5$};
\node[draw, circle] (6) at (5,-2) {$6$};

\node[draw, circle] (3) at (8,0) {$3$};
\node[draw, circle] (7) at (8,-2) {$7$};
\node[draw, circle] (8) at (8,-4) {$8$};

\path[draw,thick,->] (4) -- (1);
\path[draw,thick,->] (2) -- (5);
\path[draw,thick,->] (6) -- (5);
\path[draw,thick,->] (7) -- (3);
\path[draw,thick,->] (8) -- (7);
\end{tikzpicture}
\end{center}
\caption{Union-find-rakenne joukoille $\{1,4\}$, $\{2,5,6\}$ ja $\{3,7,8\}$.}
\label{fig:unifin}
\end{figure}

Jotta saamme toteutettua union-find-rakenteen,
pidämme yllä jokaiselle alkiolle $x$ arvoa $\texttt{vanhempi}[x]$,
joka kertoo seuraavan alkion ylempänä puussa.
Kuitenkin jos $x$ on joukon edustaja,
$\texttt{vanhempi}[x]=x$.
Esimerkiksi kuvassa \ref{fig:unifin}
$\texttt{vanhempi}[2]=5$ ja $\texttt{vanhempi}[5]=5$.
Tämän ansiosta pystymme selvittämään alkion $x$
edustajan seuraavasti:                                                          

\begin{code}
function edustaja(x)
    while x != vanhempi[x]
        x = vanhempi[x]
    return x
\end{code}

Tämän jälkeen voimme tarkastaa seuraavalla operaatiolla,
ovatko alkiot $a$ ja $b$ samassa joukossa.
Alkiot ovat samassa joukossa täsmälleen silloin, kun niillä
on sama edustaja:

\begin{code}
function sama(a,b)
    return edustaja(a) == edustaja(b)
\end{code}

Haluamme toteuttaa vielä operaation, jolla voimme
yhdistää kaksi joukkoa toisiinsa.
Tämän operaation toteutus ratkaisee, kuinka tehokas rakenteemme on.
Alkion edustajan etsiminen vie aikaa $O(k)$,
missä $k$ on polun pituus,
joten haluamme toteuttaa yhdistämiset niin,
että puussa on vain lyhyitä polkuja.
Saavutamme tämän tavoitteen yhdistämällä kaksi joukkoa
aina asettamalla \emph{pienemmän} joukon
edustajan osoittamaan \emph{suuremman} joukon edustajaan.
Jos joukot ovat yhtä suuria, voimme toteuttaa yhdistämisen kummin päin vain.

\begin{figure}
\center
\begin{center}
\begin{tikzpicture}[scale=0.6]
\node[draw, circle] (1) at (0,0) {$1$};
\node[draw, circle] (2) at (0,-2) {$4$};
\node[draw, circle] (3) at (4,0) {$5$};
\node[draw, circle] (4) at (3,-2) {$2$};
\node[draw, circle] (5) at (5,-2) {$6$};
\path[draw,thick,->] (2) -- (1);
\path[draw,thick,->] (4) -- (3);
\path[draw,thick,->] (5) -- (3);
\path[draw,thick,->,dashed] (1) -- (3);
\end{tikzpicture}
\end{center}
\caption{Tehokas yhdistäminen. Alkion $1$ joukon koko on 2
ja alkion $5$ joukon koko on 3,
joten yhdistämme alkion 1 alkioon 5.}
\label{fig:tehyhd}
\end{figure}

Kuva \ref{fig:tehyhd} näyttää, mitä tapahtuu, kun yhdistämme joukot
$A=\{1,4\}$ ja $B=\{2,5,6\}$.
Joukon $A$ edustaja on $1$ ja siinä on kaksi alkiota,
kun taas joukon $B$ edustaja on $5$ ja siinä on kolme alkiota.
Koska joukko $A$ on pienempi, asetamme joukon $A$
edustajan osoittamaan joukon $B$ edustajaan.
Tämän jälkeen kaikki alkiot kuuluvat samaan joukkoon ja
alkio $5$ on tästä lähtien koko joukon edustaja.

Nyt olemme valmiita toteuttamaan operaation,
joka yhdistää toisiinsa joukot, joissa on alkiot $a$ ja $b$.
Oletamme, että alkiot ovat eri joukoissa ennen yhdistämistä.
Jotta voimme toteuttaa yhdistämisen tehokkaasti,
meidän täytyy myös pitää kirjaa kunkin joukon koosta.
Seuraavassa toteutuksessa $\texttt{koko}[x]$ kertoo,
montako alkiota alkion $x$ edustama joukko sisältää.

\begin{code}
function yhdista(a,b)
    a = edustaja(a)
    b = edustaja(b)
    if koko[a] < koko[b]
        swap(a,b)
    vanhempi[b] = a
    koko[a] += koko[b]
\end{code}

Kun toteutamme yhdistämiset tällä tavalla, jokainen
puussa esiintyvä polku sisäl\-tää vain $O(\log n)$ alkiota.
Tämä johtuu siitä, että aina kun kuljemme polkua
askeleen ylöspäin alkiosta $a$ alkioon $b$,
$\texttt{koko}[b] \ge 2 \cdot \texttt{koko}[a]$ eli
edustajaa vastaavan joukon koko ainakin \emph{kaksinkertaistuu}.
Koska joukossa on enintään $n$ alkiota,
kuljemme siis yhteensä enintään $O(\log n)$ askelta.
Niinpä kaikki union-find-rakenteen operaatiot
toimivat ajassa $O(\log n)$.

\subsection{Esimerkki: Kaupungit}

Bittimaassa on $n$ kaupunkia, joiden välillä ei ole vielä yhtään tietä.
Sitten teitä aletaan rakentaa yksi kerrallaan, yhteensä $m$ tietä.
Jokainen tie yhdistää kaksi kaupunkia toisiinsa.
Minkä tien rakentamisen jälkeen kaikki kaupungit ovat ensimmäistä
kertaa yhteydessä toisiinsa?

\begin{figure}
\center
\begin{center}
\begin{tikzpicture}[scale=0.6,label distance=-1.5mm]
\small
\newcommand\verkko[1]{
\node[draw, circle] (1) at (0,-1) {$1$};
\node[draw, circle] (2) at (2,0) {$2$};
\node[draw, circle] (3) at (2,-2) {$3$};
\node[draw, circle] (4) at (4,0) {$4$};
\node[draw, circle] (5) at (4,-2) {$5$};
\node at (2,-3.5) {vaihe #1};
}
\begin{scope}
\verkko{1}
\path[draw,thick,-] (1) -- (2);
\end{scope}
\begin{scope}[xshift=6.5cm]
\verkko{2}
\path[draw,thick,-] (1) -- (2);
\path[draw,thick,-] (1) -- (3);
\end{scope}
\begin{scope}[xshift=13cm]
\verkko{3}
\path[draw,thick,-] (1) -- (2);
\path[draw,thick,-] (1) -- (3);
\path[draw,thick,-] (2) -- (3);
\end{scope}
\begin{scope}[yshift=-5.5cm]
\verkko{4}
\path[draw,thick,-] (1) -- (2);
\path[draw,thick,-] (1) -- (3);
\path[draw,thick,-] (2) -- (3);
\path[draw,thick,-] (4) -- (5);
\end{scope}
\begin{scope}[yshift=-5.5cm,xshift=6.5cm]
\verkko{5}
\path[draw,thick,-] (1) -- (2);
\path[draw,thick,-] (1) -- (3);
\path[draw,thick,-] (2) -- (3);
\path[draw,thick,-] (4) -- (5);
\path[draw,thick,-] (2) -- (4);
\end{scope}
\begin{scope}[yshift=-5.5cm,xshift=13cm]
\verkko{6}
\path[draw,thick,-] (1) -- (2);
\path[draw,thick,-] (1) -- (3);
\path[draw,thick,-] (2) -- (3);
\path[draw,thick,-] (4) -- (5);
\path[draw,thick,-] (2) -- (4);
\path[draw,thick,-] (2) -- (5);
\end{scope}
\end{tikzpicture}
\end{center}
\caption{Esimerkki kaupunkien yhdistämisestä teillä. Vaiheen 5 jälkeen
kaikki kaupungit ovat yhteydessä toisiinsa.}
\label{fig:kauesi}
\end{figure}


Kuva \ref{fig:kauesi} näyttää esimerkkitapauksen, jossa $n=5$, $m=6$
ja tiet rakennetaan järjestyksessä $(1,2)$, $(1,3)$, $(2,3)$, $(4,5)$, $(2,4)$ ja $(2,5)$.
Kaikki kaupungit ovat yhteydessä toisiinsa vaiheen 5 jälkeen.

\subsubsection{Ratkaisu 1: Union-find-rakenne}

Pidämme yllä verkon komponentteja
union-find-rakenteen avulla.
Aluksi jokainen solmu on omassa komponentissaan
eli joukot ovat $\{1\},\{2\},\dots,\{n\}$.
Sitten jokaisen kaaren kohdalla tarkastamme,
ovatko sen päätesolmut eri joukoissa,
ja jos ovat, yhdistämme joukot.
Kun lopulta kaikki solmut ovat samassa joukossa,
verkko on tullut yhtenäiseksi.

Tuloksena oleva algoritmi vie aikaa $O(n+m \log n)$,
koska luomme ensin $n$ komponenttia ajassa $O(n)$
ja käsittelemme tämän jälkeen $m$ kaarta.
Jokaisen kaaren kohdalla suoritamme enintään kaksi
operaatiota union-find-rakenteessa ajassa $O(\log n)$.

\subsubsection{Ratkaisu 2: Binäärihaku}

Toinen tapa ratkaista tehtävä on hyödyntää \emph{binäärihakua}.
Jos meillä on arvaus, että kaikki kaupungit ovat yhteydessä
$x$ lisäyksen jälkeen, voimme tarkistaa helposti,
pitääkö arvaus paikkansa:
lisäämme ensin $x$ ensimmäistä tietä tyhjään verkkoon ja tarkastamme
sitten, onko verkko yhtenäinen. Tämä vie aikaa $O(n+m)$
käyttäen syvyyshakua.

Jos verkko on yhtenäinen ensimmäistä kertaa vaiheessa $k$,
selvästikin verkko ei ole yhtenäinen vaiheissa
$1,2,\dots,k-1$ ja on yhtenäinen vaiheissa $k,k+1,\dots,m$,
koska kaarten lisääminen ei voi poistaa verkon yhtenäisyyttä.
Tämän ansiosta voimme etsiä arvon $k$ binäärihaun avulla.
Binäärihaku suorittaa $O(\log m)$ askelta ja ratkaisu vie
aikaa $O((n+m) \log m)$.

\section{Pienin virittävä puu}

\index{virittävä puu}
\index{pienin virittävä puu}

Verkon \emph{virittävä puu} on kokoelma verkon kaaria,
jotka kytkevät kaikki verkon solmut toisiinsa.
Kuten puut yleensäkin, virittävä puu on yhtenäinen ja syklitön eli
jokaisen kahden solmun välillä on yksikäsitteinen polku.
Kuvassa \ref{fig:virpuu} on esimerkkinä verkko ja yksi sen virittävistä puista.

\begin{figure}
\center
\begin{center}
\begin{tikzpicture}[scale=0.6]
\begin{scope}
\small
\node[draw, circle] (1) at (0,-1) {$1$};
\node[draw, circle] (2) at (2,0) {$2$};
\node[draw, circle] (3) at (2,-2) {$3$};
\node[draw, circle] (4) at (4,0) {$4$};
\node[draw, circle] (5) at (4,-2) {$5$};
\path[draw,thick,-] (1) -- (2);
\path[draw,thick,-] (1) -- (3);
\path[draw,thick,-] (2) -- (3);
\path[draw,thick,-] (4) -- (5);
\path[draw,thick,-] (2) -- (4);
\path[draw,thick,-] (3) -- (5);
\end{scope}
\begin{scope}[xshift=6.5cm]
\node[draw, circle] (1) at (0,-1) {$1$};
\node[draw, circle] (2) at (2,0) {$2$};
\node[draw, circle] (3) at (2,-2) {$3$};
\node[draw, circle] (4) at (4,0) {$4$};
\node[draw, circle] (5) at (4,-2) {$5$};
%\path[draw,thick,-] (1) -- (2);
\path[draw,thick,-] (1) -- (3);
\path[draw,thick,-] (2) -- (3);
%\path[draw,thick,-] (4) -- (5);
\path[draw,thick,-] (2) -- (4);
\path[draw,thick,-] (3) -- (5);
\end{scope}
\end{tikzpicture}
\end{center}
\caption{Verkko ja yksi sen virittävistä puista.}
\label{fig:virpuu}
\end{figure}

Jos verkko on painotettu, kiinnostava ongelma on etsiä verkon
\emph{pienin virittävä puu}.
Tämä on virittävä puu, jonka kaarten painojen summa on
mahdollisimman pieni.
Esimerkiksi kuvassa \ref{fig:pievir} on painotettu verkko ja kaksi sen virittävää puuta,
joiden painot ovat 12 ja 10.
Näistä jälkimmäinen on verkon pienin virittävä puu.

\begin{figure}
\center
\begin{center}
\begin{tikzpicture}[scale=0.6,label distance=-1.5mm]
\begin{scope}
\small
\node[draw, circle] (1) at (0,-1) {$1$};
\node[draw, circle] (2) at (2,0) {$2$};
\node[draw, circle] (3) at (2,-2) {$3$};
\node[draw, circle] (4) at (4,0) {$4$};
\node[draw, circle] (5) at (4,-2) {$5$};
\path[draw,thick,-] (1) -- node[font=\small,label=above:2] {} (2);
\path[draw,thick,-] (1) -- node[font=\small,label=below:4] {} (3);
\path[draw,thick,-] (2) -- node[font=\small,label=right:1] {} (3);
\path[draw,thick,-] (4) -- node[font=\small,label=right:5] {} (5);
\path[draw,thick,-] (2) -- node[font=\small,label=above:2] {} (4);
\path[draw,thick,-] (3) -- node[font=\small,label=below:7] {} (5);
\end{scope}
\begin{scope}[xshift=6.5cm]
\node[draw, circle] (1) at (0,-1) {$1$};
\node[draw, circle] (2) at (2,0) {$2$};
\node[draw, circle] (3) at (2,-2) {$3$};
\node[draw, circle] (4) at (4,0) {$4$};
\node[draw, circle] (5) at (4,-2) {$5$};
%\path[draw,thick,-] (1) -- node[font=\small,label=above:2] {} (2);
\path[draw,thick,-] (1) -- node[font=\small,label=below:4] {} (3);
\path[draw,thick,-] (2) -- node[font=\small,label=right:1] {} (3);
\path[draw,thick,-] (4) -- node[font=\small,label=right:5] {} (5);
\path[draw,thick,-] (2) -- node[font=\small,label=above:2] {} (4);
%\path[draw,thick,-] (3) -- node[font=\small,label=below:7] {} (5);
\end{scope}
\begin{scope}[xshift=13cm]
\node[draw, circle] (1) at (0,-1) {$1$};
\node[draw, circle] (2) at (2,0) {$2$};
\node[draw, circle] (3) at (2,-2) {$3$};
\node[draw, circle] (4) at (4,0) {$4$};
\node[draw, circle] (5) at (4,-2) {$5$};
\path[draw,thick,-] (1) -- node[font=\small,label=above:2] {} (2);
%\path[draw,thick,-] (1) -- node[font=\small,label=below:5] {} (3);
\path[draw,thick,-] (2) -- node[font=\small,label=right:1] {} (3);
\path[draw,thick,-] (4) -- node[font=\small,label=right:5] {} (5);
\path[draw,thick,-] (2) -- node[font=\small,label=above:2] {} (4);
%\path[draw,thick,-] (3) -- node[font=\small,label=below:7] {} (5);
\end{scope}
\end{tikzpicture}
\end{center}
\caption{Painotettu verkko ja kaksi virittävää puuta,
joiden painot ovat $4+1+2+5=12$ ja $2+1+2+5=10$.}
\label{fig:pievir}
\end{figure}

\subsection{Kruskalin algoritmi}

\index{Kruskalin algoritmi}

Kruskalin algoritmi muodostaa verkon
pienimmän virittävän puun aloittamalla tyhjästä verkosta,
jossa on vain verkon solmut, ja lisäämällä siihen kaaria.
Algoritmi käy läpi tarjolla olevat kaaret
järjestyksessä niiden painon mukaan kevyimmästä raskaimpaan.
Jokaisen kaaren kohdalla algoritmi ottaa kaaren mukaan,
jos se yhdistää kaksi eri komponenttia.
Kun kaikki komponentit on yhdistetty, pienin virittävä puu on valmis.

\begin{figure}
\center
\begin{center}
\begin{tikzpicture}[scale=0.7,label distance=-1.5mm]
\small
\newcommand\verkko[1]{
\node[draw, circle] (1) at (0,-1) {$1$};
\node[draw, circle] (2) at (2,0) {$2$};
\node[draw, circle] (3) at (2,-2) {$3$};
\node[draw, circle] (4) at (4,0) {$4$};
\node[draw, circle] (5) at (4,-2) {$5$};
\path[draw,thick,-] (1) -- node[font=\small,label=above:2] {} (2);
\path[draw,thick,-] (1) -- node[font=\small,label=below:4] {} (3);
\path[draw,thick,-] (2) -- node[font=\small,label=right:1] {} (3);
\path[draw,thick,-] (4) -- node[font=\small,label=right:5] {} (5);
\path[draw,thick,-] (2) -- node[font=\small,label=above:2] {} (4);
\path[draw,thick,-] (3) -- node[font=\small,label=below:7] {} (5);
\node at (2,-3.5) {vaihe #1};
}
\begin{scope}
\verkko{1}
\path[draw,thick,-,red,line width=2pt] (2) -- (3);
\end{scope}
\begin{scope}[xshift=6.5cm]
\verkko{2}
\path[draw,thick,-,red,line width=2pt] (2) -- (3);
\path[draw,thick,-,red,line width=2pt] (1) -- (2);
\end{scope}
\begin{scope}[xshift=13cm]
\verkko{3}
\path[draw,thick,-,red,line width=2pt] (2) -- (3);
\path[draw,thick,-,red,line width=2pt] (1) -- (2);
\path[draw,thick,-,red,line width=2pt] (2) -- (4);
\end{scope}
\begin{scope}[yshift=-5.5cm]
\verkko{4}
\path[draw,thick,-,red,line width=2pt] (2) -- (3);
\path[draw,thick,-,red,line width=2pt] (1) -- (2);
\path[draw,thick,-,red,line width=2pt] (2) -- (4);
\path[draw,thick,-,red,line width=2pt,dashed] (3) -- (1);
\end{scope}
\begin{scope}[yshift=-5.5cm,xshift=6.5cm]
\verkko{5}
\path[draw,thick,-,red,line width=2pt] (2) -- (3);
\path[draw,thick,-,red,line width=2pt] (1) -- (2);
\path[draw,thick,-,red,line width=2pt] (2) -- (4);
\path[draw,thick,-,red,line width=2pt] (4) -- (5);
\end{scope}
\end{tikzpicture}
\end{center}
\caption{Esimerkki Kruskalin algoritmin toiminnasta.}
\label{fig:kruesi}
\end{figure}

Kuva \ref{fig:kruesi} näyttää, kuinka Kruskalin algoritmi löytää pienimmän virittävän
puun esimerkkiverkossamme.
Verkon kaaret järjestyksessä kevyim\-mästä raskaimpaan ovat:

\begin{center}
\begin{tabular}{rr}
kaari & paino \\
\hline
$(2,3)$ & $1$ \\
$(1,2)$ & $2$ \\
$(2,4)$ & $2$ \\
$(1,3)$ & $4$ \\
$(4,5)$ & $5$ \\
$(3,5)$ & $7$ \\
\end{tabular}
\end{center}

Algoritmi käsittelee ensin kaaren $(2,3)$.
Solmut $2$ ja $3$ ovat eri komponenteissa,
joten kaari otetaan mukaan puuhun.
Tämän jälkeen algoritmi käsittelee kaaret $(1,2)$ ja $(2,4)$,
jotka valitaan myös puuhun.
Seuraavaksi vuorossa on kaari $(1,3)$,
mutta tämä kaari ei tule puuhun,
koska solmut $1$ ja $3$ ovat jo samassa komponentissa.
Lopuksi algoritmi ottaa mukaan kaaren $(4,5)$,
jolloin pienin virittävä puu on valmis.

Voimme toteuttaa Kruskalin algoritmin tehokkaasti
käyttäen union-find-rakennetta.
Algoritmin alussa järjestämme kaaret painojärjestykseen,
missä kuluu aikaa $O(m \log m)$.
Kun oletamme, että jokaisen solmuparin välillä
on enintään yksi kaari, tämä sievenee muotoon $O(m \log n)$.
Tämän jälkeen luomme kullekin solmulle komponentin,
käymme kaaret läpi ja jokaisen kaaren
kohdalla otamme kaaren mukaan, jos se yhdistää kaksi eri komponenttia.
Tässä kuluu aikaa $O(n+m \log n)$,
kun käytämme union-find-rakennetta.
Algoritmi vie siis yhteensä aikaa $O(n+m \log n)$.

\subsection{Primin algoritmi}

\index{Primin algoritmi}

Primin algoritmi tarjoaa toisen lähestymistavan
pienimmän virittävän puun muodostamiseen.
Algoritmi aloittaa puun muodostamisen tilanteesta,
jossa puussa on vain yksi solmu.
Tämän jälkeen se etsii joka vaiheessa kevyimmän kaaren,
jonka toinen päätesolmu kuuluu puuhun ja toinen
päätesolmu on vielä puun ulkopuolella, ja lisää puuhun tämän kaaren.
Kun kaikki solmut on lisätty puuhun, pienin virittävä puu on valmis.

\begin{figure}
\center
\begin{center}
\begin{tikzpicture}[scale=0.7,label distance=-1.5mm]
\small
\newcommand\verkko[1]{
\node[draw, circle] (1) at (0,-1) {$1$};
\node[draw, circle] (2) at (2,0) {$2$};
\node[draw, circle] (3) at (2,-2) {$3$};
\node[draw, circle] (4) at (4,0) {$4$};
\node[draw, circle] (5) at (4,-2) {$5$};
\path[draw,thick,-] (1) -- node[font=\small,label=above:2] {} (2);
\path[draw,thick,-] (1) -- node[font=\small,label=below:4] {} (3);
\path[draw,thick,-] (2) -- node[font=\small,label=right:1] {} (3);
\path[draw,thick,-] (4) -- node[font=\small,label=right:5] {} (5);
\path[draw,thick,-] (2) -- node[font=\small,label=above:2] {} (4);
\path[draw,thick,-] (3) -- node[font=\small,label=below:7] {} (5);
\node at (2,-3.5) {vaihe #1};
}
\begin{scope}
\verkko{1}
\end{scope}
\begin{scope}[xshift=6.5cm]
\verkko{2}
\path[draw,thick,-,red,line width=2pt] (1) -- (2);
\end{scope}
\begin{scope}[xshift=13cm]
\verkko{3}
\path[draw,thick,-,red,line width=2pt] (1) -- (2);
\path[draw,thick,-,red,line width=2pt] (2) -- (3);
\end{scope}
\begin{scope}[yshift=-5.5cm]
\verkko{4}
\path[draw,thick,-,red,line width=2pt] (1) -- (2);
\path[draw,thick,-,red,line width=2pt] (2) -- (3);
\path[draw,thick,-,red,line width=2pt] (2) -- (4);
\end{scope}
\begin{scope}[yshift=-5.5cm,xshift=6.5cm]
\verkko{5}
\path[draw,thick,-,red,line width=2pt] (1) -- (2);
\path[draw,thick,-,red,line width=2pt] (2) -- (3);
\path[draw,thick,-,red,line width=2pt] (2) -- (4);
\path[draw,thick,-,red,line width=2pt] (4) -- (5);
\end{scope}
\end{tikzpicture}
\end{center}
\caption{Esimerkki Primin algoritmin toiminnasta.}
\label{fig:priesi}
\end{figure}

Kuva \ref{fig:priesi} näyttää esimerkin Primin algoritmin toiminnasta.
Voimme aloittaa puun rakentamisen mistä tahansa solmusta;
tässä esimerkissä aloitamme solmusta $1$.
Solmuun $1$ on yhteydessä kaksi kaarta $(1,2)$ ja $(1,3)$,
joista valitsemme kaaren $(1,2)$, koska se on kevyempi.
Seuraavaksi tarjolla ovat kaaret $(1,3)$, $(2,3)$ ja $(2,4)$,
joista valitsemme kaaren $(2,3)$.
Tämän jälkeen lisäämme puuhun vastaavalla tavalla kaaret
$(2,4)$, $(4,5)$, minkä jälkeen pienin virittävä puu on valmis.

Primin algoritmi muistuttaa paljon Dijkstran algoritmia.
Erona on, että Dijkstran algoritmissa valitsemme
seuraavaksi solmun, jonka etäisyys \emph{alkusolmuun} on pienin,
mutta Primin algoritmissa valitsemme solmun, jonka etäisyys
\emph{johonkin solmuun} puussa on pienin.
Voimme myös toteuttaa Primin algoritmin tehokkaasti samaan
tapaan kuin Dijkstran algoritmin keon avulla,
jolloin algoritmi vie aikaa $O(n+m \log n)$.
Aikavaativuus on siis sama kuin Kruskalin algoritmissa,
ja on käytännössä makuasia, kumman algoritmin valitsemme.

\subsection{Miksi algoritmit toimivat?}

Kruskalin ja Primin algoritmit ovat ahneita algoritmeja:
ne lisäävät joka askeleella kevyimmän mahdollisen kaaren puuhun.
Miksi on varmaa, että algoritmit tuottavat pienimmän virittävän
puun joka tilanteessa?

Voimme ajatella asiaa näin: 
Jos meillä on kaksi solmua $a$ ja $b$, jotka ovat eri komponenteissa,
meidän on yhdistettävä ne jotenkin samaan komponenttiin algoritmin aikana.
Jos kevyin saatavilla oleva kaari on solmujen $a$ ja $b$ välillä,
meidän kannattaa valita se, koska muuten joutuisimme yhdistämään komponentit
myöhemmin käyttäen raskaampaa kaarta.

Tarkastellaan esimerkkinä Kruskalin algoritmin alkua:
mitä tapahtuu, jos emme valitse puuhun kevyintä kaarta?
Kuvassa \ref{fig:pietod} näkyy kuvitteellinen tilanne,
jossa katkoviivoilla esitetty pienin virittävä puu ei sisällä
kevyintä kaarta $(2,3)$, jonka paino on 1.
Ei ole kuitenkaan mahdollista, että tämä olisi todellisuudessa
pienin virittävä puu, koska voisimme vaihtaa jonkin puun kaaren
kaareen $(2,3)$, jolloin puun paino pienenee.
Tämä tarkoittaa, että on varmasti turvallinen ratkaisu valita
kevyin kaari puuhun Kruskalin algoritmin alussa.
Vastaavasti voimme perustella, miksi tämän jälkeen kannattaa
valita seuraavaksi kevyin kaari, jne.

\begin{figure}
\center
\begin{center}
\begin{tikzpicture}[scale=0.7,label distance=-1.5mm]
\begin{scope}
\node[draw, circle] (1) at (0,-1) {$1$};
\node[draw, circle] (2) at (2,0) {$2$};
\node[draw, circle] (3) at (2,-2) {$3$};
\node[draw, circle] (4) at (4,0) {$4$};
\node[draw, circle] (5) at (4,-2) {$5$};
\path[draw,thick,-] (1) -- node[font=\small,label=above:2] {} (2);
\path[draw,thick,-] (1) -- node[font=\small,label=below:4] {} (3);
\path[draw,thick,-] (2) -- node[font=\small,label=right:1] {} (3);
\path[draw,thick,-] (4) -- node[font=\small,label=right:5] {} (5);
\path[draw,thick,-] (2) -- node[font=\small,label=above:2] {} (4);
\path[draw,thick,-] (3) -- node[font=\small,label=below:7] {} (5);
\end{scope}
\begin{scope}[xshift=6.5cm]
\node[draw, circle] (1) at (0,-1) {$1$};
\node[draw, circle] (2) at (2,0) {$2$};
\node[draw, circle] (3) at (2,-2) {$3$};
\node[draw, circle] (4) at (4,0) {$4$};
\node[draw, circle] (5) at (4,-2) {$5$};
\path[draw,thick,-,dashed] (1) -- (2);
\path[draw,thick,-,dashed] (1) -- (3);
\path[draw,thick,-,dashed] (4) -- (5);
\path[draw,thick,-,dashed] (2) -- (4);
\end{scope}
\begin{scope}[xshift=13cm]
\node[draw, circle] (1) at (0,-1) {$1$};
\node[draw, circle] (2) at (2,0) {$2$};
\node[draw, circle] (3) at (2,-2) {$3$};
\node[draw, circle] (4) at (4,0) {$4$};
\node[draw, circle] (5) at (4,-2) {$5$};
\path[draw,thick,-,dashed] (1) -- (3);
\path[draw,thick,-] (2) -- node[font=\small,label=right:1] {} (3);
\path[draw,thick,-,dashed] (4) -- (5);
\path[draw,thick,-,dashed] (2) -- (4);
\end{scope}
\end{tikzpicture}
\end{center}
\caption{Pienin virittävä puu sisältää varmasti kaaren $(2,3)$,
koska muuten voisimme pienentää puun painoa valitsemalla sen.}
\label{fig:pietod}
\end{figure}
