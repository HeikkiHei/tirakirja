\chapter{Tehokkuus}

\section{Aikavaativuus}

Voimme arvioida algoritmin tehokkuutta laskemalla,
montako kertaa siinä olevia komentoja suoritetaan.
Tavoitteena on arvioida tehokkuutta suhteessa
syötteen kokoon $n$.
Esimerkiksi jos syötteenä on taulukko,
$n$ on taulukon koko,
ja jos syötteenä on merkkijono,
$n$ on merkkijonon pituus.

Tarkastellaan esimerkkinä seuraavaa algoritmia,
joka laskee, montako kertaa luku $x$ esiintyy
$n$-kokoisessa taulukossa.

\begin{code}[numbers=left]
int maara = 0;
for (int i = 0; i < n; i++) {
    if (luvut[i] == x) {
        maara++;
    }
}
\end{code}

Tässä algoritmissa oleelliset komennot ovat riveillä
1, 3 ja 4.
Rivin 1 komento suoritetaan vain kerran algoritmin alussa.
Rivin 3 komento suoritetaan $n$ kertaa jokaisella silmukan
kierroksella.
Rivin 4 komento taas suoritetaan $0 \dots n$
kertaa riippuen siitä, kuinka usein
luku $x$ esiintyy taulukossa.
Algoritmissa suoritetaan siis vähintään $n+1$ ja enintään $2n+1$
komentoa.

Näin tarkka analyysi ei ole kuitenkaan yleensä tarpeen,
vaan meille riittää usein määrittää karkea ajankäytön yläraja.
Sanomme, että algoritmi toimii ajassa $O(f(n))$ eli sen
\emph{aikavaativuus} on $O(f(n))$, jos se suorittaa
enintään $c f(n)$ komentoa aina silloin kun $n \ge n_0$,
missä $c$ ja $n_0$ ovat vakioita.
Esimerkiksi yllä oleva algoritmi toimii ajassa $O(n)$,
koska se suorittaa selkeästi enintään $3n$ komentoa
kaikilla $n$:n arvoilla.

Aikavaativuden mukavana puolena on, että yleensä voimme
määrittää aikavaativuuden hyvin helposti algoritmin
rakenteesta. Tutustumme seuraavaksi laskusääntöihin,
joiden avulla tämä on mahdollista.

\subsection{Aikavaativuuden laskusäännöt}

Jos algoritmissa ei ole silmukoita vaan vain
yksittäisiä komentoja, sen aikavaativuus on $O(1)$.
Näin on esimerkiksi seuraavassa algoritmissa.

\begin{code}
c = a+b;
b = a;
if (a > b) a++;
\end{code}

Merkitsemme \texttt{...} koodia,
jonka aikavaativuus on $O(1)$.
Jos algoritmissa on yksi silmukka,
joka suorittaa $n$ askelta,
sen aikavaativuus on $O(n)$:

\begin{code}
for (int i = 0; i < n; i++) {
    ...
}
\end{code}

Jos tällaisia silmukoita on kaksi sisäkkäin,
aikavaativuus on $O(n^2)$:

\begin{code}
for (int i = 0; i < n; i++) {
    for (int j = 0; j < n; j++) {
        ...
    }
}
\end{code}

Yleisemmin jos algoritmissa on vastaavalla tavalla
$k$ sisäkkäistä silmukkaa,
sen aikavaativuus on $O(n^k)$.

Huomaa, että vakiokertoimet ja matalammat termit eivät vaikuta aikavaativuuteen.
Esimerkiksi seuraavia koodeja suoritetaan $2n$ ja $n-1$ kertaa,
mutta kummankin koodin aikavaativuus on $O(n)$.

\begin{code}
for (int i = 0; i < 2*n; i++) {
    ...
}
\end{code}

\begin{code}
for (int i = 0; i < n-1; i++) {
    ...
}
\end{code}

Jos algoritmissa on peräkkäisiä osuuksia, kokonaisaikavaativuus on suurin
yksittäinen aikavaativuus. Esimerkiksi seuraavan algoritmin aikavaativuus on $O(n^2)$,
koska sen osuudet ovat $O(n)$, $O(n^2)$ ja $O(n)$.

\begin{code}
for (int i = 0; i < n; i++) {
    ...
}
for (int i = 0; i < n; i++) {
    for (int j = 0; j < n; j++) {
        ...
    }
}
for (int i = 0; i < n; i++) {
    ...
}
\end{code}

Joskus aikavaativuus riippuu useammasta tekijästä,
jolloin kaavassa on monta muuttujaa.
Esimerkiksi seuraavan koodin aikavaativuus on $O(nm)$:

\begin{code}
for (int i = 0; i < n; i++) {
    for (int j = 0; j < m; j++) {
        ...
    }
}
\end{code}

\subsection{Yleisiä aikavaativuuksia}

Tietyt aikavaativuudet esiintyvät usein algoritmien analyysissa.
Seuraavaksi käymme läpi joukon tällaisia aikavaativuuksia.

\subsubsection{$O(1)$ (vakioaikainen)}

Vakioaikainen algoritmi suorittaa vain kiinteän määrän komentoja.
Tyypillinen vakioaikainen algoritmi on kaava, joka laskee
suoraan vastauksen. Esimerkiksi seuraava algoritmi laskee
summan $1+2+\dots+n$ vakioajassa:

\begin{code}
summa = n*(n+1)/2;
\end{code}

\subsubsection{$O(\log n)$ (logaritminen)}

Logaritminen algoritmi puolittaa usein syötteen koon
joka askeleella. Tyypillinen esimerkki logaritmisesta algoritmista
on binäärihaku, joka etsii alkiota järjestetystä taulukosta.

\begin{code}
int a = 0;
int b = n-1;
while (a <= b) {
    int c = (a+b)/2;
    if (taulu[c] == x) break;
    if (taulu[c] < x) a = c+1;
    else b = c-1;
}
\end{code}

\subsubsection{$O(n)$ (lineaarinen)}

Lineaarinen algoritmi voi käydä läpi syötteen kiinteän määrän kertoja.
Esimerkiksi seuraava algoritmi laskee taulukon lukujen summan $O(n)$-ajassa.

\begin{code}
int summa = 0;
for (int i = 0; i < n; i++) {
    summa += taulu[i];
}
\end{code}

\subsubsection{$O(n \log n)$ (järjestäminen)}

Aikavaativuus $O(n \log n)$ viittaa usein siihen,
että algoritmi järjestää taulukon,
koska tehokkaat algoritmit taulukon järjestämiseen
vievät aikaa $O(n \log n)$.
Esimerkiksi seuraava algoritmi laskee ajassa
$O(n \log n)$, montako eri alkiota taulukko sisältää.

\begin{code}
Arrays.sort(taulu);
int maara = 1;
for (int i = 1; i < n; i++) {
    if (taulu[i] != taulu[i-1]) maara++;
}
\end{code}

\subsubsection{$O(n^2)$ (neliöllinen)}

Neliöllinen algoritmi voi käydä läpi kaikki tavat valita
kaksi alkiota taulukosta.
Esimerkiksi seuraava $O(n^2)$-algoritmi tutkii, onko taulukossa
kahta lukua, joiden summa on $x$.

\begin{code}
boolean ok = false;
for (int i = 0; i < n; i++) {
    for (int j = i+1; j < n; j++) {
        if (taulu[i]+taulu[j] == x) ok = true;
    }
}
\end{code}

\subsubsection{$O(n^3)$ (kuutiollinen)}

Kuutiollinen algoritmi voi käydä läpi kaikki tavat valita
kolme alkiota taulukosta.
Esimerkiksi seuraava $O(n^3)$-algoritmi tutkii, onko taulukossa
kolmea lukua, joiden summa on $x$.

\begin{code}
boolean ok = false;
for (int i = 0; i < n; i++) {
    for (int j = i+1; j < n; j++) {
        for (int k = j+1; k < n; k++) {
            if (taulu[i]+taulu[j]+taulu[k] == x) ok = true;
        }
    }
}
\end{code}

\subsubsection{$O(2^n)$ (osajoukot)}

Aikavaativuus $O(2^n)$ viittaa usein siihen,
että algoritmi käy läpi syötteen alkioiden osajoukot.
Esimerkiksi alkioiden $\{1,2,3\}$ osajoukot ovat
$\emptyset$, $\{1\}$, $\{2\}$, $\{3\}$, $\{1,2\}$, $\{1,3\}$, $\{2,3\}$ ja $\{1,2,3\}$

\subsubsection{$O(n!)$ (permutaatiot)}

Aikavaativuus $O(n!)$ viittaa usein siihen,
että algoritmi käy läpi syötteen alkioiden permutaatiot.
Esimerkiksi alkioiden $\{1,2,3\}$ permutaatiot ovat
$(1,2,3)$, $(1,3,2)$, $(2,1,3)$, $(2,3,1)$, $(3,1,2)$ ja $(3,2,1)$.

\section{Esimerkkejä}
