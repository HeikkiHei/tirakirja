\chapter{Tehokkuus}

Algoritmien suunnittelussa tavoitteemme on saada aikaan
algoritmeja, jotka toimivat \emph{tehokkaasti}.
Haluamme luoda algoritmeja, joiden avulla voimme
käsitellä myös suuria aineistoja ilman, että joudumme
odottamaan kauan aikaa.
Ajattelemmekin, että algoritmi on \emph{hyvä},
jos se kykenee antamaan meille nopean vastauksen myös silloin,
kun annamme sille paljon tietoa.

Tässä luvussa tutustumme työkaluihin, joiden avulla
voimme arvioida algoritmien tehokkuutta.
Keskeinen käsite on \emph{aikavaativuus}, joka antaa
tiiviissä muodossa kuvan algoritmin ajankäytöstä.
Aikavaativuuden avulla voimme muodostaa pika-arvion
algoritmin tehokkuudesta sen rakenteen perusteella,
eikä meidän tarvitse toteuttaa ja testata algoritmia
vain saadaksemme tietää, miten nopea se on.

\section{Aikavaativuus}

Algoritmin tehokkuus riippuu siitä,
montako askelta se suorittaa.
Tavoitteemme on nyt arvioida tehokkuutta suhteessa
syötteen kokoon $n$.
Esimerkiksi jos syötteenä on taulukko,
$n$ on taulukon koko,
ja jos syötteenä on merkkijono,
$n$ on merkkijonon pituus.

Tarkastellaan esimerkkinä seuraavaa algoritmia,
joka laskee, montako kertaa luku $x$ esiintyy
$n$ lukua sisältävässä taulukossa.

\begin{code}[numbers=left]
laskuri = 0
for i = 1 to n
    if luvut[i] == x
        laskuri++
\end{code}

Tämän algoritmin oleelliset askeleet ovat riveillä
1, 3 ja 4.
Rivi 1 suoritetaan vain kerran algoritmin alussa.
Rivi 3 suoritetaan $n$ kertaa jokaisella silmukan
kierroksella.
Rivi 4 taas suoritetaan $0 \dots n$
kertaa riippuen siitä, kuinka usein
luku $x$ esiintyy taulukossa.
Algoritmi suorittaa siis vähintään $n+1$ ja enintään $2n+1$
askelta.

Näin tarkka analyysi ei ole kuitenkaan yleensä tarpeen,
vaan meille riittää määrittää karkea ajankäytön yläraja.
Sanomme, että algoritmi toimii ajassa $O(f(n))$ eli sen
\emph{aikavaativuus} on $O(f(n))$, jos se suorittaa
enintään $c f(n)$ askelta aina silloin kun $n \ge n_0$,
missä $c$ ja $n_0$ ovat vakioita.
Esimerkiksi yllä oleva algoritmi toimii ajassa $O(n)$,
koska se suorittaa selkeästi enintään $3n$ askelta
kaikilla $n$:n arvoilla.

Aikavaativuuden mukavana puolena on, että voimme yleensä
päätellä aikavaativuuden helposti algoritmin
rakenteesta. Tutustumme seuraavaksi laskusääntöihin,
joiden avulla tämä on mahdollista.

\subsection{Laskusääntöjä}

Jos koodissa ei ole silmukoita vaan vain
yksittäisiä komentoja, sen aikavaativuus on $O(1)$.
Näin on esimerkiksi seuraavassa koodissa:

\begin{code}
c = a+b
b = a
a = c
\end{code}

Merkitsemme \texttt{...} koodia,
jonka aikavaativuus on $O(1)$.
Jos koodissa on yksi silmukka,
joka suorittaa $n$ askelta,
sen aikavaativuus on $O(n)$:

\begin{code}
for i = 1 to n
    ...
\end{code}

Jos tällaisia silmukoita on kaksi sisäkkäin,
aikavaativuus on $O(n^2)$:

\begin{code}
for i = 1 to n
    for j = 1 to n
        ...
\end{code}

Yleisemmin jos koodissa on vastaavalla tavalla
$k$ sisäkkäistä silmukkaa, sen aikavaativuus on $O(n^k)$.

Huomaa, että vakiokertoimet ja matalammat termit eivät vaikuta aikavaativuuteen.
Esimerkiksi seuraavissa koodeissa silmukoissa on $2n$ ja $n-1$ askelta,
mutta kummankin koodin aikavaativuus on $O(n)$.

\begin{code}
for i = 1 to 2*n
    ...
\end{code}

\begin{code}
for i = 1 to n-1
    ...
\end{code}

Jos koodissa on peräkkäisiä osuuksia, kokonaisaikavaativuus on suurin
yksittäinen aikavaativuus. Esimerkiksi seuraavan koodin aikavaativuus on $O(n^2)$,
koska sen osuudet ovat $O(n)$, $O(n^2)$ ja $O(n)$.

\begin{code}
for i = 1 to n
    ...
for i = 1 to n
    for j = 1 to n
        ...
for i = 1 to n
    ...
\end{code}

Joskus aikavaativuus riippuu useammasta tekijästä,
jolloin kaavassa on monta muuttujaa.
Seuraavan koodin aikavaativuus on $O(nm)$:

\begin{code}
for i = 1 to n
    for j = 1 to m
        ...
\end{code}

Rekursiivisessa algoritmissa laskemme,
montako rekursiivista kutsua teh\-dään ja kauanko
kutsut vievät aikaa.
Tarkastellaan esimerkkinä seuraavia
algoritmeja, joita kutsutaan parametrilla $n$:

\begin{code}
algo1(n)
    if n == 1
        return
    algo1(n-1)
\end{code}

\begin{code}
algo2(n)
    if n == 1
        return
    algo2(n-1)
    algo2(n-1)
\end{code}

Ensimmäisessä algoritmissa rekursiivisia kutsuja on $O(n)$
ja jokainen kutsu vie aikaa $O(1)$,
joten aikavaativuus on $O(n)$.
Toisessa algoritmissa rekursiivisia kutsuja on $O(2^n)$,
koska jokainen kutsu tuottaa kaksi uutta kutsua,
ja algoritmin aikavaativuus on $O(2^n)$.

\subsection{Yleisiä aikavaativuuksia}

Tietyt aikavaativuudet esiintyvät usein algoritmien analyysissa.
Seuraavaksi käymme läpi joukon tällaisia aikavaativuuksia.

\subsubsection{$O(1)$ (vakioaikainen)}

\emph{Vakioaikainen} algoritmi suorittaa kiinteän määrän komentoja,
eikä syötteen suuruus vaikuta algoritmin nopeuteen.
Esimerkiksi seuraava algoritmi laskee summan $1+2+\dots+n$
vakioajassa summakaavan $n(n+1)/2$ avulla:

\begin{code}
summa = n*(n+1)/2
\end{code}

\subsubsection{$O(\log n)$ (logaritminen)}

\emph{Logaritminen} algoritmi puolittaa usein syötteen koon
joka askeleella.
Esimerkiksi seuraavan algoritmin aikavaativuus on $O(\log n)$:

\begin{code}
while n > 0
    n /= 2
\end{code}

Tärkeä seikka logaritmeihin liittyen on, että
$\log n$ on \emph{pieni} luku, kun $n$ on mikä tahansa 
tyypillinen algoritmeissa esiintyvä luku.
Esimerkiksi $\log 10^6 \approx 20$ ja $\log 10^9 \approx 30$.
Niinpä jos algoritmi tekee jotain logaritmista,
tässä ei kulu kauan aikaa.

\subsubsection{$O(n)$ (lineaarinen)}

\emph{Lineaarinen} algoritmi voi käydä läpi syötteen kiinteän määrän kertoja.
Esimerkiksi seuraava algoritmi laskee taulukon lukujen summan $O(n)$-ajassa.

\begin{code}
summa = 0
for i = 0 to n-1
    summa += taulu[i]
\end{code}

Usein lineaarinen aikavaativuus on paras mahdollinen,
minkä voimme saavuttaa,
koska algoritmin täytyy käydä syöte ainakin
kerran läpi.

\subsubsection{$O(n \log n)$ (järjestäminen)}

Aikavaativuus $O(n \log n)$ viittaa usein siihen,
että algoritmin osana on \emph{järjes\-tämistä},
koska tehokkaat järjestämisalgoritmit
toimivat ajassa $O(n \log n)$.
Esimerkiksi seuraava $O(n \log n)$-aikainen
algoritmi tarkastaa, onko taulukossa kahta samaa alkiota:

\begin{code}
sort(taulu)
ok = false
for i = 1 to n-1
    if taulu[i] == taulu[i-1]
        ok = true
\end{code}

Algoritmi järjestää ensin taulukon, minkä jälkeen yhtä
suuret alkiot ovat \emph{vierekkäin} ja ne on helppoa löytää.
Järjestäminen vie aikaa $O(n \log n)$
ja silmukka vie aikaa $O(n)$, joten algoritmi vie
yhteensä aikaa $O(n \log n)$.

\subsubsection{$O(n^2)$ (neliöllinen)}

\emph{Neliöllinen} algoritmi voi käydä läpi kaikki tavat valita
kaksi alkiota syötteestä.
Esimerkiksi seuraava $O(n^2)$-algoritmi tutkii, onko taulukossa
kahta lukua, joiden summa on $x$.

\begin{code}
ok = false
for i = 0 to n-1
    for j = i+1 to n-1
        if taulu[i]+taulu[j] == x
            ok = true
\end{code}

\subsubsection{$O(n^3)$ (kuutiollinen)}

\emph{Kuutiollinen} algoritmi voi käydä läpi kaikki tavat valita
kolme alkiota syöt\-teestä.
Esimerkiksi seuraava $O(n^3)$-algoritmi tutkii, onko taulukossa
kolmea lukua, joiden summa on $x$.

\begin{code}
ok = false
for i = 0 to n-1
    for j = i+1 to n-1
        for k = j+1 to n-1
            if taulu[i]+taulu[j]+taulu[k] == x
                ok = true
\end{code}

\subsubsection{$O(2^n)$ (osajoukot)}

Aikavaativuus $O(2^n)$ viittaa usein siihen,
että algoritmi käy läpi syötteen alkioiden osajoukot.

\subsubsection{$O(n!)$ (permutaatiot)}

Aikavaativuus $O(n!)$ viittaa usein siihen,
että algoritmi käy läpi syötteen alkioiden permutaatiot.

\subsection{Tehokkuuden arviointi}

Mitä hyötyä on määrittää algoritmin aikavaativuus?
Hyötynä on, että aikavaativuus antaa meille pika-arvion siitä,
kuinka \emph{hyvä} algoritmi on eli miten suuria syötteitä
sillä voi käsitellä tehokkaasti.
Kun meille kertyy kokemusta algoritmien suunnittelusta,
meille alkaa muodostua selkeä kuva,
mitä eri aikavaativuudet tarkoittavat käytännössä.

Aikavaativuutta voi ajatella samalla tavalla kuin vaikkapa
hotellin tähti\-luokitusta: se kertoo tiiviissä muodossa,
mistä asiassa on kysymys, eikä mei\-dän tarvitse ottaa selvää yksityiskohdista.
Jos meille tarjotaan majoitusta neljän tähden hotellissa,
saamme heti jonkin käsityksen huoneen tasosta
tähtiluokituksen ansiosta,
vaikka emme saisi tarkkaa listausta huoneen varustelusta.
Vastaavasti jos kuulemme, että jonkin algoritmin aikavaativuus on $O(n \log n)$,
voimme heti arvioida karkeasti, miten suuria syötteitä voimme käsitellä,
vaikka emme tuntisi tarkemmin algoritmin toimintaa

\begin{table}
\center
\begin{tabular}{rrr}
syötteen kokoluokka $n$ & tarvittava aikavaativuus \\
\hline
10 & $O(n!)$ \\
20 & $O(2^n)$ \\
500 & $O(n^3)$ \\
5000 & $O(n^2)$ \\
$10^6$ & $O(n \log n)$ \\
$10^7$ & $O(n)$ \\
suuri & $O(1)$ tai $O(\log n)$ \\
\end{tabular}
\caption{Kuinka suuren syötteen algoritmi voi käsitellä nopeasti?}
\label{tab:algteh}
\end{table}

Yksi kiinnostava näkökulma algoritmin tehokkuuteen on,
miten suuren syötteen algoritmi voi käsitellä nopeasti.
Tässä ''nopeasti'' tarkoittaa, että algoritmi käsittelee
sille annetun syötteen hetkessä.
Tämä on hyvä vaatimus, kun haluamme käyttää algoritmia
jossakin käytännön sovelluksessa.
Taulukossa \ref{tab:algteh} on joitakin hyödyllisiä arvioita,
kun Java-kielellä toteutettu
algoritmi suoritetaan nykyaikaisella tietokoneella.
Esimerkiksi jos meillä on $O(n^2)$-algoritmi, voimme käsitellä sillä
nopeasti syötteen, jossa on luokkaa 5000 alkiota.
Jos haluamme käsitellä tehokkaasti suurempia syötteitä,
meidän tulisi löytää $O(n)$- tai $O(n \log n)$-aikainen algoritmi.

Kannattaa silti pitää mielessä, että nämä luvut ovat vain arvioita ja algoritmin
todelliseen ajankäyttöön vaikuttavat monet asiat.
Saman algoritmin hyvä toteutus saattaa olla
kymmeniä kertoja nopeampi kuin huono toteutus,
ja suuri merkitys on myös ohjelmointikielellä,
jolla algoritmi on toteutettu.
Tässä kirjassa analysoimme algoritmeja sekä aikavaativuuksien
avulla että mittaamalla todellisia suoritusaikoja.

\subsection{Esimerkki: Bittijonot}

Seuraavassa esimerkissä vertailemme kahta
erilaista algoritmia samaan tehtä\-vään.
Ensimmäinen algoritmi on suoraviivainen raa'an voiman
algoritmi, joka toimii ajassa $O(n^2)$.
Toinen algoritmi taas on tehokas algoritmi,
joka toimii ajassa $O(n)$.

Tehtävämme on seuraava: Annettuna on bittijono, jossa on $n$ bittiä,
ja haluamme laskea, monellako tavalla voimme valita kaksi kohtaa
niin, että vasen bitti on 0 ja oikea bitti on 1.
Esimerkiksi bittijonossa 01001 tapoja on neljä:
\underline{01}001, \underline{0}100\underline{1},
01\underline{0}0\underline{1} ja 010\underline{01}.

\subsubsection{$O(n^2)$-ratkaisu}

Voimme ratkaista tehtävän raa'alla voimalla
käymällä läpi kaikki mahdolliset tavat valita vasen ja oikea kohta.
Tällöin voimme laskea yksi kerrallaan,
monessako tavassa vasen bitti on 0 ja oikea bitti on 1.
Seuraava koodi toteuttaa algoritmin olettaen,
että taulukko \texttt{bitit} sisältää $n$ bittiä.

\begin{code}
laskuri = 0
for i = 0 to n-1
    for j = i+1 to n-1
        if bitit[i] == 0 and bitit[j] == 1
            laskuri++
print(laskuri)
\end{code}

Algoritmin aikavaativuus on $O(n^2)$, koska siinä on kaksi
sisäkkäistä silmukkaa, jotka käyvät läpi syötteen.

\subsubsection{$O(n)$-ratkaisu}

Kuinka voisimme ratkaista tehtävän tehokkaammin?
Meidän tulisi käytän\-nössä keksiä tapa, jolla saisimme
pois toisen silmukan koodista.

Tässä auttaa lähestyä ongelmaa hieman toisesta
näkökulmasta: kun olemme tietyssä kohdassa bittijonoa,
monellako tavalla voimme muodostaa parin,
jonka oikea bitti on nykyisessä kohdassamme?
Jos olemme bitin 0 kohdalla, pareja ei ole yhtään,
mutta jos bittinä on 1, voimme valita minkä tahansa
vasemmalla puolella olevan bitin 0 pariin.

Tämän havainnon ansiosta meidän riittää käydä läpi
bittijono kerran vasemmalta oikealle ja pitää kirjaa,
montako bittiä 0 olemme nähneet.
Sitten jokaisen bitin 1 kohdalla kasvatamme
vastausta tällä bittien 0 määrällä.
Seuraava koodi toteuttaa algoritmin:

\begin{code}
laskuri = 0
nollat = 0
for i = 0 to n-1
    if bitit[i] == 0
        nollat++
    else
        laskuri += nollat
print(laskuri)
\end{code}

Algoritmissa on vain yksi silmukka, joka käy syötteen läpi,
joten sen aikavaativuus on $O(n)$.

\subsubsection{Algoritmien vertailua}

Meillä on nyt siis kaksi algoritmia, joiden aikavaativuudet ovat
$O(n^2)$ ja $O(n)$, mutta mitä tämä tarkoittaa käytännössä?
Tämän selvittämiseksi teimme testin,
jossa toteutimme algoritmit Javalla ja mittasimme niiden
suoritusaikoja satunnaisilla syötteillä eri $n$:n arvoilla.

Taulukko \ref{tab:algver} näyttää tällaisen testin tulokset.
Pienillä $n$:n arvoilla molemmat algoritmit toimivat
hyvin tehokkaasti, mutta suuremmilla syötteillä on
nähtävissä huomattavia eroja.
Raakaan voimaan perustuva $O(n^2)$-algoritmi
alkaa hidastua selvästi testistä $n=10000$ alkaen,
ja testissä $n=1000000$ sillä kuluu jo lähes kolme minuuttia aikaa.
Tehokas $O(n)$-algoritmi taas selvittää suuretkin testit
salamannopeasti.

\begin{table}
\center
\begin{tabular}{rrr}
syötteen koko $n$ & $O(n^2)$-algoritmi & $O(n)$-algoritmi \\
\hline
$10$ & 0.00 s & 0.00 s\\
$100$ & 0.00 s & 0.00 s\\
$1000$ & 0.00 s & 0.00 s\\
$10000$ & 0.14 s & 0.00 s \\
$100000$ & 1.66 s & 0.00 s \\
$1000000$ & 172.52 s & 0.01 s \\
\end{tabular}
\caption{Algoritmien suoritusaikojen vertailu.}
\label{tab:algver}
\end{table}

Tämän kurssin jatkuvana teemana on luoda algoritmeja,
jotka toimivat tehokkaasti myös silloin, kun niille annetaan suuria syötteitä.
Tämä tarkoittaa käytännössä sitä, että algoritmin aikavaativuuden tulisi
olla $O(n)$ tai $O(n \log n)$.
Jos algoritmin aikavaativuus on esimerkiksi $O(n^2)$,
se on auttamatta liian hidas suurien syötteiden käsittelyyn.

\section{Lisää algoritmien analysoinnista}

Aikavaativuuksissa esiintyvä $O$-merkintä on yksi monista merkinnöistä,
joiden avulla voimme arvioida funktioiden kasvunopeutta.
Tutustumme seuraavaksi tarkemmin näihin merkintöihin.

\subsection{Merkinnät $O$, $\Omega$ ja $\Theta$}

Algoritmien analysoinnissa usein esiintyviä merkintöjä ovat:

\begin{itemize}
\item \emph{Yläraja}: Funktio $g(n)$ on luokkaa $O(f(n))$, jos on olemassa vakiot $c$ ja $n_0$
niin, että $g(n) \le c f(n)$ aina kun $n \ge n_0$.
\item \emph{Alaraja}: Funktio $g(n)$ on luokkaa $\Omega(f(n))$, jos on olemassa vakiot $c$ ja $n_0$
niin, että $g(n) \ge c f(n)$ aina kun $n \ge n_0$.
\item \emph{Tarkka arvio}: Funktio $g(n)$ on luokkaa $\Theta(f(n))$, jos se on sekä luokkaa $O(f(n))$
että luokkaa $\Omega(f(n))$.
\end{itemize}

Vakion $c$ tarkoituksena on, että saamme arvion kasvunopeuden suuruusluokalle välittämättä
vakiokertoimista. Vakion $n_0$ tarkoituksena on, että keskitymme tarkastelemaan
kasvunopeutta suurilla $n$:n arvoilla.
Voimme myös kirjoittaa $g(n)=O(f(n))$, kun haluamme ilmaista,
että funktio $g(n)$ on luokkaa $O(f(n))$,
ja vastaavasti $\Omega$- ja $\Theta$-merkinnöissä.

Kun sanomme, että algoritmi toimii ajassa $O(f(n))$, tarkoitamme, että se suorittaa
\emph{pahimmassa tapauksessa} $O(f(n))$ askelta.
Tämä on yleensä hyvä tapa ilmoittaa algoritmin tehokkuus,
koska silloin annamme takuun siitä, että algoritmin ajankäytöllä on tietty yläraja,
vaikka syöte olisi valittu mahdollisimman ikävästi algoritmin kannalta.

Tarkastellaan esimerkkinä seuraavaa algoritmia, joka laskee
taulukon lukujen summan:

\begin{code}
summa = 0
for i = 0 to n-1
    summa += taulu[i]
\end{code}

Tämä algoritmi toimii samalla tavalla riippumatta taulukon sisällöstä,
koska se käy aina läpi koko taulukon.
Niinpä yläraja ajankäytölle on $O(n)$ ja alaraja ajankäytölle on samoin $\Omega(n)$,
joten voimme sanoa, että algoritmi vie aikaa $\Theta(n)$ kaikissa tapauksissa.

Tarkastellaan sitten seuraavaa algoritmia, joka selvittää,
onko taulukossa lukua $x$:

\begin{code}
ok = false
for i = 0 to n-1
    if taulu[i] == x
        ok = true
        break
\end{code}

Tässä algoritmin pahin ja paraus tapaus eroavat.
Ajankäytön yläraja on $O(n)$, koska algoritmi joutuu käymään
läpi kaikki taulukon alkiot silloin, kun luku $x$
ei esiinny taulukossa.
Toisaalta ajankäytön alaraja on $\Omega(1)$,
koska jos luku $x$ on taulukon ensimmäinen alkio,
algoritmi pysähtyy heti taulukon alussa.
Voimme myös sanoa, että algoritmi suorittaa pahimmassa
tapauksessa $\Theta(n)$ askelta ja parhaassa tapauksessa
$\Theta(1)$ askelta.

Tärkeä huomio on, että $O$-merkinnän antama yläraja voi olla
\emph{mikä tahansa} yläraja, ei välttämättä tarkka yläraja.
On siis oikein sanoa esimerkiksi, että äskeinen
algoritmi vie aikaa $O(n^2)$, vaikka on olemassa parempi yläraja $O(n)$.
Miksi sitten käytämme $O$-merkintää, vaikka voisimme usein myös ilmaista tarkan
ajankäytön $\Theta$-merkinnällä?
Tämä on vakiintunut ja käytännössä toimiva tapa.
Olisi hyvin harhaanjohtavaa antaa algoritmille yläraja $O(n^2)$,
jos näemme suoraan, että aikaa kuluu vain $O(n)$.

Asiaa voi ajatella niin, että $O$-merkintää käytetään algoritmin
\emph{markkinoinnissa}. Jos annamme liian suuren ylärajan, algoritmista
tulee väärä käsitys yleisölle.
Vertauksena jos myymme urheiluautoa, jonka huippunopeus on 250 km/h,
on sinänsä paikkansa pitävä väite, että autolla pystyy ajamaan 100 km/h.
Meidän ei kuitenkaan kannata antaa tällaista vähättelevää tietoa,
vaan kertoa, että autolla pystyy ajamaan 250 km/h.

\subsection{Tilavaativuus}

Merkintöjä $O$, $\Omega$ ja $\Theta$ voi käyttää
kaikenlaisissa yhteyksissä, ei vain algoritmin ajankäytön arvoinnissa.
Esimerkiksi voimme sanoa, että algoritmi suorittaa silmukkaa $O(\log n)$ kierrosta
tai että taulukossa on $O(n^2)$ lukua.

Aikavaativuuden lisäksi kiinnostava tieto algoritmista on sen
\emph{tilavaativuus}. Tämä tarkoittaa yleensä sitä, miten paljon algoritmi
käyttää muistia syötteen \emph{lisäksi}.
Jos tilavaativuus on $O(1)$, algoritmi tarvitsee muistia
vain yksittäisille muuttujille syötteen lisäksi.
Jos tilavaativuus on $O(n)$, algoritmi voi varata esimerkiksi aputaulukon,
jonka koko vastaa syötteen kokoa.

Tarkastellaan esimerkkinä tehtävää, jossa meille annetaan taulukko,
joka sisältää luvut $1,2,\dots,n$ yhtä lukuun ottamatta,
ja tehtävämme on selvittää puuttuva luku.
Yksi tapa ratkaista tehtävä $O(n)$-ajassa on luoda aputaulukko,
joka pitää kirjaa mukana olevista luvuista.
Tällaisen ratkaisun tilavaativuus on $O(n)$,
koska aputaulukko vie $O(n)$ muistia.

\begin{code}
for i = 0 to n-2
    mukana[taulu[i]] = true
for i = 1 to n
    if not mukana[i]
        puuttuva = i
\end{code}

Tehtävään on kuitenkin olemassa myös toinen algoritmi,
jossa aikavaativuus on edelleen $O(n)$ mutta tilavaativuus on vain $O(1)$.
Tällainen algoritmi laskee ensin lukujen $1,2,\dots,n$ summan
ja vähentää sitten taulukossa esiintyvät luvut siitä.
Jäljelle jäävä luku on puuttuva luku.

\begin{code}
summa = n*(n+1)/2
for i = 0 to n-2
    summa -= taulu[i]
puuttuva = summa
\end{code}

Käytännössä tilavaativuus on yleensä sivuroolissa algoritmeissa,
koska jos algoritmi vie vain vähän aikaa, se ei \emph{ehdi} käyttää kovin paljon muistia.
Erityisesti tilavaativuus ei voi olla suurempi kuin aikavaativuus.
Niinpä meidän riittää tavallisesti keskittyä suunnittelemaan algoritmeja,
jotka toimivat nopeasti, ja vertailla algoritmien aikavaativuuksia.

\subsection{Rajojen todistaminen}

Jos haluamme todistaa täsmällisesti, että jokin raja pätee,
meidän täytyy löytää vakiot $c$ ja $n_0$, jotka osoittavat asian.
Jos taas haluamme todistaa, että raja ei päde,
meidän täytyy näyttää, että mikään vakioiden $c$ ja $n_0$ valinta ei ole kelvollinen.

Jos haluamme todistaa rajan pätemisen,
tämä onnistuu yleensä helposti valitsemalla vakio $c$
tarpeeksi suureksi ja arvioimalla summan osia ylöspäin tarvittaessa.
Esimerkiksi jos haluamme todistaa, että $3n+5 = O(n)$, meidän tulee löytää
vakiot $c$ ja $n_0$, joille pätee, että $3n+5 \le cn$ aina kun $n \ge n_0$.
Tässä tapauksessa voimme valita esimerkiksi $c=8$ ja $n_0=1$,
jolloin voimme arvioida $3n+5 \le 3n+5n=8n$, kun $n \ge 1$.

Jos haluamme todistaa, että raja ei päde, tilanne on hankalampi,
koska meidän täytyy näyttää, että ei ole olemassa \emph{mitään} kelvollista
tapaa valita vakioita $c$ ja $n_0$.
Tässä auttaa tyypillisesti vastaoletuksen tekeminen: oletamme,
että raja pätee ja voimme valita vakiot,
ja näytämme sitten, että tämä oletus johtaa ristiriitaan.

Todistetaan esimerkkinä, että $n^2 \neq O(n)$.
Jos pätisi $n^2=O(n)$, niin olisi olemassa vakiot $c$ ja $n_0$,
joille $n^2 \le cn$ aina kun $n \ge n_0$.
Voimme kuitenkin osoittaa, että tämä aiheuttaa ristiriidan.
Jos $n^2 \le cn$, niin voimme jakaa epäyhtälön molemmat puolet $n$:llä
ja saamme $n \le c$.
Tämä tarkoittaa, että $n$ on aina enintään yhtä suuri kuin vakio $c$.
Tämä ei ole kuitenkaan mahdollista, koska $n$ voi olla miten
suuri tahansa, joten ei voi päteä $n^2 = O(n)$.

Määritelmistä lähtevä todistaminen on sinänsä mukavaa ajanvietettä,
mutta sille on äärimmäisen harvoin tarvetta käytännössä,
kun haluamme tutkia algoritmien tehokkuutta.
Voimme koko kurssin ajan huoletta päätellä algoritmin aikavaativuuden
katsomalla, mikä sen rakenne on, kuten olemme tehneet tämän luvun alkuosassa.