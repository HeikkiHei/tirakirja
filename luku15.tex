\chapter{NP-ongelmat}

Olemme tässä kirjassa tutustuneet menetelmiin, joiden avulla voimme
ratkaista tehokkaasti ongelmia.
Kuitenkin on myös monia ongelmia, joiden ratkaisemiseen ei tällä
hetkellä tunneta mitään tehokasta algoritmia.
Jos vastaamme tulee tällaisia ongelmia, hyvät neuvot ovat kalliit.

Vaikeiden ongelmien yhteydessä vastaan tulee usein
kirjainyhdistelmä $NP$.
Erityisen tunnettu on $P$ vs. $NP$ -ongelma,
jonka ratkaisijalle on luvattu miljoonan dollarin potti.
Hankalalta tuntuvasta ongelmasta saatetaan mainita, että
se on $NP$-täydellinen tai $NP$-vaikea.
Nyt on aika selvittää, mitä nämä käsitteet oikeastaan tarkoittavat.

\section{Luokat $P$ ja $NP$}

Kun tässä luvussa käsittelemme algoritmisia ongelmia,
muotoilemme ne niin, että voimme esittää ne \emph{päätösongelmina}.
Tämä tarkoittaa, että algoritmin tulee antaa jokaiselle syötteelle
vastaus ''kyllä'' tai ''ei''.
Tämä voi tuntua aluksi hieman rajoittavalta, mutta huomaamme myöhemmin,
että voimme esittää helposti monenlaisia ongelmia päätösongelmina.

\subsection{Luokka $P$}

Luokka $P$ sisältää päätösongelmat, joiden ratkaisemiseen on
\emph{polynominen} algoritmi eli algoritmi, jonka aikavaativuus
on muotoa $O(n^k)$, missä $k$ on vakio.
Lähes kaikki tässä kirjassa esitetyt algoritmit toimivat
polynomisessa ajassa.
Tuttuja polynomisia aikavaativuuksia ovat esimerkiksi
$O(1)$, $O(\log n)$ $O(n)$, $O(n \log n)$, $O(n^2)$ ja $O(n^3)$.

Esimerkki luokkaan $P$ kuuluvasta ongelmasta on
''onko taulukossa kahta samaa alkiota?''.
Tämä ongelma kuuluu luokkaan $P$, koska voimme ratkaista
sen monellakin tavalla polynomisessa ajassa.
Helpoin tapa on käydä läpi kaikki tavat valita taulukosta
kaksi alkiota ja tarkastaa ovatko jotkin kaksi samat,
jolloin tuloksena on ajassa $O(n^2)$ toimiva algoritmi.

Luokan $P$ tarkoituksena on kuvata ongelmia, jotka voimme
ratkaista \emph{tehokkaasti}.
Tässä tehokkuuden määritelmä on varsin karkea:
pidämme algoritmia tehokkaana, jos sillä on mikä tahansa
polynominen aikavaativuus.
Onko $O(n^{100})$-aikainen algoritmi siis tehokas?
Ei, mutta tällaisia algoritmeja ei esiinny käytännössä
eikä meidän tarvitse murehtia asiasta.

\subsection{Luokka $NP$}

Luokka $NP$ sisältää päätösongelmat, joissa jokaiseen
algoritmin antamaan ''kyllä''-vastaukseen voidaan liittää
polynomisen kokoinen \emph{todiste}, jonka avulla voimme
\emph{tarkastaa} polynomisessa ajassa, että vastaus on oikein.
Todiste antaa meille lisätietoa siitä,
minkä takia algoritmin ''kyllä''-vastaus
pitää paikkansa syötteelle.

Esimerkki luokkaan $NP$ kuuluvasta ongelmasta on
\emph{kauppamatkustajan ongelma}: onko verkossa reittiä,
joka alkaa jostakin solmusta, käy kerran kaikissa muissa
solmuissa ja palaa lähtösolmuun?
Tämän ongelman ratkaisemiseen ei tunneta tehokasta algoritmia,
mutta jokaiseen ''kyllä''-tapaukseen liittyy todiste,
jonka voimme tarkastaa tehokkaasti.
Todisteena on reitti, jota kauppamatkustaja noudattaa.
Voimme tarkastaa tehokkaasti, että reitti kulkee verkon
kaaria pitkin ja käy kerran jokaisessa solmussa.

Jos algoritmi antaa ''ei''-vastauksen, tähän ei tarvitse
liittyä tehokkaasti tarkastettavaa todistetta.
Usein olisikin hankalaa antaa todiste siitä, että jotain
asiaa \emph{ei} ole olemassa.
Esimerkiksi kauppamatkustajan ongelmassa on helppoa
todistaa, että reitti on olemassa, koska voimme vain
näyttää reitin, mutta vaikeaa todistaa, että reittiä ei ole olemassa.

Huomaa, että kaikki luokan $P$ ongelmat kuuluvat myös
luokkaan $NP$. Tämä johtuu siitä, että luokan $P$ ongelmissa
voimme tarkastaa ''kyllä''-vastauksen
\emph{tyhjän} todisteen avulla: voimme saman tien ratkaista
koko ongelman alusta alkaen polynomisessa ajassa.

\subsection{$P$ vs. $NP$}

\section{$NP$-täydellisyys}