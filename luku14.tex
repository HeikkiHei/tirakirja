\chapter{Virtauslaskenta}

Lähtökohtanamme on suunnattu verkko, jossa on kaksi erityistä solmua:
\emph{lähtösolmu} ja \emph{kohdesolmu}.
Jokaisella kaarella on paino, joka rajoittaa,
miten paljon virtausta voi kulkea kaarta pitkin,
ja haluamme selvittää, mikä on \emph{maksimivirtaus} lähtösolmusta
kohdesolmuun.

Lähtökohtanamme on suunnattu verkko, jossa on kaksi erityistä solmua:
\emph{lähtösolmu} ja \emph{kohdesolmu}.
Jokaisella kaarella on paino, joka rajoittaa,
miten paljon virtausta voi kulkea kaarta pitkin,
ja haluamme selvittää, mikä on \emph{maksimivirtaus} lähtösolmusta
kohdesolmuun.

\section{Maksimivirtaus}

Maksimivirtausongelmassa haluamme selvittää
suurimman \emph{virtauksen} suunnatun verkon
lähtösolmusta kohdesolmuun.
Virtaus lähtee liikkeelle lähtö\-solmusta ja saapuu kohdesolmuun
niin, että jokaiseen välisolmuun tuleva virtaus on
yhtä suuri kuin solmusta lähtevä virtaus.
Jokaisella kaarella on \emph{kapasiteetti}, jota virtauksen
määrä kaarta pitkin ei saa ylittää.

\begin{figure}
\center
\begin{center}
\begin{tikzpicture}[scale=0.8,label distance=-1.5mm]
\node[draw, circle] (1) at (0,-1) {$1$};
\node[draw, circle] (2) at (2,0) {$2$};
\node[draw, circle] (3) at (2,-2) {$3$};
\node[draw, circle] (4) at (4,0) {$4$};
\node[draw, circle] (5) at (4,-2) {$5$};
\path[draw,thick,->] (1) -- node[font=\small,label=above:$4/4$] {} (2);
\path[draw,thick,->] (1) -- node[font=\small,label=below:$3/6$] {} (3);
\path[draw,thick,->] (2) -- node[font=\small,label=right:$1/8$] {} (3);
\path[draw,thick,->] (2) -- node[font=\small,label=above:$3/3$] {} (4);
\path[draw,thick,->] (3) -- node[font=\small,label=below:$4/4$] {} (5);
\path[draw,thick,->] (4) -- node[font=\small,label=right:$3/5$] {} (5);
\end{tikzpicture}
\end{center}
\caption{Maksimivirtaus solmusta $1$ solmuun $5$ on $7$.}
\label{fig:makvir}
\end{figure}

Tarkastellaan esimerkkinä kuvan \ref{fig:makvir} verkkoa.
Tässä verkossa maksimivirtaus solmusta $1$ solmuun $5$ on $7$.
Jokaisessa kaaressa merkintä $v/k$ tarkoittaa,
että kaaren kautta kulkee virtausta $v$ ja
kaaren kapasiteetti on $k$.
Solmusta $1$ lähtevä virtauksen määrä on $4+3=7$,
solmuun $5$ saapuva virtauksen määrä on $3+4=7$
ja kaikissa muissa solmuissa saapuva virtaus
on yhtä suuri kuin lähtevä virtaus.

\subsection{Ford-Fulkersonin algoritmi}

Ford-Fulkersonin algoritmi on tunnetuin menetelmä verkon
maksimivirtauksen etsimiseen,
ja tutustumme seuraavaksi tämän algoritmin toimintaan.
Algoritmi etsii polkuja lähtösolmusta kohdesolmuun,
jotka kasvattavat virtausta pikkuhiljaa.
Kun mitään polkua ei voi enää muodostaa, algoritmi on onnistunut
muodostamaan maksimivirtauksen.

\begin{figure}
\center
\begin{center}
\begin{tikzpicture}[scale=0.9,label distance=-1.5mm]
\node[draw, circle] (1) at (0,-1) {$1$};
\node[draw, circle] (2) at (2,0) {$2$};
\node[draw, circle] (3) at (2,-2) {$3$};
\node[draw, circle] (4) at (4,0) {$4$};
\node[draw, circle] (5) at (4,-2) {$5$};
\path[draw,thick,->] (1) edge [bend left=20] node[font=\small,label=above:$4$] {} (2);
\path[draw,thick,->] (2) edge [bend left=20] node[font=\small,label=above:$0$] {} (1);
\path[draw,thick,->] (1) edge [bend left=20] node[font=\small,label=below:$6$] {} (3);
\path[draw,thick,->] (3) edge [bend left=20] node[font=\small,label=below:$0$] {} (1);
\path[draw,thick,->] (2) edge [bend left=20] node[font=\small,label=right:$8$] {} (3);
\path[draw,thick,->] (3) edge [bend left=20] node[font=\small,label=right:$0$] {} (2);
\path[draw,thick,->] (2) edge [bend left=20] node[font=\small,label=above:$3$] {} (4);
\path[draw,thick,->] (4) edge [bend left=20] node[font=\small,label=above:$0$] {} (2);
\path[draw,thick,->] (3) edge [bend left=20] node[font=\small,label=below:$4$] {} (5);
\path[draw,thick,->] (5) edge [bend left=20] node[font=\small,label=below:$0$] {} (3);
\path[draw,thick,->] (4) edge [bend left=20] node[font=\small,label=right:$5$] {} (5);
\path[draw,thick,->] (5) edge [bend left=20] node[font=\small,label=right:$0$] {} (4);
\end{tikzpicture}
\end{center}
\caption{Verkon esitysmuoto Ford-Fulkersonin algoritmissa.}
\label{fig:floesi}
\end{figure}

Algoritmin käyttäminen vaatii, että verkko on esitetty
erityisessä muodossa, jossa jokaista alkuperäisen verkon
kaarta vastaa kaksi kaarta:
alkuperäinen kaari, jonka painona on aluksi kaaren kapasiteetti,
sekä sille kään\-teinen kaari, jonka painona on aluksi $0$.
Käänteisten kaarten avulla pystymme tarvittaessa \emph{peruuttamaan}
virtausta algoritmin aikana.
Kuva \ref{fig:floesi} näyttää, kuinka esitämme esimerkkiverkkomme algoritmissa.

Algoritmin jokaisessa vaiheessa muodostamme polun
lähtösolmusta kohdesolmuun.
Polku voi olla mikä tahansa, kunhan jokaisen kaaren paino
polulla on positiivinen.
Polun muodostamisen jälkeen virtaus lähtösolmusta kohdesolmuun
kasvaa $p$:llä, missä $p$ on pienin kaaren paino polulla.
Lisäksi jokaisen polulla olevan kaaren paino vähenee $p$:llä
ja jokaisen niille käänteisen kaaren paino kasvaa $p$:llä.
Etsimme vastaavasti uusia polkuja, kunnes mitään sallittua
polkua ei voi enää muodostaa.

Kuva \ref{fig:floesi} näyttää, kuinka Ford-Fulkersonin algoritmi muodostaa
maksimivirtauksen esimerkkiverkossamme.
Algoritmi muodostaa ensin polun $1 \rightarrow 2 \rightarrow 3 \rightarrow 5$,
jossa pienin paino on $4$.
Tämän seurauksena virtaus kasvaa $4$:llä,
polulla olevien kaarten paino vähenee $4$:llä
ja käänteisten kaarten paino kasvaa $4$:llä.
Tämän jälkeen algoritmi muodostaa polun
$1 \rightarrow 3 \rightarrow 2 \rightarrow 4 \rightarrow 5$,
joka kasvattaa virtausta $3$:lla.
Huomaa, että tämä polku peruuttaa
kaarta $2 \rightarrow 3$ menevää virtausta,
koska se kulkee käänteisen kaaren $3 \rightarrow 2$ kautta.
Tämän jälkeen algoritmi ei enää pysty muodostamaan mitään polkua
solmusta $1$ solmuun $5$, joten maksimivirtaus on $4+3=7$.

\begin{figure}
\center
\begin{center}
\begin{tikzpicture}[scale=0.9,label distance=-1.5mm]
\newcommand\verkko[0]{
\node[draw, circle] (1) at (0,-1) {$1$};
\node[draw, circle] (2) at (2,0) {$2$};
\node[draw, circle] (3) at (2,-2) {$3$};
\node[draw, circle] (4) at (4,0) {$4$};
\node[draw, circle] (5) at (4,-2) {$5$};
}
\begin{scope}
\verkko
\path[draw,thick,->] (1) edge [bend left=20] node[font=\small,label=above:$4$] {} (2);
\path[draw,thick,->] (2) edge [bend left=20] node[font=\small,label=above:$0$] {} (1);
\path[draw,thick,->] (1) edge [bend left=20] node[font=\small,label=below:$6$] {} (3);
\path[draw,thick,->] (3) edge [bend left=20] node[font=\small,label=below:$0$] {} (1);
\path[draw,thick,->] (2) edge [bend left=20] node[font=\small,label=right:$8$] {} (3);
\path[draw,thick,->] (3) edge [bend left=20] node[font=\small,label=right:$0$] {} (2);
\path[draw,thick,->] (2) edge [bend left=20] node[font=\small,label=above:$3$] {} (4);
\path[draw,thick,->] (4) edge [bend left=20] node[font=\small,label=above:$0$] {} (2);
\path[draw,thick,->] (3) edge [bend left=20] node[font=\small,label=below:$4$] {} (5);
\path[draw,thick,->] (5) edge [bend left=20] node[font=\small,label=below:$0$] {} (3);
\path[draw,thick,->] (4) edge [bend left=20] node[font=\small,label=right:$5$] {} (5);
\path[draw,thick,->] (5) edge [bend left=20] node[font=\small,label=right:$0$] {} (4);
\path[draw,thick,->,red,line width=2pt] (1) edge [bend left=20] (2);
\path[draw,thick,->,red,line width=2pt] (2) edge [bend left=20] (3);
\path[draw,thick,->,red,line width=2pt] (3) edge [bend left=20] (5);
\draw[->,thick] (5.5,-1) -- (6.5,-1);
\node at (6,-3) {vaihe 1};
\end{scope}
\begin{scope}[xshift=7.5cm]
\verkko
\path[draw,thick,->] (1) edge [bend left=20] node[font=\small,label=above:$0$] {} (2);
\path[draw,thick,->] (2) edge [bend left=20] node[font=\small,label=above:$4$] {} (1);
\path[draw,thick,->] (1) edge [bend left=20] node[font=\small,label=below:$6$] {} (3);
\path[draw,thick,->] (3) edge [bend left=20] node[font=\small,label=below:$0$] {} (1);
\path[draw,thick,->] (2) edge [bend left=20] node[font=\small,label=right:$4$] {} (3);
\path[draw,thick,->] (3) edge [bend left=20] node[font=\small,label=right:$4$] {} (2);
\path[draw,thick,->] (2) edge [bend left=20] node[font=\small,label=above:$3$] {} (4);
\path[draw,thick,->] (4) edge [bend left=20] node[font=\small,label=above:$0$] {} (2);
\path[draw,thick,->] (3) edge [bend left=20] node[font=\small,label=below:$0$] {} (5);
\path[draw,thick,->] (5) edge [bend left=20] node[font=\small,label=below:$4$] {} (3);
\path[draw,thick,->] (4) edge [bend left=20] node[font=\small,label=right:$5$] {} (5);
\path[draw,thick,->] (5) edge [bend left=20] node[font=\small,label=right:$0$] {} (4);
\end{scope}
\begin{scope}[yshift=-5cm]
\verkko
\path[draw,thick,->] (1) edge [bend left=20] node[font=\small,label=above:$0$] {} (2);
\path[draw,thick,->] (2) edge [bend left=20] node[font=\small,label=above:$4$] {} (1);
\path[draw,thick,->] (1) edge [bend left=20] node[font=\small,label=below:$6$] {} (3);
\path[draw,thick,->] (3) edge [bend left=20] node[font=\small,label=below:$0$] {} (1);
\path[draw,thick,->] (2) edge [bend left=20] node[font=\small,label=right:$4$] {} (3);
\path[draw,thick,->] (3) edge [bend left=20] node[font=\small,label=right:$4$] {} (2);
\path[draw,thick,->] (2) edge [bend left=20] node[font=\small,label=above:$3$] {} (4);
\path[draw,thick,->] (4) edge [bend left=20] node[font=\small,label=above:$0$] {} (2);
\path[draw,thick,->] (3) edge [bend left=20] node[font=\small,label=below:$0$] {} (5);
\path[draw,thick,->] (5) edge [bend left=20] node[font=\small,label=below:$4$] {} (3);
\path[draw,thick,->] (4) edge [bend left=20] node[font=\small,label=right:$5$] {} (5);
\path[draw,thick,->] (5) edge [bend left=20] node[font=\small,label=right:$0$] {} (4);
\path[draw,thick,->,red,line width=2pt] (1) edge [bend left=20] (3);
\path[draw,thick,->,red,line width=2pt] (3) edge [bend left=20] (2);
\path[draw,thick,->,red,line width=2pt] (2) edge [bend left=20] (4);
\path[draw,thick,->,red,line width=2pt] (4) edge [bend left=20] (5);
\draw[->,thick] (5.5,-1) -- (6.5,-1);
\node at (6,-3) {vaihe 2};
\end{scope}
\begin{scope}[yshift=-5cm,xshift=7.5cm]
\verkko
\path[draw,thick,->] (1) edge [bend left=20] node[font=\small,label=above:$0$] {} (2);
\path[draw,thick,->] (2) edge [bend left=20] node[font=\small,label=above:$4$] {} (1);
\path[draw,thick,->] (1) edge [bend left=20] node[font=\small,label=below:$3$] {} (3);
\path[draw,thick,->] (3) edge [bend left=20] node[font=\small,label=below:$3$] {} (1);
\path[draw,thick,->] (2) edge [bend left=20] node[font=\small,label=right:$7$] {} (3);
\path[draw,thick,->] (3) edge [bend left=20] node[font=\small,label=right:$1$] {} (2);
\path[draw,thick,->] (2) edge [bend left=20] node[font=\small,label=above:$0$] {} (4);
\path[draw,thick,->] (4) edge [bend left=20] node[font=\small,label=above:$3$] {} (2);
\path[draw,thick,->] (3) edge [bend left=20] node[font=\small,label=below:$0$] {} (5);
\path[draw,thick,->] (5) edge [bend left=20] node[font=\small,label=below:$4$] {} (3);
\path[draw,thick,->] (4) edge [bend left=20] node[font=\small,label=right:$2$] {} (5);
\path[draw,thick,->] (5) edge [bend left=20] node[font=\small,label=right:$3$] {} (4);
\end{scope}
\end{tikzpicture}
\end{center}
\caption{Esimerkki Ford-Fulkersonin algoritmin toiminnasta.}
\label{fig:floesi}
\end{figure}

\subsection{Yhteys minimileikkaukseen}

Ford-Fulkersonin algoritmin toimintaidea on sinänsä järkevä,
mutta ei ole silti päältä päin todellakaan selvää,
miksi algoritmi löytää varmasti maksimivirtauksen.
Jotta voimme ymmärtää paremmin algoritmin toimintaa,
tarkastelemme seuraavaksi toista verkko-ongelmaa,
joka antaa meille uuden näkökulman maksimivirtaukseen.

Lähtökohtanamme on edelleen suunnattu verkko,
jossa on lähtösolmu ja kohdesolmu.
Sanomme, että joukko kaaria muodostaa \emph{leikkauksen},
jos niiden poistaminen verkosta estää kulkemisen
lähtösolmusta kohdesolmuun.
\emph{Minimileikkaus} on puolestaan leikkaus,
jossa kaarten yhteispaino on mahdollisimman pieni.
Kuvassa \ref{fig:minlei} näkyy esimerkkiverkkomme minimileikkaus,
jossa poistamme kaaret $2 \rightarrow 4$ ja $3 \rightarrow 5$
ja jonka paino on $3+4=7$.

\begin{figure}
\center
\begin{center}
\begin{tikzpicture}[scale=0.8,label distance=-1.5mm]
\node[draw, circle] (1) at (0,-1) {$1$};
\node[draw, circle] (2) at (2,0) {$2$};
\node[draw, circle] (3) at (2,-2) {$3$};
\node[draw, circle] (4) at (4,0) {$4$};
\node[draw, circle] (5) at (4,-2) {$5$};
\path[draw,thick,->] (1) -- node[font=\small,label=above:$4$] {} (2);
\path[draw,thick,->] (1) -- node[font=\small,label=below:$6$] {} (3);
\path[draw,thick,->] (2) -- node[font=\small,label=right:$8$] {} (3);
\path[draw,thick,->] (2) -- node[font=\small,label=above:$3$] {} (4);
\path[draw,thick,->] (3) -- node[font=\small,label=below:$4$] {} (5);
\path[draw,thick,->] (4) -- node[font=\small,label=right:$5$] {} (5);
\path[draw,thick,-,red,line width=2pt] (2.75,-0.25) -- (3.25,0.25);
\path[draw,thick,-,red,line width=2pt] (2.75,0.25) -- (3.25,-0.25);
\path[draw,thick,-,red,line width=2pt] (2.75,-2.25) -- (3.25,-1.75);
\path[draw,thick,-,red,line width=2pt] (2.75,-1.75) -- (3.25,-2.25);
\end{tikzpicture}
\end{center}
\caption{Minimileikkaus, jossa poistamme kaaret $2 \rightarrow 4$ ja $3 \rightarrow 5$.}
\label{fig:minlei}
\end{figure}

Osoittautuu, että verkon maksimivirtaus on aina yhtä suuri kuin
minimileikkaus, ja tämä yhteys auttaa perustelemaan,
miksi Ford-Fulkersonin algoritmi toimii.
Ensinnäkin voimme havaita, että \emph{mikä tahansa}
verkon leikkaus on yhtä suuri tai suurempi
kuin maksimivirtaus.
Tämä johtuu siitä, että virtauksen täytyy ylittää
leikkaukseen kuuluvat kaaret, jotta se pääsee
lähtösolmusta kohdesolmuun.
Esimerkiksi kuvassa \ref{fig:minlei} virtaus voi
päästä solmusta $1$ solmuun $5$
joko kulkemalla kaarta $2 \rightarrow 4$ tai kaarta $3 \rightarrow 5$.
Niinpä virtaus ei voi olla suurempi kuin näiden kaarten
painojen summa.

\begin{figure}
\center
\begin{center}
\begin{tikzpicture}[scale=0.9,label distance=-1.5mm]
\node[draw, circle, fill=lightgray] (1) at (0,-1) {$1$};
\node[draw, circle, fill=lightgray] (2) at (2,0) {$2$};
\node[draw, circle, fill=lightgray] (3) at (2,-2) {$3$};
\node[draw, circle] (4) at (4,0) {$4$};
\node[draw, circle] (5) at (4,-2) {$5$};
\path[draw,thick,->] (1) edge [bend left=20] node[font=\small,label=above:$0$] {} (2);
\path[draw,thick,->] (2) edge [bend left=20] node[font=\small,label=above:$4$] {} (1);
\path[draw,thick,->] (1) edge [bend left=20] node[font=\small,label=below:$3$] {} (3);
\path[draw,thick,->] (3) edge [bend left=20] node[font=\small,label=below:$3$] {} (1);
\path[draw,thick,->] (2) edge [bend left=20] node[font=\small,label=right:$7$] {} (3);
\path[draw,thick,->] (3) edge [bend left=20] node[font=\small,label=right:$1$] {} (2);
\path[draw,thick,->] (2) edge [bend left=20] node[font=\small,label=above:$0$] {} (4);
\path[draw,thick,->] (4) edge [bend left=20] node[font=\small,label=above:$3$] {} (2);
\path[draw,thick,->] (3) edge [bend left=20] node[font=\small,label=below:$0$] {} (5);
\path[draw,thick,->] (5) edge [bend left=20] node[font=\small,label=below:$4$] {} (3);
\path[draw,thick,->] (4) edge [bend left=20] node[font=\small,label=right:$2$] {} (5);
\path[draw,thick,->] (5) edge [bend left=20] node[font=\small,label=right:$3$] {} (4);
\end{tikzpicture}
\end{center}
\caption{Solmut 1, 2 ja 3 ovat saavutettavissa lähtösolmusta.}
\label{fig:flolei}
\end{figure}

Toisaalta Ford-Fulkersonin algoritmi muodostaa sivutuotteenaan
myös verkon leikkauksen, joka on yhtä suuri kuin maksimivirtaus.
Löydämme leikkauksen etsimällä ensin kaikki solmut,
joihin pääsemme lähtösolmusta positiivisia kaaria pitkin
algoritmin lopputilanteessa.
Kuva \ref{fig:flolei} näyttää nämä solmut esimerkkiverkossamme:
solmut ovat 1, 2 ja 3.
Kun valitsemme sitten alkuperäisen verkon kaaret,
jotka johtavat näiden solmujen ulkopuolelle
ja joiden kapasiteetti on käytetty kokonaan,
saamme aikaan verkon leikkauksen.
Esimerkissämme nämä kaaret ovat $2 \rightarrow 4$ ja $3 \rightarrow 5$.

Koska olemme löytäneet virtauksen, joka on yhtä suuri kuin leikkaus,
ja toisaalta virtaus ei voi olla mitään leikkausta suurempi,
olemme siis löytäneet maksimivirtauksen ja minimileikkauksen,
joten Ford-Fulkersonin algoritmi toimii oikein.

\subsection{Polkujen valitseminen}

Voimme muodostaa Ford-Fulkersonin algoritmin aikana miten tahansa,
mutta polkujen valintatapa vaikuttaa algoritmin tehokkuuteen.
Pahimmassa tapauksessa jokainen polku tuottaa vain yhden yksikön
lisää virtausta, jolloin algoritmi joutuu muodostamaan kaikkiaan $f$ polkua,
missä $f$ on verkon maksimivirtaus.
Jos muodostamme polut syvyyshaulla, saamme siis algoritmin
ajankäytölle ylärajan $O(f (n+m))$.

Edmonds-Karpin algoritmi on Ford-Fulkersonin algoritmin versio,
jossa muodostammekin polut \emph{leveyshaulla}.
Tämä tarkoittaa, että valitsemme aina
polun, jossa on mahdollisimman vähän kaaria.
On mahdollista osoittaa, että tällaisen valinnan ansiosta
algoritmi muodostaa aina enintään $nm$ polkua ja
sen aikavaativuus on $O(nm (n+m))$.

\section{Sovelluksia}

Virtauslaskennan merkitys on siinä, että voimme \emph{palauttaa}
monia ongelmia maksimivirtauksen laskemiseen.
Tämä tarkoittaa, että muunnamme ongelman jotenkin
sellaiseen muotoon, että se vastaa maksimivirtauksen etsimistä.
Tutustumme seuraavaksi joihinkin tällaisiin ongelmiin.

\subsection{Erilliset polut}

Ensimmäinen tehtävämme on muodostaa mahdollisimman monta
\emph{erillistä} polkua verkon lähtösolmusta kohdesolmuun.
Tämä tarkoittaa, että jokainen verkon kaari saa esiintyä
enintään yhdellä polulla.
Saamme kuitenkin halutessamme kulkea saman solmun kautta useita kertoja.
Esimerkiksi kuvassa \ref{fig:eripol} voimme muodostaa
kaksi erillistä polkua solmusta $1$ solmuun $5$,
mutta ei ole mahdollista muodostaa kolmea erillistä polkua.

\begin{figure}
\center
\begin{center}
\begin{tikzpicture}[scale=0.8,label distance=-1.5mm]
\newcommand\verkko[0]{
\node[draw, circle] (1) at (0,-1) {$1$};
\node[draw, circle] (2) at (2,0) {$2$};
\node[draw, circle] (3) at (2,-2) {$3$};
\node[draw, circle] (4) at (4,0) {$4$};
\node[draw, circle] (5) at (4,-2) {$5$};
\path[draw,thick,->] (1) -- (2);
\path[draw,thick,->] (1) -- (3);
\path[draw,thick,->] (2) -- (3);
\path[draw,thick,->] (4) -- (2);
\path[draw,thick,->] (3) -- (5);
\path[draw,thick,->] (4) -- (5);
\path[draw,thick,->] (3) -- (4);
}
\begin{scope}
\verkko
\path[draw,thick,->,red,line width=2pt] (1) -- (2);
\path[draw,thick,->,red,line width=2pt] (2) -- (3);
\path[draw,thick,->,red,line width=2pt] (3) -- (4);
\path[draw,thick,->,red,line width=2pt] (4) -- (5);
\end{scope}
\begin{scope}[xshift=7cm]
\verkko
\path[draw,thick,->,red,line width=2pt] (1) -- (3);
\path[draw,thick,->,red,line width=2pt] (3) -- (5);
\end{scope}
\end{tikzpicture}
\end{center}
\caption{Kaksi erillistä polkua solmusta $1$ solmuun $5$.}
\label{fig:eripol}
\end{figure}

\begin{figure}
\center
\begin{center}
\begin{tikzpicture}[scale=0.8,label distance=-1.5mm]
\node[draw, circle] (1) at (0,-1) {$1$};
\node[draw, circle] (2) at (2,0) {$2$};
\node[draw, circle] (3) at (2,-2) {$3$};
\node[draw, circle] (4) at (4,0) {$4$};
\node[draw, circle] (5) at (4,-2) {$5$};
\path[draw,thick,->] (1) -- node[font=\small,label=above:$1/1$] {} (2);
\path[draw,thick,->] (1) -- node[font=\small,label=below:$1/1$] {} (3);
\path[draw,thick,->] (2) -- node[font=\small,label=left:$1/1$] {} (3);
\path[draw,thick,->] (4) -- node[font=\small,label=above:$0/1$] {} (2);
\path[draw,thick,->] (3) -- node[font=\small,label=below:$1/1$] {} (5);
\path[draw,thick,->] (4) -- node[font=\small,label=right:$1/1$] {} (5);
\path[draw,thick,->] (3) -- node[font=\small,label=above:$1/1$] {} (4);
\end{tikzpicture}
\end{center}
\caption{Erilliset polut tulkittuna maksimivirtauksena.}
\label{fig:erivir}
\end{figure}

Voimme ratkaista erillisten polkujen ongelman
tulkitsemalla tehtävän maksimivirtauksena.
Ideana on etsiä maksimivirtaus lähtösolmusta kohdesolmuun
niin, että jokaisen kaaren kapasiteetti on $1$.
Tämä maksimivirtaus on yhtä suuri kuin suurin
erillisten polkujen määrä.
Kuva \ref{fig:erivir} näyttää maksimivirtauksen
esimerkkiverkossamme.

Miksi maksimivirtaus ja erillisten polkujen määrä ovat yhtä suuret?
Ensinnäkin erilliset polut muodostavat yhdessä virtauksen,
joten maksimivirtaus ei voi olla pienempi kuin erillisten polkujen määrä.
Toisaalta jos virtaus on $k$, voimme muodostaa $k$ erillistä polkua
valitsemalla kaaria ahneesti lähtösolmusta alkaen,
joten maksimivirtaus ei voi olla suurempi kuin erillisten polkujen määrä.
Ainoa mahdollisuus on, että maksimivirtaus ja erillisten polkujen määrä
ovat yhtä suuret.

\subsection{Maksimiparitus}

Verkon \emph{paritus} on joukko kaaria, joille pätee,
että jokainen solmu on enintään yhden kaaren päätepisteenä.
\emph{Maksimiparitus} on puolestaan paritus,
jossa on mahdollisimman paljon kaaria.
Keskitymme tapaukseen,
jossa verkko on \emph{kaksijakoinen} eli
voimme jakaa verkon solmut
vasempaan ja oikeaan ryhmään niin, että jokainen
kaari kulkee ryhmien välillä.

Kuvassa \ref{fig:makpar} on esimerkkinä kaksijakoinen verkko,
jonka maksimiparitus on $3$.
Tässä vasen ryhmä on $\{1,2,3,4\}$, oikea ryhmä on $\{5,6,7\}$
ja maksimiparitus muodostuu kaarista
$(1,6)$, $(3,5)$ ja $(4,7)$.

\begin{figure}
\center
\begin{center}
\begin{tikzpicture}[scale=0.8,label distance=-1.5mm]
\node[draw, circle] (1) at (0,0) {$1$};
\node[draw, circle] (2) at (0,-1.25) {$2$};
\node[draw, circle] (3) at (0,-2.5) {$3$};
\node[draw, circle] (4) at (0,-3.75) {$4$};
\node[draw, circle] (5) at (4,-0.625) {$5$};
\node[draw, circle] (6) at (4,-1.875) {$6$};
\node[draw, circle] (7) at (4,-3.125) {$7$};
\path[draw,thick,-] (1) -- (6);
\path[draw,thick,-] (2) -- (6);
\path[draw,thick,-] (4) -- (6);
\path[draw,thick,-] (3) -- (5);
\path[draw,thick,-] (3) -- (7);
\path[draw,thick,-] (4) -- (7);
\path[draw,thick,-,red,line width=2pt] (1) -- (6);
\path[draw,thick,-,red,line width=2pt] (3) -- (5);
\path[draw,thick,-,red,line width=2pt] (4) -- (7);
\end{tikzpicture}
\end{center}
\caption{Kaksijakoisen verkon maksimiparitus.}
\label{fig:makpar}
\end{figure}

\begin{figure}
\center
\begin{center}
\begin{tikzpicture}[scale=0.8,label distance=-1.5mm]
\node[draw, circle] (1) at (0,0) {$1$};
\node[draw, circle] (2) at (0,-1.25) {$2$};
\node[draw, circle] (3) at (0,-2.5) {$3$};
\node[draw, circle] (4) at (0,-3.75) {$4$};
\node[draw, circle] (5) at (4,-0.625) {$5$};
\node[draw, circle] (6) at (4,-1.875) {$6$};
\node[draw, circle] (7) at (4,-3.125) {$7$};
\node[draw, circle] (a) at (-3,-1.875) {\phantom{$a$}};
\node[draw, circle] (b) at (7,-1.875) {\phantom{$b$}};
\path[draw,thick,->] (1) -- (6);
\path[draw,thick,->] (2) -- (6);
\path[draw,thick,->] (4) -- (6);
\path[draw,thick,->] (3) -- (5);
\path[draw,thick,->] (3) -- (7);
\path[draw,thick,->] (4) -- (7);
\path[draw,thick,->] (a) -- (1);
\path[draw,thick,->] (a) -- (2);
\path[draw,thick,->] (a) -- (3);
\path[draw,thick,->] (a) -- (4);
\path[draw,thick,->] (5) -- (b);
\path[draw,thick,->] (6) -- (b);
\path[draw,thick,->] (7) -- (b);
\end{tikzpicture}
\end{center}
\caption{Maksimiparitus tulkittuna maksimivirtauksena.}
\label{fig:parver}
\end{figure}

Voimme tulkita maksimiparituksen maksimivirtauksena
lisäämällä verkkoon kaksi uutta solmua: lähtösolmun ja kohdesolmun.
Lähtösolmusta pääsee kaarella jokaiseen vasemman ryhmän solmuun,
ja jokaisesta oikean ryhmän solmusta pääsee kaarella kohdesolmuun.
Lisäksi suuntaamme alkuperäiset kaaret niin,
että ne kulkevat vasemmasta ryhmästä oikeaan ryhmään.
Kuva \ref{fig:parver} näyttää tuloksena olevan verkon
esimerkissämme.
Maksimivirtaus tässä verkossa vastaa alkuperäisen verkon
maksimiparitusta.

\subsection{Polkupeite}

\emph{Polkupeite} on joukko verkon polkuja,
jotka kattavat yhdessä kaikki verkon solmut.
Oletamme seuraavaksi, että verkko on suunnattu ja syklitön,
ja haluamme muodostaa mahdollisimman pienen polkupeitteen
niin, että jokainen solmu esiintyy tarkalleen yhdessä polussa.
Kuvassa \ref{fig:polpei} on esimerkkinä verkko ja sen pienin polkupeite,
joka muodostuu kahdesta polusta.

\begin{figure}
\center
\begin{center}
\begin{tikzpicture}[scale=0.8,label distance=-1.5mm]
\node[draw, circle] (1) at (0,0) {$1$};
\node[draw, circle] (2) at (2,0) {$2$};
\node[draw, circle] (3) at (4,0) {$3$};
\node[draw, circle] (4) at (6,0) {$4$};
\node[draw, circle] (5) at (2,-2) {$5$};
\path[draw,thick,->] (1) -- (2);
\path[draw,thick,->] (2) -- (3);
\path[draw,thick,->] (3) -- (4);
\path[draw,thick,->] (5) -- (2);
\path[draw,thick,->] (5) -- (3);
\path[draw,thick,->,red,line width=2pt] (1) -- (2);
\path[draw,thick,->,red,line width=2pt] (5) -- (3);
\path[draw,thick,->,red,line width=2pt] (3) -- (4);
\end{tikzpicture}
\end{center}
\caption{Polkupeite, joka muodostuu poluista $1 \rightarrow 2$ ja $5 \rightarrow 3 \rightarrow 4$.}
\label{fig:polpei}
\end{figure}

Voimme ratkaista pienimmän polkupeitteen etsimisen
ongelman maksimivirtauksen avulla muodostamalla verkon,
jossa jokaista alkuperäistä solmua vastaa kaksi solmua:
vasen ja oikea solmu.
Vasemmasta solmusta on kaari oikeaan solmuun,
jos alkuperäisessä verkossa on vastaava kaari.
Lisäämme vielä verkkoon lähtösolmun ja kohdesolmun niin,
että lähtösolmusta pääsee kaikkiin vasempiin solmuihin
ja kaikista oikeista solmuista pääsee kohdesolmuihin.
Tämän verkon maksimivirtaus antaa meille alkuperäisen
verkon pienimmän solmupeitteen.

\begin{figure}
\center
\begin{center}
\begin{tikzpicture}[scale=0.8,label distance=-1.5mm]
\node[draw, circle] (1a) at (0,0) {$1$};
\node[draw, circle] (2a) at (0,-1.25) {$2$};
\node[draw, circle] (3a) at (0,-2.5) {$3$};
\node[draw, circle] (4a) at (0,-3.75) {$4$};
\node[draw, circle] (5a) at (0,-5) {$5$};
\node[draw, circle] (1b) at (4,0) {$1$};
\node[draw, circle] (2b) at (4,-1.25) {$2$};
\node[draw, circle] (3b) at (4,-2.5) {$3$};
\node[draw, circle] (4b) at (4,-3.75) {$4$};
\node[draw, circle] (5b) at (4,-5) {$5$};
\node[draw, circle] (a) at (-3,-2.5) {\phantom{$a$}};
\node[draw, circle] (b) at (7,-2.5) {\phantom{$b$}};
\path[draw,thick,->] (1a) -- (2b);
\path[draw,thick,->] (2a) -- (3b);
\path[draw,thick,->] (3a) -- (4b);
\path[draw,thick,->] (5a) -- (2b);
\path[draw,thick,->] (5a) -- (3b);
\path[draw,thick,->] (a) -- (1a);
\path[draw,thick,->] (a) -- (2a);
\path[draw,thick,->] (a) -- (3a);
\path[draw,thick,->] (a) -- (4a);
\path[draw,thick,->] (a) -- (5a);
\path[draw,thick,->] (1b) -- (b);
\path[draw,thick,->] (2b) -- (b);
\path[draw,thick,->] (3b) -- (b);
\path[draw,thick,->] (4b) -- (b);
\path[draw,thick,->] (5b) -- (b);
\end{tikzpicture}
\end{center}
\caption{Polkupeitteen etsiminen maksimivirtauksen avulla.}
\label{fig:polvir}
\end{figure}

Kuva \ref{fig:polvir} näyttää tuloksena olevan verkon
esimerkissämme.
Ideana on, että maksimivirtaus etsii,
mitkä kaaret kuuluvat polkuihin:
jos kaari solmusta $a$ solmuun $b$ kuuluu virtaukseen,
niin vastaavasti polkupeitteessä on polku,
jossa on kaari $a \rightarrow b$.
Koska virtauksessa voi olla valittuna vain yksi kaari,
joka alkaa tietystä solmusta tai päättyy tiettyyn solmuun,
tuloksena on varmasti joukko polkuja.
Toisaalta mitä enemmän kaaria saamme laitettua polkuihin,
sitä pienempi on polkujen määrä.