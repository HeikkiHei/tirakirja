\chapter{Maksimivirtaus}

\index{virtaus}
\index{maksimivirtaus}

Tässä luvussa tarkastelemme ongelmaa,
jossa haluamme välittää mahdollisimman paljon
\emph{virtausta} (\emph{flow}) verkon solmusta toiseen,
kun jokaisella kaarella on tietty kapasiteetti,
jota emme saa ylittää.
Voimme esimerkiksi haluta siirtää tietokoneverkossa tietoa
mahdollisimman tehokkaasti koneesta toiseen,
kun tiedämme verkon rakenteen ja yhteyksien nopeudet.

Tutustumme aluksi Ford–Fulkersonin algoritmiin,
jonka avulla voimme sekä selvittää maksimivirtauksen
että ymmärtää paremmin, mistä ongelmassa on kysymys.
Tämän jälkeen tarkastelemme joitakin verkko-ongelmia,
jotka pystymme ratkaisemaan \emph{palauttamalla}
ne maksimivirtaukseen.

\section{Maksimivirtauksen laskeminen}

Käsittelemme suunnattua verkkoa,
jossa on kaksi erityistä solmua:
\emph{lähtösolmu} (\emph{source}) ja \emph{kohdesolmu} (\emph{sink}).
Haluamme muodostaa verkkoon mahdollisimman
suuren virtauksen eli \emph{maksimivirtauksen} (\emph{maximum flow})
lähtö\-solmusta kohdesolmuun niin, että
jokaiseen välisolmuun tuleva virtaus on
yhtä suuri kuin solmusta lähtevä virtaus.
Kullakin verkon kaarella on
\emph{kapasiteetti} (\emph{capacity}), jota virtauksen
määrä kaarta pitkin ei saa ylittää.

\begin{figure}
\center
\begin{center}
\begin{tikzpicture}[scale=0.8,label distance=-1.5mm]
\node[draw, circle] (1) at (0,-1) {$1$};
\node[draw, circle] (2) at (2,0) {$2$};
\node[draw, circle] (3) at (2,-2) {$3$};
\node[draw, circle] (4) at (4,0) {$4$};
\node[draw, circle] (5) at (4,-2) {$5$};
\path[draw,thick,->] (1) -- node[font=\small,label=above:$4/4$] {} (2);
\path[draw,thick,->] (1) -- node[font=\small,label=below:$3/6$] {} (3);
\path[draw,thick,->] (2) -- node[font=\small,label=right:$1/8$] {} (3);
\path[draw,thick,->] (2) -- node[font=\small,label=above:$3/3$] {} (4);
\path[draw,thick,->] (3) -- node[font=\small,label=below:$4/4$] {} (5);
\path[draw,thick,->] (4) -- node[font=\small,label=right:$3/5$] {} (5);
\end{tikzpicture}
\end{center}
\caption{Maksimivirtaus solmusta $1$ solmuun $5$ on $7$.}
\label{fig:makvir}
\end{figure}

Kuvassa \ref{fig:makvir} näkyy esimerkkinä verkon maksimivirtaus,
kun lähtösolmu on $1$ ja kohdesolmu $5$.
Tässä tapauksessa maksimivirtauksen suuruus on $7$.
Jokaisessa kaaressa merkintä $v/k$ tarkoittaa,
että kaaren kautta kulkee virtausta $v$ ja
kaaren kapasiteetti on $k$.
Solmusta $1$ lähtevä virtauksen määrä on $4+3=7$,
solmuun $5$ saapuva virtauksen määrä on $3+4=7$
ja kaikissa muissa solmuissa saapuva virtaus
on yhtä suuri kuin lähtevä virtaus.

\subsection{Ford–Fulkersonin algoritmi}

\index{Ford–Fulkersonin algoritmi}

\emph{Ford–Fulkersonin algoritmi} on tavallisin menetelmä verkon
maksimivirtauksen etsimiseen,
ja tutustumme seuraavaksi tämän algoritmin toimintaan.
Algoritmi muodostaa lähtösolmusta kohdesolmuun polkuja,
jotka kasvattavat virtausta pikkuhiljaa.
Kun mitään polkua ei voi enää muodostaa, algoritmi on
saanut valmiiksi maksimivirtauksen.

\begin{figure}
\center
\begin{center}
\begin{tikzpicture}[scale=0.9,label distance=-1.5mm]
\node[draw, circle] (1) at (0,-1) {$1$};
\node[draw, circle] (2) at (2,0) {$2$};
\node[draw, circle] (3) at (2,-2) {$3$};
\node[draw, circle] (4) at (4,0) {$4$};
\node[draw, circle] (5) at (4,-2) {$5$};
\path[draw,thick,->] (1) edge [bend left=20] node[font=\small,label=above:$4$] {} (2);
\path[draw,thick,->] (2) edge [bend left=20] node[font=\small,label=above:$0$] {} (1);
\path[draw,thick,->] (1) edge [bend left=20] node[font=\small,label=below:$6$] {} (3);
\path[draw,thick,->] (3) edge [bend left=20] node[font=\small,label=below:$0$] {} (1);
\path[draw,thick,->] (2) edge [bend left=20] node[font=\small,label=right:$8$] {} (3);
\path[draw,thick,->] (3) edge [bend left=20] node[font=\small,label=right:$0$] {} (2);
\path[draw,thick,->] (2) edge [bend left=20] node[font=\small,label=above:$3$] {} (4);
\path[draw,thick,->] (4) edge [bend left=20] node[font=\small,label=above:$0$] {} (2);
\path[draw,thick,->] (3) edge [bend left=20] node[font=\small,label=below:$4$] {} (5);
\path[draw,thick,->] (5) edge [bend left=20] node[font=\small,label=below:$0$] {} (3);
\path[draw,thick,->] (4) edge [bend left=20] node[font=\small,label=right:$5$] {} (5);
\path[draw,thick,->] (5) edge [bend left=20] node[font=\small,label=right:$0$] {} (4);
\end{tikzpicture}
\end{center}
\caption{Verkon esitysmuoto Ford–Fulkersonin algoritmissa.}
\label{fig:flouus}
\end{figure}

Jotta voimme käyttää algoritmia, esitämme verkon
erityisessä muodossa, jossa jokaista alkuperäisen verkon
kaarta vastaa \emph{kaksi} kaarta:
alkuperäinen kaari, jonka painona on kaaren kapasiteetti,
sekä sille kään\-teinen kaari, jonka painona on $0$.
Käänteisten kaarten avulla pystymme tarvittaessa \emph{peruuttamaan}
virtausta algoritmin aikana.
Kuva \ref{fig:flouus} näyttää, kuinka esitämme esimerkkiverkkomme algoritmissa.

Algoritmi muodostaa joka vaiheessa \emph{täydennyspolun}
(\emph{augmenting path}),
joka on mikä tahansa polku lähtösolmusta kohdesolmuun,
jossa jokaisen kaaren paino on positiivinen.
Polun muodostamisen jälkeen virtaus lähtösolmusta kohdesolmuun
kasvaa $p$:llä, missä $p$ on pienin kaaren paino polulla.
Lisäksi jokaisen polulla olevan kaaren paino vähenee $p$:llä
ja jokaisen niille käänteisen kaaren paino kasvaa $p$:llä.
Etsimme tällä tavalla uusia polkuja, kunnes mitään
täydennyspolkua ei voi enää muodostaa.

Kuva \ref{fig:floesi} näyttää, kuinka Ford–Fulkersonin algoritmi muodostaa
maksimivirtauksen esimerkkiverkossamme.
Algoritmi muodostaa ensin polun $1 \rightarrow 2 \rightarrow 3 \rightarrow 5$,
jossa pienin paino on $4$.
Tämän seurauksena virtaus kasvaa $4$:llä,
polulla olevien kaarten paino vähenee $4$:llä
ja käänteisten kaarten paino kasvaa $4$:llä.
Tämän jälkeen algoritmi muodostaa polun
$1 \rightarrow 3 \rightarrow 2 \rightarrow 4 \rightarrow 5$,
joka kasvattaa virtausta $3$:lla.
Huomaa, että tämä polku peruuttaa
kaarta $2 \rightarrow 3$ menevää virtausta,
koska se kulkee käänteisen kaaren $3 \rightarrow 2$ kautta.
Tämän jälkeen algoritmi ei enää pysty muodostamaan mitään
täydennyspolkua, joten maksimivirtaus on $4+3=7$.

\begin{figure}
\center
\begin{center}
\begin{tikzpicture}[scale=0.9,label distance=-1.5mm]
\newcommand\verkko[0]{
\node[draw, circle] (1) at (0,-1) {$1$};
\node[draw, circle] (2) at (2,0) {$2$};
\node[draw, circle] (3) at (2,-2) {$3$};
\node[draw, circle] (4) at (4,0) {$4$};
\node[draw, circle] (5) at (4,-2) {$5$};
}
\begin{scope}
\verkko
\path[draw,thick,->] (1) edge [bend left=20] node[font=\small,label=above:$4$] {} (2);
\path[draw,thick,->] (2) edge [bend left=20] node[font=\small,label=above:$0$] {} (1);
\path[draw,thick,->] (1) edge [bend left=20] node[font=\small,label=below:$6$] {} (3);
\path[draw,thick,->] (3) edge [bend left=20] node[font=\small,label=below:$0$] {} (1);
\path[draw,thick,->] (2) edge [bend left=20] node[font=\small,label=right:$8$] {} (3);
\path[draw,thick,->] (3) edge [bend left=20] node[font=\small,label=right:$0$] {} (2);
\path[draw,thick,->] (2) edge [bend left=20] node[font=\small,label=above:$3$] {} (4);
\path[draw,thick,->] (4) edge [bend left=20] node[font=\small,label=above:$0$] {} (2);
\path[draw,thick,->] (3) edge [bend left=20] node[font=\small,label=below:$4$] {} (5);
\path[draw,thick,->] (5) edge [bend left=20] node[font=\small,label=below:$0$] {} (3);
\path[draw,thick,->] (4) edge [bend left=20] node[font=\small,label=right:$5$] {} (5);
\path[draw,thick,->] (5) edge [bend left=20] node[font=\small,label=right:$0$] {} (4);
\path[draw,thick,->,red,line width=2pt] (1) edge [bend left=20] (2);
\path[draw,thick,->,red,line width=2pt] (2) edge [bend left=20] (3);
\path[draw,thick,->,red,line width=2pt] (3) edge [bend left=20] (5);
\draw[->,thick] (5.5,-1) -- (6.5,-1);
\node at (6,-3) {vaihe 1};
\end{scope}
\begin{scope}[xshift=7.5cm]
\verkko
\path[draw,thick,->] (1) edge [bend left=20] node[font=\small,label=above:$0$] {} (2);
\path[draw,thick,->] (2) edge [bend left=20] node[font=\small,label=above:$4$] {} (1);
\path[draw,thick,->] (1) edge [bend left=20] node[font=\small,label=below:$6$] {} (3);
\path[draw,thick,->] (3) edge [bend left=20] node[font=\small,label=below:$0$] {} (1);
\path[draw,thick,->] (2) edge [bend left=20] node[font=\small,label=right:$4$] {} (3);
\path[draw,thick,->] (3) edge [bend left=20] node[font=\small,label=right:$4$] {} (2);
\path[draw,thick,->] (2) edge [bend left=20] node[font=\small,label=above:$3$] {} (4);
\path[draw,thick,->] (4) edge [bend left=20] node[font=\small,label=above:$0$] {} (2);
\path[draw,thick,->] (3) edge [bend left=20] node[font=\small,label=below:$0$] {} (5);
\path[draw,thick,->] (5) edge [bend left=20] node[font=\small,label=below:$4$] {} (3);
\path[draw,thick,->] (4) edge [bend left=20] node[font=\small,label=right:$5$] {} (5);
\path[draw,thick,->] (5) edge [bend left=20] node[font=\small,label=right:$0$] {} (4);
\end{scope}
\begin{scope}[yshift=-5cm]
\verkko
\path[draw,thick,->] (1) edge [bend left=20] node[font=\small,label=above:$0$] {} (2);
\path[draw,thick,->] (2) edge [bend left=20] node[font=\small,label=above:$4$] {} (1);
\path[draw,thick,->] (1) edge [bend left=20] node[font=\small,label=below:$6$] {} (3);
\path[draw,thick,->] (3) edge [bend left=20] node[font=\small,label=below:$0$] {} (1);
\path[draw,thick,->] (2) edge [bend left=20] node[font=\small,label=right:$4$] {} (3);
\path[draw,thick,->] (3) edge [bend left=20] node[font=\small,label=right:$4$] {} (2);
\path[draw,thick,->] (2) edge [bend left=20] node[font=\small,label=above:$3$] {} (4);
\path[draw,thick,->] (4) edge [bend left=20] node[font=\small,label=above:$0$] {} (2);
\path[draw,thick,->] (3) edge [bend left=20] node[font=\small,label=below:$0$] {} (5);
\path[draw,thick,->] (5) edge [bend left=20] node[font=\small,label=below:$4$] {} (3);
\path[draw,thick,->] (4) edge [bend left=20] node[font=\small,label=right:$5$] {} (5);
\path[draw,thick,->] (5) edge [bend left=20] node[font=\small,label=right:$0$] {} (4);
\path[draw,thick,->,red,line width=2pt] (1) edge [bend left=20] (3);
\path[draw,thick,->,red,line width=2pt] (3) edge [bend left=20] (2);
\path[draw,thick,->,red,line width=2pt] (2) edge [bend left=20] (4);
\path[draw,thick,->,red,line width=2pt] (4) edge [bend left=20] (5);
\draw[->,thick] (5.5,-1) -- (6.5,-1);
\node at (6,-3) {vaihe 2};
\end{scope}
\begin{scope}[yshift=-5cm,xshift=7.5cm]
\verkko
\path[draw,thick,->] (1) edge [bend left=20] node[font=\small,label=above:$0$] {} (2);
\path[draw,thick,->] (2) edge [bend left=20] node[font=\small,label=above:$4$] {} (1);
\path[draw,thick,->] (1) edge [bend left=20] node[font=\small,label=below:$3$] {} (3);
\path[draw,thick,->] (3) edge [bend left=20] node[font=\small,label=below:$3$] {} (1);
\path[draw,thick,->] (2) edge [bend left=20] node[font=\small,label=right:$7$] {} (3);
\path[draw,thick,->] (3) edge [bend left=20] node[font=\small,label=right:$1$] {} (2);
\path[draw,thick,->] (2) edge [bend left=20] node[font=\small,label=above:$0$] {} (4);
\path[draw,thick,->] (4) edge [bend left=20] node[font=\small,label=above:$3$] {} (2);
\path[draw,thick,->] (3) edge [bend left=20] node[font=\small,label=below:$0$] {} (5);
\path[draw,thick,->] (5) edge [bend left=20] node[font=\small,label=below:$4$] {} (3);
\path[draw,thick,->] (4) edge [bend left=20] node[font=\small,label=right:$2$] {} (5);
\path[draw,thick,->] (5) edge [bend left=20] node[font=\small,label=right:$3$] {} (4);
\end{scope}
\end{tikzpicture}
\end{center}
\caption{Esimerkki Ford–Fulkersonin algoritmin toiminnasta.}
\label{fig:floesi}
\end{figure}

Kun olemme saaneet maksimivirtauksen muodostettua,
voimme selvittää jokaisessa alkuperäisessä kaaressa kulkevan
virtauksen tutkimalla, miten kaaren paino on muuttunut algoritmin aikana.
Kaarta pitkin kulkeva virtaus on yhtä suuri kuin kaaren painon
vähennys algoritmin aikana.
Esimerkiksi kuvassa \ref{fig:floesi} kaaren $4 \rightarrow 5$
paino on alussa 5 ja algoritmin suorituksen jälkeen 2,
joten kaarta pitkin kulkevan virtauksen määrä on $5-2=3$.

\subsection{Yhteys minimileikkaukseen}

\index{leikkaus}
\index{minimileikkaus}

Ford–Fulkersonin algoritmin toimintaidea on sinänsä järkevä,
koska täyden\-nyspolut lähtösolmusta kohdesolmuun kasvattavat virtausta,
mutta ei ole silti todellakaan päältä päin selvää,
miksi algoritmi löytää varmasti \emph{suurimman} mahdollisen virtauksen.
Jotta voimme ymmärtää paremmin algoritmin toimintaa,
tarkastelemme seuraavaksi toista verkko-ongelmaa,
joka antaa meille uuden näkökulman maksimivirtaukseen.

Lähtökohtanamme on edelleen suunnattu verkko,
jossa on lähtösolmu ja kohdesolmu.
Sanomme, että joukko kaaria muodostaa \emph{leikkauksen} (\emph{cut}),
jos niiden poistaminen verkosta estää kulkemisen
lähtösolmusta kohdesolmuun.
\emph{Minimileikkaus} (\emph{minimum cut}) on puolestaan leikkaus,
jossa kaarten yhteispaino on mahdollisimman pieni.
Kuvassa \ref{fig:minlei} on minimileikkaus,
jossa poistamme kaaret $2 \rightarrow 4$ ja $3 \rightarrow 5$
ja jonka paino on $3+4=7$.

\begin{figure}
\center
\begin{center}
\begin{tikzpicture}[scale=0.8,label distance=-1.5mm]
\node[draw, circle] (1) at (0,-1) {$1$};
\node[draw, circle] (2) at (2,0) {$2$};
\node[draw, circle] (3) at (2,-2) {$3$};
\node[draw, circle] (4) at (4,0) {$4$};
\node[draw, circle] (5) at (4,-2) {$5$};
\path[draw,thick,->] (1) -- node[font=\small,label=above:$4$] {} (2);
\path[draw,thick,->] (1) -- node[font=\small,label=below:$6$] {} (3);
\path[draw,thick,->] (2) -- node[font=\small,label=right:$8$] {} (3);
\path[draw,thick,->] (2) -- node[font=\small,label=above:$3$] {} (4);
\path[draw,thick,->] (3) -- node[font=\small,label=below:$4$] {} (5);
\path[draw,thick,->] (4) -- node[font=\small,label=right:$5$] {} (5);
\path[draw,thick,-,red,line width=2pt] (2.75,-0.25) -- (3.25,0.25);
\path[draw,thick,-,red,line width=2pt] (2.75,0.25) -- (3.25,-0.25);
\path[draw,thick,-,red,line width=2pt] (2.75,-2.25) -- (3.25,-1.75);
\path[draw,thick,-,red,line width=2pt] (2.75,-1.75) -- (3.25,-2.25);
\end{tikzpicture}
\end{center}
\caption{Minimileikkaus, jossa poistamme kaaret $2 \rightarrow 4$ ja $3 \rightarrow 5$.}
\label{fig:minlei}
\end{figure}

Osoittautuu, että verkon maksimivirtaus on aina yhtä suuri kuin
minimileikkaus, ja tämä yhteys auttaa perustelemaan,
miksi Ford–Fulkersonin algoritmi toimii.
Ensinnäkin voimme havaita, että \emph{mikä tahansa}
verkon leikkaus on yhtä suuri tai suurempi
kuin maksimivirtaus.
Tämä johtuu siitä, että virtauksen täytyy ylittää
jokin leikkaukseen kuuluva kaari, jotta se pääsee
lähtösolmusta kohdesolmuun.
Esimerkiksi kuvassa \ref{fig:minlei} virtaus voi
päästä solmusta $1$ solmuun $5$
joko kulkemalla kaarta $2 \rightarrow 4$ tai kaarta $3 \rightarrow 5$.
Niinpä virtaus ei voi olla suurempi kuin näiden kaarten
painojen summa.

\begin{figure}
\center
\begin{center}
\begin{tikzpicture}[scale=0.9,label distance=-1.5mm]
\node[draw, circle, fill=lightgray] (1) at (0,-1) {$1$};
\node[draw, circle, fill=lightgray] (2) at (2,0) {$2$};
\node[draw, circle, fill=lightgray] (3) at (2,-2) {$3$};
\node[draw, circle] (4) at (4,0) {$4$};
\node[draw, circle] (5) at (4,-2) {$5$};
\path[draw,thick,->] (1) edge [bend left=20] node[font=\small,label=above:$0$] {} (2);
\path[draw,thick,->] (2) edge [bend left=20] node[font=\small,label=above:$4$] {} (1);
\path[draw,thick,->] (1) edge [bend left=20] node[font=\small,label=below:$3$] {} (3);
\path[draw,thick,->] (3) edge [bend left=20] node[font=\small,label=below:$3$] {} (1);
\path[draw,thick,->] (2) edge [bend left=20] node[font=\small,label=right:$7$] {} (3);
\path[draw,thick,->] (3) edge [bend left=20] node[font=\small,label=right:$1$] {} (2);
\path[draw,thick,->] (2) edge [bend left=20] node[font=\small,label=above:$0$] {} (4);
\path[draw,thick,->] (4) edge [bend left=20] node[font=\small,label=above:$3$] {} (2);
\path[draw,thick,->] (3) edge [bend left=20] node[font=\small,label=below:$0$] {} (5);
\path[draw,thick,->] (5) edge [bend left=20] node[font=\small,label=below:$4$] {} (3);
\path[draw,thick,->] (4) edge [bend left=20] node[font=\small,label=right:$2$] {} (5);
\path[draw,thick,->] (5) edge [bend left=20] node[font=\small,label=right:$3$] {} (4);
\end{tikzpicture}
\end{center}
\caption{Solmut 1, 2 ja 3 ovat saavutettavissa lähtösolmusta.}
\label{fig:flolei}
\end{figure}

Toisaalta Ford–Fulkersonin algoritmi muodostaa sivutuotteenaan
myös verkon leikkauksen, joka on yhtä suuri kuin maksimivirtaus.
Löydämme leikkauksen etsimällä ensin kaikki solmut,
joihin pääsemme lähtösolmusta positiivisia kaaria pitkin
algoritmin lopputilanteessa.
Esimerkkiverkossamme nämä solmut ovat 1, 2 ja 3
kuvan \ref{fig:flolei} mukaisesti.
Kun valitsemme sitten alkuperäisen verkon kaaret,
jotka johtavat näiden solmujen ulkopuolelle
ja joiden kapasiteetti on käytetty kokonaan,
saamme aikaan verkon leikkauksen.
Esimerkissämme nämä kaaret ovat $2 \rightarrow 4$ ja $3 \rightarrow 5$.

Koska olemme löytäneet virtauksen, joka on yhtä suuri kuin leikkaus,
ja toisaalta virtaus ei voi olla mitään leikkausta suurempi,
olemme siis löytäneet maksimivirtauksen ja minimileikkauksen,
joten Ford–Fulkersonin algoritmi toimii oikein.

\subsection{Polkujen valitseminen}

Voimme muodostaa täydennyspolkuja Ford–Fulkersonin algoritmin aikana miten tahansa,
mutta polkujen valintatapa vaikuttaa algoritmin tehokkuuteen.
Riippumatta polkujen valintatavasta on selvää,
että jokainen polku kasvattaa virtausta ainakin \emph{yhdellä} yksiköllä.
Niinpä tiedämme, että joudumme etsimään enintään $f$ täydennyspolkua,
kun verkon maksimivirtaus on $f$.
Jos muodostamme polut syvyyshaulla,
jokaisen polun muodostaminen vie aikaa $O(m)$,
joten saamme algoritmin ajankäytölle ylärajan $O(fm)$.

\begin{figure}
\center
\begin{center}
\begin{tikzpicture}[scale=0.8,label distance=-1.5mm]
\node[draw, circle] (1) at (0,-1) {$1$};
\node[draw, circle] (2) at (2,0) {$2$};
\node[draw, circle] (3) at (2,-2) {$3$};
\node[draw, circle] (4) at (4,-1) {$4$};
\path[draw,thick,->] (1) -- node[font=\small,label=above:$10^9$] {} (2);
\path[draw,thick,->] (1) -- node[font=\small,label=below:$10^9$] {} (3);
\path[draw,thick,->] (2) -- node[font=\small,label=right:$1$] {} (3);
\path[draw,thick,->] (2) -- node[font=\small,label=above:$10^9$] {} (4);
\path[draw,thick,->] (3) -- node[font=\small,label=below:$10^9$] {} (4);
\end{tikzpicture}
\end{center}
\caption{Verkko, joka voi aiheuttaa ongelmia algoritmille.}
\label{fig:polhuo}
\end{figure}

Voiko todella käydä niin, että jokainen polku parantaa
virtausta vain yhdellä?
Tämä on mahdollista, ja kuva \ref{fig:polhuo} tarjoaa esimerkin asiasta.
Jos muodostamme polut syvyyshaulla ja valitsemme haussa aina solmun,
jonka tunnus on pienin,
muodostamme vuorotellen polkuja $1 \rightarrow 2 \rightarrow 3 \rightarrow 4$
ja $1 \rightarrow 3 \rightarrow 2 \rightarrow 4$ niin, että
lisäämme ja peruutamme edestakaisin kaarta $2 \rightarrow 3$ kulkevaa virtausta.
Tämän vuoksi joudumme muodostamaan $2 \cdot 10^9$ polkua,
ennen kuin olemme saaneet selville verkon maksimivirtauksen.

\index{Edmonds–Karpin algoritmi}

Voimme kuitenkin estää tämän ilmiön määrittelemällä tarkemmin,
miten muodostamme täydennyspolkuja algoritmin aikana.
\emph{Edmonds–Karpin algoritmi} on Ford–Fulkersonin algoritmin versio,
jossa muodostamme polut \emph{leveyshaulla}.
Tämä tarkoittaa, että valitsemme aina
polun, jossa on mahdollisimman vähän kaaria.
Leveyshakua käyttäen muodostamme kuvan \ref{fig:polhuo} verkossa
vain kaksi täydennyspolkua $1 \rightarrow 2 \rightarrow 4$ ja
$1 \rightarrow 3 \rightarrow 4$, jotka antavat suoraan maksimivirtauksen.

Osoittautuu, että kun käytämme Edmonds–Karpin algoritmia,
meidän täytyy muodostaa aina vain $O(nm)$ täydennyspolkua,
joten algoritmi vie aikaa $O(nm^2)$
riippumatta virtauksen suuruudesta.
Voimme perustella tämän aikavaativuuden kahden havainnon avulla:

\emph{Havainto 1}: Kun $x$ on mikä tahansa verkon solmu,
etäisyys lyhintä polkua pitkin lähtösolmusta solmuun $x$
ei voi koskaan pienentyä täydennyspolun muodostamisen jälkeen.
Voimme perustella tämän tekemällä vastaoletuksen,
että täydennyspolku on juuri pienentänyt etäisyyttä solmuun $b$
ja lyhin polku lähtösolmusta solmuun $b$ päättyy kaareen $a \rightarrow b$,
missä etäisyys solmuun $a$ ei ole pienentynyt.
Tarkastellaan nyt tilannetta ennen kyseisen täydennyspolun muodostamista.
Ensinnäkään verkossa ei voinnut tuolloin olla positiivista kaarta $a \rightarrow b$,
koska muuten myös etäisyys solmuun $a$ olisi pienentynyt.
Toisaalta verkossa ei myöskään voinut olla positiivista kaarta $b \rightarrow a$,
koska tällöin täydennyspolku olisi kulkenut tätä kaarta pitkin
ja $a$:n etäisyys olisi ollut suurempi kuin $b$:n etäisyys.
Niinpä ei ole mahdollista, että solmun $b$ etäisyys olisi pienentynyt.

\emph{Havainto 2}: Jokaiseen täydennyspolkuun liittyy jokin kaari $a \rightarrow b$,
jonka kapasiteetti käytetään kokonaan,
ja jokainen kaari voi esiintyä enintään $n/2$ kertaa
tässä roolissa.
Syynä on, että kun kaaren $a \rightarrow b$ kapasiteetti käytetään kokonaan,
algoritmin täytyy muodostaa täydennyspolku,
joka kulkee vastakkaiseen suuntaan kaarta $b \rightarrow a$,
ennen kuin kaaren $a \rightarrow b$ kapasiteettia voi käyttää uudestaan.
Koska etäisyys lähtösolmusta solmuun ei voi pienentyä,
etäisyys lähtösolmusta kaaren alkusolmuun $a$ kasvaa
ainakin kahdella aina, kun täydennyspolku kulkee toiseen
suuntaan kaarta $b \rightarrow a$.

Näiden havaintojen perusteella Edmonds–Karpin algoritmi
muodostaa enintään $nm/2$ täydennyspolkua, koska voidaan valita
enintään $m$ tavalla, minkä kaaren kapasiteetti käytetään kokonaan, ja
enintään $n/2$ tavalla, mikä on kaaren alkusolmun etäisyys.

\section{Maksimivirtauksen sovelluksia}

\index{palautus}

Maksimivirtauksen etsiminen on tärkeä ongelma,
koska pystymme \emph{palauttamaan} monia verkko-ongelmia maksimivirtaukseen.
Tämä tarkoittaa, että muutamme toisen ongelman jotenkin
sellaiseen muotoon, että se vastaa maksimivirtausta.
Tutustumme seuraavaksi joihinkin tällaisiin ongelmiin.

\subsection{Erilliset polut}

\index{erilliset polut}

Ensimmäinen tehtävämme on muodostaa mahdollisimman monta
\emph{erillistä} polkua verkon lähtösolmusta kohdesolmuun.
Erillisyys tarkoittaa, että jokainen verkon kaari saa esiintyä
enintään yhdellä polulla.
Saamme kuitenkin halutessamme kulkea saman solmun kautta useita kertoja.
Esimerkiksi kuvassa \ref{fig:eripol} voimme muodostaa
kaksi erillistä polkua solmusta $1$ solmuun $5$,
mutta ei ole mahdollista muodostaa kolmea erillistä polkua.

\begin{figure}
\center
\begin{center}
\begin{tikzpicture}[scale=0.8,label distance=-1.5mm]
\newcommand\verkko[0]{
\node[draw, circle] (1) at (0,-1) {$1$};
\node[draw, circle] (2) at (2,0) {$2$};
\node[draw, circle] (3) at (2,-2) {$3$};
\node[draw, circle] (4) at (4,0) {$4$};
\node[draw, circle] (5) at (4,-2) {$5$};
\path[draw,thick,->] (1) -- (2);
\path[draw,thick,->] (1) -- (3);
\path[draw,thick,->] (2) -- (3);
\path[draw,thick,->] (4) -- (2);
\path[draw,thick,->] (3) -- (5);
\path[draw,thick,->] (4) -- (5);
\path[draw,thick,->] (3) -- (4);
}
\begin{scope}
\verkko
\path[draw,thick,->,red,line width=2pt] (1) -- (2);
\path[draw,thick,->,red,line width=2pt] (2) -- (3);
\path[draw,thick,->,red,line width=2pt] (3) -- (4);
\path[draw,thick,->,red,line width=2pt] (4) -- (5);
\end{scope}
\begin{scope}[xshift=7cm]
\verkko
\path[draw,thick,->,red,line width=2pt] (1) -- (3);
\path[draw,thick,->,red,line width=2pt] (3) -- (5);
\end{scope}
\end{tikzpicture}
\end{center}
\caption{Kaksi erillistä polkua solmusta $1$ solmuun $5$.}
\label{fig:eripol}
\end{figure}

\begin{figure}
\center
\begin{center}
\begin{tikzpicture}[scale=0.8,label distance=-1.5mm]
\node[draw, circle] (1) at (0,-1) {$1$};
\node[draw, circle] (2) at (2,0) {$2$};
\node[draw, circle] (3) at (2,-2) {$3$};
\node[draw, circle] (4) at (4,0) {$4$};
\node[draw, circle] (5) at (4,-2) {$5$};
\path[draw,thick,->] (1) -- node[font=\small,label=above:$1/1$] {} (2);
\path[draw,thick,->] (1) -- node[font=\small,label=below:$1/1$] {} (3);
\path[draw,thick,->] (2) -- node[font=\small,label=left:$1/1$] {} (3);
\path[draw,thick,->] (4) -- node[font=\small,label=above:$0/1$] {} (2);
\path[draw,thick,->] (3) -- node[font=\small,label=below:$1/1$] {} (5);
\path[draw,thick,->] (4) -- node[font=\small,label=right:$1/1$] {} (5);
\path[draw,thick,->] (3) -- node[font=\small,label=above:$1/1$] {} (4);
\end{tikzpicture}
\end{center}
\caption{Erilliset polut tulkittuna maksimivirtauksena.}
\label{fig:erivir}
\end{figure}

Voimme ratkaista ongelman
tulkitsemalla erilliset polut maksimivirtauksena.
Ideana on etsiä maksimivirtaus lähtösolmusta kohdesolmuun
olettaen, että jokaisen kaaren kapasiteetti on $1$.
Tämä maksimivirtaus on yhtä suuri kuin suurin
erillisten polkujen määrä.
Kuva \ref{fig:erivir} näyttää maksimivirtauksen
esimerkkiverkossamme.

Miksi maksimivirtaus ja erillisten polkujen määrä ovat yhtä suuret?
Ensinnäkin erilliset polut muodostavat yhdessä virtauksen,
joten maksimivirtaus ei voi olla pienempi kuin erillisten polkujen määrä.
Toisaalta jos verkossa on virtaus, jonka suuruus on $k$,
voimme muodostaa $k$ erillistä polkua
valitsemalla kaaria ahneesti lähtösolmusta alkaen,
joten maksimivirtaus ei voi olla suurempi kuin erillisten polkujen määrä.
Ainoa mahdollisuus on, että maksimivirtaus ja erillisten polkujen määrä
ovat yhtä suuret.

\subsection{Maksimiparitus}

\index{paritus}
\index{maksimiparitus}
\index{kaksijakoinen verkko}

Verkon \emph{paritus} (\emph{matching}) on joukko kaaria, joille pätee,
että jokainen solmu on enintään yhden kaaren päätepisteenä.
\emph{Maksimiparitus} (\emph{maximum matching}) on puolestaan paritus,
jossa on mahdollisimman paljon kaaria.
Keskitymme tapaukseen,
jossa verkko on \emph{kaksijakoinen} (\emph{bipartite}) eli
voimme jakaa verkon solmut
vasempaan ja oikeaan ryhmään niin, että jokainen
kaari kulkee ryhmien välillä.

Kuvassa \ref{fig:makpar} on esimerkkinä kaksijakoinen verkko,
jonka maksimiparitus on $3$.
Tässä vasen ryhmä on $\{1,2,3,4\}$, oikea ryhmä on $\{5,6,7\}$
ja maksimiparitus muodostuu kaarista
$(1,6)$, $(3,5)$ ja $(4,7)$.

\begin{figure}
\center
\begin{center}
\begin{tikzpicture}[scale=0.8,label distance=-1.5mm]
\node[draw, circle] (1) at (0,0) {$1$};
\node[draw, circle] (2) at (0,-1.25) {$2$};
\node[draw, circle] (3) at (0,-2.5) {$3$};
\node[draw, circle] (4) at (0,-3.75) {$4$};
\node[draw, circle] (5) at (4,-0.625) {$5$};
\node[draw, circle] (6) at (4,-1.875) {$6$};
\node[draw, circle] (7) at (4,-3.125) {$7$};
\path[draw,thick,-] (1) -- (6);
\path[draw,thick,-] (2) -- (6);
\path[draw,thick,-] (4) -- (6);
\path[draw,thick,-] (3) -- (5);
\path[draw,thick,-] (3) -- (7);
\path[draw,thick,-] (4) -- (7);
\path[draw,thick,-,red,line width=2pt] (1) -- (6);
\path[draw,thick,-,red,line width=2pt] (3) -- (5);
\path[draw,thick,-,red,line width=2pt] (4) -- (7);
\end{tikzpicture}
\end{center}
\caption{Kaksijakoisen verkon maksimiparitus.}
\label{fig:makpar}
\end{figure}

\begin{figure}
\center
\begin{center}
\begin{tikzpicture}[scale=0.8,label distance=-1.5mm]
\node[draw, circle] (1) at (0,0) {$1$};
\node[draw, circle] (2) at (0,-1.25) {$2$};
\node[draw, circle] (3) at (0,-2.5) {$3$};
\node[draw, circle] (4) at (0,-3.75) {$4$};
\node[draw, circle] (5) at (4,-0.625) {$5$};
\node[draw, circle] (6) at (4,-1.875) {$6$};
\node[draw, circle] (7) at (4,-3.125) {$7$};
\node[draw, circle] (a) at (-3,-1.875) {\phantom{$a$}};
\node[draw, circle] (b) at (7,-1.875) {\phantom{$b$}};
\path[draw,thick,->] (1) -- (6);
\path[draw,thick,->] (2) -- (6);
\path[draw,thick,->] (4) -- (6);
\path[draw,thick,->] (3) -- (5);
\path[draw,thick,->] (3) -- (7);
\path[draw,thick,->] (4) -- (7);
\path[draw,thick,->] (a) -- (1);
\path[draw,thick,->] (a) -- (2);
\path[draw,thick,->] (a) -- (3);
\path[draw,thick,->] (a) -- (4);
\path[draw,thick,->] (5) -- (b);
\path[draw,thick,->] (6) -- (b);
\path[draw,thick,->] (7) -- (b);
\end{tikzpicture}
\end{center}
\caption{Maksimiparitus tulkittuna maksimivirtauksena.}
\label{fig:parver}
\end{figure}

Voimme tulkita maksimiparituksen maksimivirtauksena
lisäämällä verkkoon kaksi uutta solmua: lähtösolmun ja kohdesolmun.
Lähtösolmusta pääsee kaarella jokaiseen vasemman ryhmän solmuun,
ja jokaisesta oikean ryhmän solmusta pääsee kaarella kohdesolmuun.
Lisäksi suuntaamme alkuperäiset kaaret niin,
että ne kulkevat vasemmasta ryhmästä oikeaan ryhmään.
Kuva \ref{fig:parver} näyttää tuloksena olevan verkon
esimerkissämme.
Maksimivirtaus tässä verkossa vastaa alkuperäisen verkon
maksimiparitusta.

\subsection{Pienin polkupeite}

\index{polkupeite}
\index{pienin polkupeite}

\emph{Polkupeite} (\emph{path cover}) on joukko verkon polkuja,
jotka kattavat yhdessä kaikki verkon solmut.
Oletamme, että verkko on suunnattu ja syklitön,
ja haluamme muodostaa mahdollisimman pienen polkupeitteen
niin, että jokainen solmu esiintyy tarkalleen yhdessä polussa.
Kuvassa \ref{fig:polpei} on esimerkkinä verkko ja sen pienin polkupeite,
joka muodostuu kahdesta polusta.

\begin{figure}
\center
\begin{center}
\begin{tikzpicture}[scale=0.8,label distance=-1.5mm]
\node[draw, circle] (1) at (0,0) {$1$};
\node[draw, circle] (2) at (2,0) {$2$};
\node[draw, circle] (3) at (4,0) {$3$};
\node[draw, circle] (4) at (6,0) {$4$};
\node[draw, circle] (5) at (2,-2) {$5$};
\path[draw,thick,->] (1) -- (2);
\path[draw,thick,->] (2) -- (3);
\path[draw,thick,->] (3) -- (4);
\path[draw,thick,->] (5) -- (2);
\path[draw,thick,->] (5) -- (3);
\path[draw,thick,->,red,line width=2pt] (1) -- (2);
\path[draw,thick,->,red,line width=2pt] (5) -- (3);
\path[draw,thick,->,red,line width=2pt] (3) -- (4);
\end{tikzpicture}
\end{center}
\caption{Polkupeite, joka muodostuu poluista $1 \rightarrow 2$ ja $5 \rightarrow 3 \rightarrow 4$.}
\label{fig:polpei}
\end{figure}

Voimme ratkaista pienimmän polkupeitteen etsimisen
ongelman maksimivirtauksen avulla muodostamalla verkon,
jossa jokaista alkuperäistä solmua vastaa kaksi solmua:
vasen ja oikea solmu.
Vasemmasta solmusta on kaari oikeaan solmuun,
jos alkuperäisessä verkossa on vastaava kaari.
Lisäämme vielä verkkoon lähtösolmun ja kohdesolmun niin,
että lähtösolmusta pääsee kaikkiin vasempiin solmuihin
ja kaikista oikeista solmuista pääsee kohdesolmuihin.
Tämän verkon maksimivirtaus antaa meille alkuperäisen
verkon pienimmän solmupeitteen.

\begin{figure}
\center
\begin{center}
\begin{tikzpicture}[scale=0.8,label distance=-1.5mm]
\node[draw, circle] (1a) at (0,0) {$1$};
\node[draw, circle] (2a) at (0,-1.25) {$2$};
\node[draw, circle] (3a) at (0,-2.5) {$3$};
\node[draw, circle] (4a) at (0,-3.75) {$4$};
\node[draw, circle] (5a) at (0,-5) {$5$};
\node[draw, circle] (1b) at (4,0) {$1$};
\node[draw, circle] (2b) at (4,-1.25) {$2$};
\node[draw, circle] (3b) at (4,-2.5) {$3$};
\node[draw, circle] (4b) at (4,-3.75) {$4$};
\node[draw, circle] (5b) at (4,-5) {$5$};
\node[draw, circle] (a) at (-3,-2.5) {\phantom{$a$}};
\node[draw, circle] (b) at (7,-2.5) {\phantom{$b$}};
\path[draw,thick,->] (1a) -- (2b);
\path[draw,thick,->] (2a) -- (3b);
\path[draw,thick,->] (3a) -- (4b);
\path[draw,thick,->] (5a) -- (2b);
\path[draw,thick,->] (5a) -- (3b);
\path[draw,thick,->] (a) -- (1a);
\path[draw,thick,->] (a) -- (2a);
\path[draw,thick,->] (a) -- (3a);
\path[draw,thick,->] (a) -- (4a);
\path[draw,thick,->] (a) -- (5a);
\path[draw,thick,->] (1b) -- (b);
\path[draw,thick,->] (2b) -- (b);
\path[draw,thick,->] (3b) -- (b);
\path[draw,thick,->] (4b) -- (b);
\path[draw,thick,->] (5b) -- (b);
\end{tikzpicture}
\end{center}
\caption{Polkupeitteen etsiminen maksimivirtauksen avulla.}
\label{fig:polvir}
\end{figure}

Kuva \ref{fig:polvir} näyttää tuloksena olevan verkon
esimerkissämme.
Ideana on, että maksimivirtaus etsii,
mitkä kaaret kuuluvat polkuihin:
jos kaari solmusta $a$ solmuun $b$ kuuluu virtaukseen,
niin vastaavasti polkupeitteessä on polku,
jossa on kaari $a \rightarrow b$.
Koska virtauksessa voi olla valittuna vain yksi kaari,
joka alkaa tietystä solmusta tai päättyy tiettyyn solmuun,
tuloksena on varmasti joukko polkuja.
Toisaalta mitä enemmän kaaria saamme laitettua polkuihin,
sitä pienempi on polkujen määrä.
