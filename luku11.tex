\chapter{Suunnatut verkot}

\begin{figure}
\center
\begin{center}
\begin{tikzpicture}[scale=0.7]
\begin{scope}
\node[draw, circle] (1) at (0,-1) {$1$};
\node[draw, circle] (2) at (2,0) {$2$};
\node[draw, circle] (3) at (2,-2) {$3$};
\node[draw, circle] (4) at (4,0) {$4$};
\node[draw, circle] (5) at (4,-2) {$5$};
\path[draw,thick,-] (1) -- (2);
\path[draw,thick,-] (1) -- (3);
\path[draw,thick,-] (3) -- (2);
\path[draw,thick,-] (2) -- (4);
\path[draw,thick,-] (3) -- (4);
\path[draw,thick,-] (5) -- (4);
\node at (2,-3.5) {(a)};
\end{scope}
\begin{scope}[xshift=8cm]
\node[draw, circle] (1) at (0,-1) {$1$};
\node[draw, circle] (2) at (2,0) {$2$};
\node[draw, circle] (3) at (2,-2) {$3$};
\node[draw, circle] (4) at (4,0) {$4$};
\node[draw, circle] (5) at (4,-2) {$5$};
\path[draw,thick,->] (1) -- (2);
\path[draw,thick,->] (1) -- (3);
\path[draw,thick,->] (3) -- (2);
\path[draw,thick,->] (2) -- (4);
\path[draw,thick,->] (3) -- (4);
\path[draw,thick,->] (5) -- (4);
\node at (2,-3.5) {(b)};
\end{scope}
\end{tikzpicture}
\end{center}
\caption{(a) Suuntaamaton verkko. (b) Suunnattu verkko}
\label{fig:versuu}
\end{figure}

Tähän mennessä olemme olettaneet,
että voimme kulkea verkon kaaria kumpaankin suuntaan,
mikä tarkoittaa, että verkko on \emph{suuntaamaton}.
Tässä luvussa kuitenkin keskitymme tilanteeseen,
jossa verkko on \emph{suunnattu}, eli jokaista kaarta
voi kulkea vain merkittyyn suuntaan.

Kuvassa \ref{fig:versuu} on esimerkkinä kaksi verkkoa,
joilla on sama rakenne, mutta vasen verkko on
suuntaamaton ja oikea verkko on suunnattu.
Oikeassa verkossa esimerkiksi solmujen $1$ ja $2$
välinen kaari on suunnattu niin, että voimme kulkea
sitä pitkin solmusta $1$ solmuun $2$,
mutta emme voi kulkea solmusta $2$ solmuun $1$.

Suunnatun verkon käsitteleminen on melko samanlaista
kuin suuntaamattoman verkon:
voimme tallentaa verkon vieruslistoina ja käyttää
syvyyshakua ja leveyshakua kuten ennenkin.
Erona on, että kaarten suunnat rajoittavat
liikkumistamme verkossa.
Suunnatussa verkossa on mahdollista, että solmusta
$a$ solmuun $b$ on polku mutta solmusta
$b$ solmuun $a$ ei ole polkua.
Esimerkiksi verkossa \ref{fig:versuu}(b)
voimme kulkea solmusta $1$ solmuun $4$
mutta emme voi kulkea solmusta $4$ solmuun $1$.

Tämän luvun alussa tutustumme topologiseen järjestämiseen,
joka on mahdollista silloin, kun suunnattu verkko on syklitön.
Se antaa meille verkon solmujen käsittelyjärjestyksen,
jonka avulla voimme käyttää dynaamista ohjelmointia.
Tämän jälkeen käymme läpi algoritmin, joka etsii tehokkaasti
suunnatun verkon vahvasti yhtenäiset komponentit.

\section{Topologinen järjestäminen}

Suunnatun verkon \emph{topologinen järjestys} on solmujen järjestys,
jossa pätee, että jos solmusta $a$ on kaari solmuun $b$,
niin solmu $a$ on ennen solmua $b$ järjestyksessä.
Topologinen järjestys voidaan esittää listana,
joka ilmaisee solmujen järjestyksen.
Kuvassa \ref{fig:topjar} on esimerkkinä verkko ja yksi sen topologinen
järjestys $[1,3,5,2,4]$.

\begin{figure}
\center
\begin{center}
\begin{tikzpicture}[scale=0.7]
\begin{scope}
\node[draw, circle] (1) at (0,-1) {$1$};
\node[draw, circle] (2) at (2,0) {$2$};
\node[draw, circle] (3) at (2,-2) {$3$};
\node[draw, circle] (4) at (4,0) {$4$};
\node[draw, circle] (5) at (4,-2) {$5$};
\path[draw,thick,->] (1) -- (2);
\path[draw,thick,->] (1) -- (3);
\path[draw,thick,->] (3) -- (2);
\path[draw,thick,->] (2) -- (4);
\path[draw,thick,->] (3) -- (4);
\path[draw,thick,->] (5) -- (4);
\end{scope}
\begin{scope}[xshift=7cm]
\node[draw, circle] (1) at (0,-1) {$1$};
\node[draw, circle] (3) at (2,-1) {$3$};
\node[draw, circle] (5) at (4,-1) {$5$};
\node[draw, circle] (2) at (6,-1) {$2$};
\node[draw, circle] (4) at (8,-1) {$4$};
\path[draw,thick,->] (1) edge [bend right] (2);
\path[draw,thick,->] (1) -- (3);
\path[draw,thick,->] (3) edge [bend left] (2);
\path[draw,thick,->] (2) -- (4);
\path[draw,thick,->] (3) edge [bend left] (4);
\path[draw,thick,->] (5) edge [bend right] (4);
\end{scope}
\end{tikzpicture}
\end{center}
\caption{Verkko ja yksi sen topologinen järjestys $[1,3,5,2,4]$.}
\label{fig:topjar}
\end{figure}

Topologinen järjestäminen vaatii, että verkossa ei ole \emph{sykliä}.
Sykli on verkossa oleva polku, jonka lähtösolmu ja
päätesolmu ovat samat.
Jos verkossa on sykli, emme voi muodostaa topologista
järjestystä, koska emme voi valita mitään syklissä
olevaa solmua järjestykseen ennen muita.
Jos taas verkossa ei ole sykliä,
voimme aina muodostaa topologisen järjestyksen.

Suunnatut syklittömät verkot ovat tärkeässä asemassa
algoritmiikassa, ja englannin kielessä niille on jopa
oma nimitys \emph{dag}, joka tulee sanoista
\emph{directed acyclic graph}.

Seuraavaksi tutustumme algoritmiin,
jonka avulla voimme muodostaa topologisen järjestyksen
tai todeta, että verkossa on sykli eikä järjestyksen
muodostaminen ole mahdollista.

\subsection{Järjestyksen muodostaminen}

Voimme muodostaa topologisen järjestyksen suorittamalla
joukon syvyyshakuja, joissa jokaisella solmulla on kolme mahdollista tilaa:

\begin{itemize}
\item tila 0 (valkoinen): solmussa ei ole käyty
\item tila 1 (harmaa): solmun käsittely on kesken
\item tila 2 (musta): solmun käsittely on valmis
\end{itemize}

\begin{figure}
\center
\begin{center}
\begin{tikzpicture}[scale=0.6]
\scriptsize
\newcommand\verkko[6]{
\node[draw, circle, fill=#2] (1) at (0,-1) {$1$};
\node[draw, circle, fill=#3] (2) at (2,0) {$2$};
\node[draw, circle, fill=#4] (3) at (2,-2) {$3$};
\node[draw, circle, fill=#5] (4) at (4,0) {$4$};
\node[draw, circle, fill=#6] (5) at (4,-2) {$5$};
\path[draw,thick,->] (1) -- (2);
\path[draw,thick,->] (1) -- (3);
\path[draw,thick,->] (3) -- (2);
\path[draw,thick,->] (2) -- (4);
\path[draw,thick,->] (3) -- (4);
\path[draw,thick,->] (5) -- (4);
\node at (2,-3) {vaihe #1};
}
\begin{scope}
\verkko{1}{lightgray}{lightgray}{white}{white}{white}
\path[draw=red,thick,->,line width=2pt] (1) -- (2);
\end{scope}
\begin{scope}[xshift=6cm]
\verkko{2}{lightgray}{lightgray}{white}{lightgray}{white}
\path[draw=red,thick,->,line width=2pt] (1) -- (2);
\path[draw=red,thick,->,line width=2pt] (2) -- (4);
\end{scope}
\begin{scope}[xshift=12cm]
\verkko{3}{lightgray}{lightgray}{white}{gray}{white}
\path[draw=red,thick,->,line width=2pt] (1) -- (2);
\end{scope}
\begin{scope}[xshift=18cm]
\verkko{4}{lightgray}{gray}{white}{gray}{white}
\end{scope}
\begin{scope}[yshift=-5cm]
\verkko{5}{lightgray}{gray}{lightgray}{gray}{white}
\path[draw=red,thick,->,line width=2pt] (1) -- (3);
\end{scope}
\begin{scope}[yshift=-5cm,xshift=6cm]
\verkko{6}{lightgray}{gray}{lightgray}{gray}{white}
\path[draw=red,thick,->,line width=2pt] (1) -- (3);
\path[draw=red,thick,->,line width=2pt] (3) -- (2);
\end{scope}
\begin{scope}[yshift=-5cm,xshift=12cm]
\verkko{7}{lightgray}{gray}{lightgray}{gray}{white}
\path[draw=red,thick,->,line width=2pt] (1) -- (3);
\path[draw=red,thick,->,line width=2pt] (3) -- (4);
\end{scope}
\begin{scope}[yshift=-5cm,xshift=18cm]
\verkko{8}{lightgray}{gray}{gray}{gray}{white}
\path[draw=red,thick,->,line width=2pt] (1) -- (3);
\end{scope}
\begin{scope}[yshift=-10cm]
\verkko{9}{gray}{gray}{gray}{gray}{white}
\end{scope}
\begin{scope}[yshift=-10cm,xshift=6cm]
\verkko{10}{gray}{gray}{gray}{gray}{lightgray}
\path[draw=red,thick,->,line width=2pt] (5) -- (4);
\end{scope}
\begin{scope}[yshift=-10cm,xshift=12cm]
\verkko{11}{gray}{gray}{gray}{gray}{gray}
\end{scope}
\end{tikzpicture}
\end{center}
\caption{Esimerkki topologisen järjestyksen muodostamisesta.}
\label{fig:topesi}
\end{figure}

Algoritmin alussa jokainen solmu on valkoinen.
Käymme läpi kaikki verkon solmut ja aloitamme aina syvyyshaun
solmusta, jos se on valkoinen.
Aina kun saavumme uuteen solmuun, sen väri muuttuu
valkoisesta harmaaksi.
Sitten kun olemme käsitelleet kaikki solmusta lähtevät
kaaret, solmun väri muuttuu harmaasta mustaksi
ja lisäämme solmun listalle.
Tämä lista käänteisessä järjestyksessä on verkon
topologinen järjestys.
Kuitenkin jos saavumme jossain vaiheessa algoritmia
toista kautta harmaaseen solmuun,
verkossa on sykli eikä topologista järjestystä voi muodostaa.

Kuva \ref{fig:topesi} näyttää, kuinka algoritmi muodostaa topologisen
järjestyksen esimerkkiverkossamme.
Tässä tapauksessa suoritamme kaksi syvyyshakua,
joista ensimmäinen alkaa solmusta 1 ja toinen alkaa solmusta 5.
Algoritmin tuloksena on lista $[4,2,3,1,5]$,
joten käänteinen lista $[5,1,3,2,4]$ on verkon topologinen järjestys.
Huomaa, että tämä on eri järjestys kuin kuvassa \ref{fig:topjar};
topologinen järjestys ei ole yksikäsitteinen ja voimme yleensä
muodostaa järjestyksen monella tavalla.

\begin{figure}
\center
\begin{center}
\begin{tikzpicture}[scale=0.6]
\scriptsize
\newcommand\verkko[6]{
\node[draw, circle, fill=#2] (1) at (0,-1) {$1$};
\node[draw, circle, fill=#3] (2) at (2,0) {$2$};
\node[draw, circle, fill=#4] (3) at (2,-2) {$3$};
\node[draw, circle, fill=#5] (4) at (4,0) {$4$};
\node[draw, circle, fill=#6] (5) at (4,-2) {$5$};
\path[draw,thick,->] (1) -- (2);
\path[draw,thick,->] (1) -- (3);
\path[draw,thick,->] (3) -- (2);
\path[draw,thick,->] (2) -- (4);
\path[draw,thick,->] (4) -- (3);
\path[draw,thick,->] (5) -- (4);
\node at (2,-3) {vaihe #1};
}
\begin{scope}
\verkko{1}{lightgray}{lightgray}{white}{white}{white}
\path[draw=red,thick,->,line width=2pt] (1) -- (2);
\end{scope}
\begin{scope}[xshift=6cm]
\verkko{2}{lightgray}{lightgray}{white}{lightgray}{white}
\path[draw=red,thick,->,line width=2pt] (1) -- (2);
\path[draw=red,thick,->,line width=2pt] (2) -- (4);
\end{scope}
\begin{scope}[xshift=12cm]
\verkko{3}{lightgray}{lightgray}{lightgray}{lightgray}{white}
\path[draw=red,thick,->,line width=2pt] (1) -- (2);
\path[draw=red,thick,->,line width=2pt] (2) -- (4);
\path[draw=red,thick,->,line width=2pt] (4) -- (3);
\end{scope}
\begin{scope}[xshift=18cm]
\verkko{4}{lightgray}{lightgray}{lightgray}{lightgray}{white}
\path[draw=red,thick,->,line width=2pt] (1) -- (2);
\path[draw=red,thick,->,line width=2pt] (2) -- (4);
\path[draw=red,thick,->,line width=2pt] (4) -- (3);
\path[draw=red,thick,->,line width=2pt] (3) -- (2);
\end{scope}
\end{tikzpicture}
\end{center}
\caption{Topologista järjestystä ei voi muodostaa syklin takia.}
\label{fig:topsyk}
\end{figure}

Kuva \ref{fig:topsyk} näyttää puolestaan esimerkin tilanteesta,
jossa topologista järjestystä ei voi muodostaa verkossa
olevan syklin takia.
Tässä verkossa on sykli $2 \rightarrow 4 \rightarrow 3 \rightarrow 2$,
jonka olemassaolon huomaamme siitä, että tulemme uudestaan
harmaaseen solmuun 2.

Algoritmi käy läpi verkon solmut ja kaaret syvyyshaun avulla,
joten sen aikavaativuus on $O(n+m)$.

\subsection{Miksi algoritmi toimii?}

Miten voimme tietää, että algoritmimme topologisen
järjestyksen muodostamiseen toimii
oikein kaikissa mahdollisissa tilanteissa?

Tarkastellaan ensin tilannetta, jossa verkossa on sykli.
Jos algoritmi saapuu uudestaan harmaaseen solmuun $x$,
on selvää, että verkossa on sykli,
koska algoritmi on onnistunut pääsemään solmusta $x$
takaisin itseensä kulkemalla jotain polkua verkossa.
Toisaalta jos verkossa on sykli, algoritmi saapuu
jossain vaiheessa ensimmäistä kertaa johonkin sykliin
kuuluvaan solmuun $x$. Sen jälkeen se käy läpi solmusta
lähtevät kaaret ja aikanaan saapuu varmasti uudestaan
solmuun $x$. Niinpä algoritmi onnistuu kaikissa tilanteissa tunnistamaan,
jos verkossa on sykli.

Jos sitten verkossa ei ole sykliä, algoritmi lisää jokaisen
solmun listalle sen jälkeen, kun se on käsitellyt
kaikki solmusta lähtevät kaaret.
Jos siis verkossa on kaari $a \rightarrow b$,
solmu $b$ lisätään listalle ennen solmua $a$.
Lopuksi lista käännetään, jolloin solmu $a$
tulee ennen solmua $b$.
Tämän ansiosta jokaiselle kaarelle $a \rightarrow b$ pätee,
että solmu $a$ on järjestyksessä ennen solmua $b$.

\subsection{Esimerkki: Kurssivalinnat}

Yliopiston kurssit ja niiden esitietovaatimukset voidaan esittää 
suunnattuna verkkona, jonka solmut ovat kursseja ja kaaret kuvaavat,
missä järjestyksessä kurssit tulisi suorittaa.
Kuvassa \ref{fig:kuresi} on esimerkkinä joitakin
tietojenkäsittely\-tieteen kandiohjelman kursseja.
Tällaisen verkon topologinen järjestys kertoo meille
yhden tavan suorittaa kurssit esitietovaatimusten mukaisesti.
Kuvassa \ref{fig:kurjar} näkyy yksi mahdollinen topologinen järjestys,
joka antaa suoritusjärjestyksen
OHPE, OHJA, TIPE, TITO, JYM, TIRA, LAMA.

\begin{figure}
\center
\begin{center}
\begin{tikzpicture}[scale=0.7]
\node[draw, rectangle] (1) at (0,0) {OHPE};
\node[draw, rectangle] (2) at (-4,-2) {OHJA};
\node[draw, rectangle] (3) at (4,-2) {TITO};
\node[draw, rectangle] (4) at (0,-2) {TIPE};
\node[draw, rectangle] (5) at (-8,-2) {JYM};
\node[draw, rectangle] (6) at (-4,-4) {TIRA};
\node[draw, rectangle] (7) at (-4,-6) {LAMA};
\path[draw,thick,->] (1) -- (2);
\path[draw,thick,->] (1) -- (3);
\path[draw,thick,->] (1) -- (4);
\path[draw,thick,->] (2) -- (4);
\path[draw,thick,->] (2) -- (6);
\path[draw,thick,->] (5) -- (6);
\path[draw,thick,->] (5) -- (7);
\path[draw,thick,->] (6) -- (7);
\end{tikzpicture}
\end{center}
\caption{Kurssien esitietovaatimukset verkkona.}
\label{fig:kuresi}
\end{figure}


\begin{figure}
\center
\begin{center}
\begin{tikzpicture}[scale=0.7]
\node[draw, rectangle] (1) at (0,0) {OHPE};
\node[draw, rectangle] (2) at (2.5,0) {OHJA};
\node[draw, rectangle] (4) at (5,0) {TIPE};
\node[draw, rectangle] (3) at (7.5,0) {TITO};
\node[draw, rectangle] (5) at (10,0) {JYM};
\node[draw, rectangle] (6) at (12.5,0) {TIRA};
\node[draw, rectangle] (7) at (15,0) {LAMA};
\path[draw,thick,->] (1) -- (2);
\path[draw,thick,->] (1) edge [bend left] (3);
\path[draw,thick,->] (1) edge [bend right] (4);
\path[draw,thick,->] (2) -- (4);
\path[draw,thick,->] (2) edge [bend left] (6);
\path[draw,thick,->] (5) -- (6);
\path[draw,thick,->] (5) edge [bend right] (7);
\path[draw,thick,->] (6) -- (7);
\end{tikzpicture}
\end{center}
\caption{Topologinen järjestys antaa kurssien suoritusjärjestyksen.}
\label{fig:kurjar}
\end{figure}

On selvää, että kurssien ja esitietovaatimusten muodostaman
verkon tulee olla syklitön, jotta kurssit voi suorittaa halutulla tavalla.
Jos verkossa on sykli, topologista järjestystä ei ole olemassa
eikä meillä ole mitään mahdollisuutta suorittaa kursseja
esitietovaatimusten mukaisesti.

\section{Dynaaminen ohjelmointi}

Kun tiedämme, että suunnattu verkko on syklitön,
voimme ratkaista helposti monia verkon polkuihin
liittyviä ongelmia \emph{dynaamisen ohjelmoinnin} avulla.
Tämä on mahdollista, koska topologinen järjestys antaa
meille selkeän järjestyksen, jossa voimme käsitellä solmut.

\subsection{Polkujen laskeminen}

Tarkastellaan esimerkkinä ongelmaa, jossa haluamme
laskea, montako erilaista polkua verkossa on
solmusta $a$ solmuun $b$.
Kuva \ref{fig:verpol} näyttää esimerkin,
jossa laskemme polkujen määrän solmusta 1
solmuun 4.
Tässä tapauksessa polkuja on kolme:
$1 \rightarrow 2 \rightarrow 4$,
$1 \rightarrow 3 \rightarrow 2 \rightarrow 4$ ja
$1 \rightarrow 3 \rightarrow 4$.

\begin{figure}
\center
\begin{center}
\begin{tikzpicture}[scale=0.7]
\begin{scope}
\node[draw, circle] (1) at (0,-1) {$1$};
\node[draw, circle] (2) at (2,0) {$2$};
\node[draw, circle] (3) at (2,-2) {$3$};
\node[draw, circle] (4) at (4,0) {$4$};
\node[draw, circle] (5) at (4,-2) {$5$};
\path[draw,thick,->] (1) -- (2);
\path[draw,thick,->] (1) -- (3);
\path[draw,thick,->] (3) -- (2);
\path[draw,thick,->] (2) -- (4);
\path[draw,thick,->] (3) -- (4);
\path[draw,thick,->] (5) -- (4);
\path[draw=red,thick,->,line width=2pt] (1) -- (2);
\path[draw=red,thick,->,line width=2pt] (2) -- (4);
\end{scope}
\begin{scope}[xshift=6.5cm]
\node[draw, circle] (1) at (0,-1) {$1$};
\node[draw, circle] (2) at (2,0) {$2$};
\node[draw, circle] (3) at (2,-2) {$3$};
\node[draw, circle] (4) at (4,0) {$4$};
\node[draw, circle] (5) at (4,-2) {$5$};
\path[draw,thick,->] (1) -- (2);
\path[draw,thick,->] (1) -- (3);
\path[draw,thick,->] (3) -- (2);
\path[draw,thick,->] (2) -- (4);
\path[draw,thick,->] (3) -- (4);
\path[draw,thick,->] (5) -- (4);
\path[draw=red,thick,->,line width=2pt] (1) -- (3);
\path[draw=red,thick,->,line width=2pt] (3) -- (2);
\path[draw=red,thick,->,line width=2pt] (2) -- (4);
\end{scope}
\begin{scope}[xshift=13cm]
\node[draw, circle] (1) at (0,-1) {$1$};
\node[draw, circle] (2) at (2,0) {$2$};
\node[draw, circle] (3) at (2,-2) {$3$};
\node[draw, circle] (4) at (4,0) {$4$};
\node[draw, circle] (5) at (4,-2) {$5$};
\path[draw,thick,->] (1) -- (2);
\path[draw,thick,->] (1) -- (3);
\path[draw,thick,->] (3) -- (2);
\path[draw,thick,->] (2) -- (4);
\path[draw,thick,->] (3) -- (4);
\path[draw,thick,->] (5) -- (4);
\path[draw=red,thick,->,line width=2pt] (1) -- (3);
\path[draw=red,thick,->,line width=2pt] (3) -- (4);
\end{scope}
\end{tikzpicture}
\end{center}
\caption{Mahdolliset polut solmusta 1 solmuun 4.}
\label{fig:verpol}
\end{figure}

Polkujen määrän laskeminen on vaikea ongelma yleisessä
verkossa, jossa voi olla syklejä.
Itse asiassa tehtävä ei ole edes mielekäs:
jos verkossa on sykli, voimme kulkea sitä miten
monta kertaa tahansa ja tuottaa aina vain uusia polkuja,
joten polkuja tulee äärettömästi.
Nyt kuitenkin oletamme, että verkko on syklitön,
jolloin voimme laskea polkujen määrän tehokkaasti
dynaamisen ohjelmoinnin avulla.

Jotta voimme käyttää dynaamista ohjelmointia,
meidän täytyy määritellä ongelma rekursiivisesti.
Sopiva funktio on $\texttt{polut}(x)$,
joka antaa polkujen määrän solmusta $a$ solmuun $x$.
Tätä funktiota käyttäen $\texttt{polut}(b)$
vastaa tehtävän ratkaisua.
Esimerkiksi kuvan \ref{fig:verpol} tilanteessa lähtösolmu on $a=1$
ja funktion arvot ovat seuraavat:

\begin{align*}
\texttt{polut}(1)&=1 \\
\texttt{polut}(2)&=2 \\
\texttt{polut}(3)&=1 \\
\texttt{polut}(4)&=3 \\
\texttt{polut}(5)&=0 \\
\end{align*}

Nyt meidän täytyy enää löytää kaava funktion
arvojen laskemiseen.
Pohjatapauksessa olemme solmussa $a$,
jolloin polkuja on aina yksi:

\[ \texttt{polut}(a)=1 \]

Entä sitten, kun olemme jossain muussa solmussa $x$?
Tällöin käymme läpi kaikki solmut, joista pääsemme
solmuun $x$ kaarella, ja laskemme yhteen näihin
solmuihin tulevien polkujen määrät.
Kun oletamme, että solmuun $x$ pääsee solmuista $u_1,u_2,\dots,u_k$,
saamme seuraavan rekursiivisen kaavan:
\[ \texttt{polut}(x)=\texttt{polut}(u_1)+\texttt{polut}(u_2)+\dots+\texttt{polut}(u_k) \]

Esimerkiksi kuvan \ref{fig:verpol} verkossa
solmuun $4$ pääsee solmuista $2$, $3$ ja $5$.
Tämän seurauksena
\[ \texttt{polut}(4)=\texttt{polut}(2)+\texttt{polut}(3)+\texttt{polut}(5) = 2+1+0 = 3.\]

Koska tiedämme, että verkko on syklitön,
voimme laskea funktion arvoja
tehokkaasti dynaamisella ohjelmoinnilla.
Oleellista on, että emme voi joutua koskaan silmukkaan
laskiessamme arvoja.
Käytännössä voimme laskea arvot solmujen
topologisessa järjestyksessä.
Tällä tavalla saamme ongelmaan ratkaisun, joka
vie aikaa $O(n+m)$.

\subsection{Ongelmat verkkoina}

Itse asiassa voimme esittää dynaamisen ohjelmoinnin toiminnan
\emph{aina} suunnattuna syklittömänä verkkona.
Ideana on, että muodostamme verkon, jossa jokainen solmu on
yksi osaongelma ja kaaret ilmaisevat,
miten osaongelmat liittyvät toisiinsa.

Tarkastellaan esimerkkinä luvusta 9.1 tuttua tehtävää,
jossa haluamme laskea, monellako tavalla voimme muodostaa
korkeuden $n$ tornin, kun voimme käyttää palikoita,
joiden korkeudet ovat $1$, $2$ ja $3$.
Voimme esittää tämän tehtävän verkkona niin,
että solmut ovat tornien korkeuksia ja kaaret kertovat,
kuinka voimme rakentaa tornia palikoista.
Jokaisesta solmusta $x$ on kaari solmuihin
$x+1$, $x+2$ ja $x+3$,
ja polkujen määrä solmusta $0$ solmuun $n$
on yhtä suuri kuin tornin rakentamistapojen määrä.

Esimerkiksi kuva \ref{fig:verkol} näyttää verkon,
joka vastaa tapausta $n=6$.
Tässä verkossa solmusta $0$ solmuun $6$ on $24$ polkua,
eli voimme rakentaa korkeuden $6$ tornin $24$ tavalla.

\begin{figure}
\center
\begin{center}
\begin{tikzpicture}[scale=0.7]
\node[draw, circle] (0) at (0,0) {$0$};
\node[draw, circle] (1) at (2,0) {$1$};
\node[draw, circle] (2) at (4,0) {$2$};
\node[draw, circle] (3) at (6,0) {$3$};
\node[draw, circle] (4) at (8,0) {$4$};
\node[draw, circle] (5) at (10,0) {$5$};
\node[draw, circle] (6) at (12,0) {$6$};
\path[draw,thick,->] (0) -- (1);
\path[draw,thick,->] (1) -- (2);
\path[draw,thick,->] (2) -- (3);
\path[draw,thick,->] (3) -- (4);
\path[draw,thick,->] (4) -- (5);
\path[draw,thick,->] (5) -- (6);
\path[draw,thick,->] (0) edge [bend left] (2);
\path[draw,thick,->] (1) edge [bend left] (3);
\path[draw,thick,->] (2) edge [bend left] (4);
\path[draw,thick,->] (3) edge [bend left] (5);
\path[draw,thick,->] (4) edge [bend left] (6);
\path[draw,thick,->] (0) edge [bend right] (3);
\path[draw,thick,->] (1) edge [bend right] (4);
\path[draw,thick,->] (2) edge [bend right] (5);
\path[draw,thick,->] (3) edge [bend right] (6);
\end{tikzpicture}
\end{center}
\caption{Tornitehtävä esitettynä verkkona.}
\label{fig:verkol}
\end{figure}

Olemme saaneet siis uuden tavan luonnehtia dynaamista ohjelmointia:
voimme käyttää dynaamista ohjelmointia,
jos pystymme esittämään ongelman suunnattuna syklittömänä verkkona.

\section{Vahvasti yhtenäisyys}

Jos suunnatussa verkossa on sykli,
emme voi muodostaa sille topologista järjestystä
emmekä käyttää dynaamista ohjelmointia.
Mikä neuvoksi, jos kuitenkin haluaisimme tehdä näin?

Joskus voimme selviytyä tilanteesta käsittelemällä
verkon vahvasti yhtenäisiä komponentteja.
Sanomme, että suunnattu verkko on \emph{vahvasti yhtenäinen},
jos mistä tahansa solmusta on polku mihin tahansa solmuun.
Voimme esittää suunnatun verkon aina yhtenä tai
useampana vahvasti yhtenäisenä komponenttina,
joista muodostuu syklitön \emph{komponenttiverkko}.
Tämä verkko esittää alkuperäisen verkon syvärakenteen.

\begin{figure}
\center
\begin{center}
\begin{tikzpicture}[scale=0.7]
\begin{scope}
\node[draw, circle] (1) at (0,0) {$1$};
\node[draw, circle] (2) at (0,-2) {$2$};
\node[draw, circle] (3) at (2,0) {$3$};
\node[draw, circle] (4) at (2,-2) {$4$};
\node[draw, circle] (5) at (4,0) {$5$};
\node[draw, circle] (6) at (4,-2) {$6$};

\path[draw,thick,->] (1) -- (3);
\path[draw,thick,->] (3) -- (4);
\path[draw,thick,->] (4) -- (2);
\path[draw,thick,->] (2) -- (1);
\path[draw,thick,->] (3) -- (2);

\path[draw,thick,->] (5) edge [bend left] (6);
\path[draw,thick,->] (6) edge [bend left] (5);

\path[draw,thick,->] (3) -- (5);
\path[draw,thick,->] (4) -- (6);

\draw[red,dashed,thick,line width=2pt] (-0.75,0.75) rectangle (2.75,-2.75);
\draw[red,dashed,thick,line width=2pt] (3.25,0.75) rectangle (4.75,-2.75);

\node at (2,-3.5) {(a)};
\end{scope}
\begin{scope}[xshift=8cm,yshift=-1cm]
\node[draw,rectangle] (1) at (0,0) {$\{1,2,3,4\}$};
\node[draw,rectangle] (2) at (4,0) {$\{5,6\}$};
\path[draw,thick,->] (1) -- (2);
\node at (2,-2.5) {(b)};
\end{scope}
\end{tikzpicture}
\end{center}
\caption{(a) Verkon vahvasti yhtenäiset komponentit.
(b) Komponenttiverkko, joka kuvaa verkon syvärakenteen.}
\label{fig:vahkom}
\end{figure}

Kuvassa \ref{fig:vahkom} on esimerkkinä verkko, joka muodostuu
kahdesta vahvasti yhtenäisestä komponentista.
Ensimmäinen komponentti on $\{1,2,3,4\}$
ja toinen komponentti on $\{5,6\}$.
Komponenteista muodostuu syklitön komponenttiverkko,
jossa on kaari solmusta $\{1,2,3,4\}$ solmuun $\{5,6\}$.

Seuraavaksi tutustumme Kosarajun algoritmiin,
jonka avulla pystymme muodostamaan tehokkaasti suunnatun
verkon vahvasti yhtenäiset komponentit ja niitä
vastaavan komponenttiverkon.

\subsection{Kosarajun algoritmi}

Kosarajun algoritmi muodostuu kahdesta vaiheesta,
joista kumpikin käy läpi verkon solmut syvyyshaun avulla.
Ensimmäinen vaihe muistuttaa topologisen järjestyksen
etsimistä ja tuottaa listan solmuista.
Toinen vaihe muodostaa vahvasti yhtenäiset komponentit
tämän listan perusteella.

Algoritmin ensimmäisessä vaiheessa käymme läpi verkon
solmut ja aloitamme uuden syvyyshaun aina,
kun tulemme solmuun, jossa emme ole vielä käyneet.
Jokaisen solmun kohdalla etenemme ensin kaaria pitkin
solmuihin, joissa emme ole vielä käyneet.
Tämän jälkeen lisäämme solmun listalle.
Toimimme siis kuten topologisen järjestyksen muodostamisessa,
mutta emme välitä, jos tulemme toista reittiä
käsittelyssä olevaan solmuun.

Algoritmin toisen vaiheen alussa
käännämme jokaisen verkon kaaren suunnan.
Tämän jälkeen käsittelemme käänteisessä järjestyksessä
ensimmäisen vaiheen listalle keräämät solmut.
Jokaisen solmun kohdalla muodostamme uuden vahvasti yhtenäisen
komponentin, jossa on kaikki vielä käsittelemät\-tömät solmut,
joihin pääsemme solmusta.

\begin{figure}
\center
\begin{center}
\begin{tikzpicture}[scale=0.6]
\scriptsize
\newcommand\verkko[7]{
\node[draw, circle, fill=#2] (1) at (0,0) {$1$};
\node[draw, circle, fill=#3] (2) at (0,-2) {$2$};
\node[draw, circle, fill=#4] (3) at (2,0) {$3$};
\node[draw, circle, fill=#5] (4) at (2,-2) {$4$};
\node[draw, circle, fill=#6] (5) at (4,0) {$5$};
\node[draw, circle, fill=#7] (6) at (4,-2) {$6$};

\path[draw,thick,->] (1) -- (3);
\path[draw,thick,->] (3) -- (4);
\path[draw,thick,->] (4) -- (2);
\path[draw,thick,->] (2) -- (1);
\path[draw,thick,->] (3) -- (2);
\path[draw,thick,->] (5) edge [bend left] (6);
\path[draw,thick,->] (6) edge [bend left] (5);
\path[draw,thick,->] (3) -- (5);
\path[draw,thick,->] (4) -- (6);

\node at (2,-3) {vaihe #1};
}
\begin{scope}
\verkko{1}{lightgray}{white}{white}{white}{white}{white}
\end{scope}
\begin{scope}[xshift=6cm]
\verkko{2}{lightgray}{white}{lightgray}{white}{white}{white}
\path[draw=red,thick,->,line width=2pt] (1) -- (3);
\end{scope}
\begin{scope}[xshift=12cm]
\verkko{3}{lightgray}{lightgray}{lightgray}{white}{white}{white}
\path[draw=red,thick,->,line width=2pt] (1) -- (3);
\path[draw=red,thick,->,line width=2pt] (3) -- (2);
\end{scope}
\begin{scope}[xshift=18cm]
\verkko{4}{lightgray}{lightgray}{lightgray}{white}{white}{white}
\path[draw=red,thick,->,line width=2pt] (1) -- (3);
\end{scope}
\begin{scope}[yshift=-5cm]
\verkko{5}{lightgray}{lightgray}{lightgray}{lightgray}{white}{white}
\path[draw=red,thick,->,line width=2pt] (1) -- (3);
\path[draw=red,thick,->,line width=2pt] (3) -- (4);
\end{scope}
\begin{scope}[yshift=-5cm,xshift=6cm]
\verkko{6}{lightgray}{lightgray}{lightgray}{lightgray}{white}{lightgray}
\path[draw=red,thick,->,line width=2pt] (1) -- (3);
\path[draw=red,thick,->,line width=2pt] (3) -- (4);
\path[draw=red,thick,->,line width=2pt] (4) -- (6);
\end{scope}
\begin{scope}[yshift=-5cm,xshift=12cm]
\verkko{7}{lightgray}{lightgray}{lightgray}{lightgray}{lightgray}{lightgray}
\path[draw=red,thick,->,line width=2pt] (1) -- (3);
\path[draw=red,thick,->,line width=2pt] (3) -- (4);
\path[draw=red,thick,->,line width=2pt] (4) -- (6);
\path[draw=red,thick,->,line width=2pt] (6) edge [bend left] (5);
\end{scope}
\begin{scope}[yshift=-5cm,xshift=18cm]
\verkko{8}{lightgray}{lightgray}{lightgray}{lightgray}{lightgray}{lightgray}
\path[draw=red,thick,->,line width=2pt] (1) -- (3);
\path[draw=red,thick,->,line width=2pt] (3) -- (4);
\path[draw=red,thick,->,line width=2pt] (4) -- (6);
\end{scope}
\begin{scope}[yshift=-10cm]
\verkko{9}{lightgray}{lightgray}{lightgray}{lightgray}{lightgray}{lightgray}
\path[draw=red,thick,->,line width=2pt] (1) -- (3);
\path[draw=red,thick,->,line width=2pt] (3) -- (4);
\end{scope}
\begin{scope}[yshift=-10cm,xshift=6cm]
\verkko{10}{lightgray}{lightgray}{lightgray}{lightgray}{lightgray}{lightgray}
\path[draw=red,thick,->,line width=2pt] (1) -- (3);
\end{scope}
\begin{scope}[yshift=-10cm,xshift=12cm]
\verkko{11}{lightgray}{lightgray}{lightgray}{lightgray}{lightgray}{lightgray}
\end{scope}
\end{tikzpicture}
\end{center}
\caption{Kosarajun algoritmin ensimmäinen läpikäynti.}
\label{fig:koseka}
\end{figure}

Tarkastelemme seuraavaksi, kuinka Kosarajun algoritmi
toimii esimerkkiverkossamme.
Kuva \ref{fig:koseka} näyttää ensimmäisen läpikäynnin,
joka muodostaa solmuista listan $[2,5,6,4,3,1]$.
Tämän jälkeen käännämme kaikki verkon kaaret ja
käsittelemme solmut järjestyksessä $[1,3,4,6,5,2]$.
Kuva \ref{fig:kostok} näyttää toisen läpikäynnin.
Vahvasti yhtenäiset komponentit syntyvät
solmuista 1 ja 6 alkaen.
Kaarten kääntämisen ansiosta
solmusta 1 alkava
vahvasti yhtenäinen komponentti ei ''vuoda''
solmujen 5 ja 6 alueelle.

\begin{figure}
\center
\begin{center}
\begin{tikzpicture}[scale=0.6]
\scriptsize
\newcommand\verkko[7]{
\node[draw, circle, fill=#2] (1) at (0,0) {$1$};
\node[draw, circle, fill=#3] (2) at (0,-2) {$2$};
\node[draw, circle, fill=#4] (3) at (2,0) {$3$};
\node[draw, circle, fill=#5] (4) at (2,-2) {$4$};
\node[draw, circle, fill=#6] (5) at (4,0) {$5$};
\node[draw, circle, fill=#7] (6) at (4,-2) {$6$};

\path[draw,thick,<-] (1) -- (3);
\path[draw,thick,<-] (3) -- (4);
\path[draw,thick,<-] (4) -- (2);
\path[draw,thick,<-] (2) -- (1);
\path[draw,thick,<-] (3) -- (2);
\path[draw,thick,<-] (5) edge [bend left] (6);
\path[draw,thick,<-] (6) edge [bend left] (5);
\path[draw,thick,<-] (3) -- (5);
\path[draw,thick,<-] (4) -- (6);

\node at (2,-3) {vaihe #1};
}
\begin{scope}
\verkko{1}{lightgray}{white}{white}{white}{white}{white}
\end{scope}
\begin{scope}[xshift=6cm]
\verkko{2}{lightgray}{lightgray}{white}{white}{white}{white}
\path[draw=red,thick,->,line width=2pt] (1) -- (2);
\end{scope}
\begin{scope}[xshift=12cm]
\verkko{3}{lightgray}{lightgray}{lightgray}{white}{white}{white}
\path[draw=red,thick,->,line width=2pt] (1) -- (2);
\path[draw=red,thick,->,line width=2pt] (2) -- (3);
\end{scope}
\begin{scope}[xshift=18cm]
\verkko{4}{lightgray}{lightgray}{lightgray}{white}{white}{white}
\path[draw=red,thick,->,line width=2pt] (1) -- (2);
\end{scope}
\begin{scope}[yshift=-5cm]
\verkko{5}{lightgray}{lightgray}{lightgray}{lightgray}{white}{white}
\path[draw=red,thick,->,line width=2pt] (1) -- (2);
\path[draw=red,thick,->,line width=2pt] (2) -- (4);
\end{scope}
\begin{scope}[yshift=-5cm,xshift=6cm]
\verkko{6}{lightgray}{lightgray}{lightgray}{lightgray}{white}{white}
\path[draw=red,thick,->,line width=2pt] (1) -- (2);
\end{scope}
\begin{scope}[yshift=-5cm,xshift=12cm]
\verkko{7}{lightgray}{lightgray}{lightgray}{lightgray}{white}{white}
\draw[red,dashed,thick,line width=1.5pt] (-0.75,0.75) rectangle (2.75,-2.75);
\end{scope}
\begin{scope}[yshift=-5cm,xshift=18cm]
\verkko{8}{lightgray}{lightgray}{lightgray}{lightgray}{white}{lightgray}
\end{scope}
\begin{scope}[yshift=-10cm]
\verkko{9}{lightgray}{lightgray}{lightgray}{lightgray}{lightgray}{lightgray}
\path[draw=red,thick,<-,line width=2pt] (5) edge [bend left] (6);
\end{scope}
\begin{scope}[yshift=-10cm,xshift=6cm]
\verkko{9}{lightgray}{lightgray}{lightgray}{lightgray}{lightgray}{lightgray}
\draw[red,dashed,thick,line width=1.5pt] (3.25,0.75) rectangle (4.75,-2.75);
\end{scope}
\end{tikzpicture}
\end{center}
\caption{Kosarajun algoritmin toinen läpikäynti.}
\label{fig:kostok}
\end{figure}

Koska algoritmi muodostuu kahdesta verkon läpikäynnistä,
sen aikavaativuus on $O(n+m)$.

\subsection{Miksi algoritmi toimii?}

Keskeinen kysymys Kosarajun algoritmiin liittyen on,
miten voimme olla varmoja, että algoritmin toinen vaihe
muodostaa vahvasti yhtenäisiä komponentteja,
joihin ei tule ylimääräisiä solmuja?

Voimme tarkastella asiaa muodostettavan
komponenttiverkon näkökul\-masta.
Jos meillä on komponentti $A$, josta pääsee kaarella
komponenttiin $B$, 
algoritmin ensimmäisessä vaiheessa
jokin $A$:n solmu lisätään listalle kaikkien $B$:n
solmujen jälkeen.
Kun sitten käymme läpi listan käänteisessä järjestyk\-sessä,
jokin $A$:n solmu tulee vastaan ennen kaikkia $B$:n
solmuja.
Niinpä alamme rakentaa ensin komponenttia $A$
emmekä mene komponentin $B$ puolelle,
koska verkon kaaret on käännetty.
Sitten kun myöhemmin muodostamme komponentin $B$,
emme mene käännettyä kaarta komponenttiin $A$,
koska komponentti $A$ on jo muodostettu.

\subsection{Esimerkki: Luolapeli}

Olemme pelissä luolastossa, joka muodostuu luolista ja niitä yhdistävistä
käytävistä. Jokainen käytävä on yksisuuntainen.
Jokaisessa luolassa on yksi aarre, jonka voimme ottaa mukaamme,
jos kuljemme luolan kautta.
Peli alkaa luolasta $a$ ja päättyy luolaan $b$.
Montako aarretta voimme saada, jos valitsemme parhaan
mahdollisen reitin?
On sallittua kulkea saman luolan kautta monta kertaa tarvittaessa.

\begin{figure}
\center
\begin{center}
\begin{tikzpicture}[scale=0.7]
\node[draw, circle] (1) at (0,0) {$1$};
\node[draw, circle] (2) at (2,-1) {$2$};
\node[draw, circle] (3) at (-2,0) {$3$};
\node[draw, circle] (4) at (0,-2) {$4$};
\node[draw, circle] (5) at (0,-4) {$5$};
\node[draw, circle] (6) at (-2,-2) {$6$};
\node[draw, circle] (7) at (-2,-4) {$7$};

\path[draw,thick,->] (1) -- (4);
\path[draw,thick,->] (4) -- (2);
\path[draw,thick,->] (2) -- (1);
\path[draw,thick,->] (4) -- (6);
\path[draw,thick,->] (1) -- (3);
\path[draw,thick,->] (3) -- (6);
\path[draw,thick,->] (6) edge [bend left] (7);
\path[draw,thick,->] (7) edge [bend left] (6);
\path[draw,thick,->] (4) -- (5);
\path[draw,thick,->] (5) -- (7);
\end{tikzpicture}
\end{center}
\caption{Luolasto, jossa on 7 luolaa ja 10 käytävää.
Haluamme kulkea luolasta 1 luolaan 7 keräten mahdollisimman paljon aarteita.}
\label{fig:luopel}
\end{figure}

Voimme mallintaa tilanteen verkkona, jonka solmut ovat luolia ja
kaaret ovat käytäviä. Haluamme löytää reitin solmusta $a$ solmuun $b$
niin, että kuljemme mahdollisimman monen solmun kautta.
Esimerkiksi kuva \ref{fig:luopel} näyttää verkkona luolaston, joka muodostuu
7 luolasta ja 10 käytävästä.
Oletetaan, että haluamme liikkua luolasta 1 luolaan 7
keräten mahdollisimman paljon aarteita.
Tällöin yksi mahdollinen reitti on
$1 \rightarrow 4 \rightarrow 2 \rightarrow 1 \rightarrow 3 \rightarrow 6 \rightarrow 7$,
jonka avulla saamme kaikki aarteet paitsi luolassa 5 olevan aarteen.
Ei ole olemassa reittiä, jota seuraamalla saisimme kaikki luolaston aarteet.

\begin{figure}
\center
\begin{center}
\begin{tikzpicture}[scale=0.7]
\node[draw, rectangle] (1) at (0,0) {$\{1,2,4\}$};
\node[draw, rectangle] (2) at (-4,-2) {$\{3\}$};
\node[draw, rectangle] (3) at (4,-2) {$\{5\}$};
\node[draw, rectangle] (4) at (0,-4) {$\{6,7\}$};

\path[draw,thick,->] (1) -- (2);
\path[draw,thick,->] (1) -- (3);
\path[draw,thick,->] (1) -- (4);
\path[draw,thick,->] (2) -- (4);
\path[draw,thick,->] (3) -- (4);
\end{tikzpicture}
\end{center}
\caption{Luolaston vahvasti yhtenäiset komponentit.}
\label{fig:luovah}
\end{figure}

Voimme ratkaista ongelman tehokkaasti määrittämällä ensin verkon
vahvasti yhtenäiset komponentit.
Tämän jälkeen meidän riittää löytää polku alkusolmun komponentista
loppusolmun komponenttiin niin, että komponenttien kokojen summa
on suurin mahdollinen.
Koska verkko on syklitön, tämä onnistuu dynaamisella ohjelmoinnilla.

Kuva \ref{fig:luovah} näyttää vahvasti yhtenäiset komponentit
esimerkkiverkossamme.
Tästä esityksestä näemme suoraan, että optimaalisia reittejä on kaksi:
voimme kulkea joko luolan 3 tai luolan 5 kautta.